\chapter{Photon Interaction Cross Sections and Sampling Techniques}
\label{ch:photon_interactions}
In the previous two chapters, the Monte Carlo random walk process for both
neutral and charged radiation was described. An important part of the random
walk process is the indirect sampling of the collision kernel, which requires
information on the differential interaction cross sections for the particle
of interest. In this chapter, four primary interactions of high energy photons 
with matter will be discussed: incoherent scattering, coherent scattering, 
pair production and the photoelectric effect. Photonuclear absorption, which can
become important in high energy coupled neutron-photon transport simulations, 
will be discussed briefly. In addition, sampling procedures that can be used 
with each differential interaction cross section will be discribed. All 
secondary particles other than photons will be neglected from the cross sections
and sampling procedures. 

\section{Incoherent Scattering}
Incoherent or Compton scattering occurs when a photon scatters off of an orbital
electron and loses some of its energy. The energy of the liberated electron
is equal to the energy lost by the photon minus the binding energy of the
shell the electron occupied. The differential incoherent scattering cross
section (per atom) is usually expressed as the product of the differential 
Klein-Nishina scattering cross section per electron and the incoherent 
scattering function. Klein and Nishina developed the differential cross section
under the assumption that the electron on which the photon scatters is free 
and at rest \citep{klein_uber_1929}. The incoherent scattering function is the 
correction to the Klein-Nishina scattering cross section from electron binding. 
The differential incoherent scattering cross section is shown below 
\citep{lux_monte_1991}. The value $r_e^2$ is the classical radius of the 
electron and $\alpha^{'}$ is the energy of the incident photon in units of 
electron rest energy.
\begin{align}
  \frac{d\sigma_{i.s.}(\alpha^{'},\theta,Z)}{d\Omega} & = 
  \frac{d\sigma_{K.N.}(\alpha^{'},\theta)}{d\Omega}S(y,Z) \nonumber \\
  & = \frac{r_e^2}{2}
  \frac{\left[1 + \cos{^{2}\theta} + \frac{\alpha^{'2}(1-\cos{\theta})^2}
                                  {1 + \alpha^{'}(1-\cos{\theta})}\right] }
  {\left[1 + \alpha^{'}(1-\cos{\theta}) \right]^2}
  S(y,Z)
  \label{eq:incoh_scat_theta}
\end{align}
The arguments of the scattering function are 
$y = sin(\frac{\theta}{2})/\lambda^{'}$ and $Z$, the atomic number. The 
differential incoherent cross section will sometimes be expressed in terms of 
the outgoing photon energy as well. The alpha variables are simply the photon
energy divided by the rest mass of the electron.
\begin{equation}
  \frac{d\sigma_{i.s.}(\alpha^{'},\theta,Z)}{d\Omega} = \frac{r_e^2}{2}
  \left(\frac{\alpha}{\alpha^{'}} \right)^2
  \left[ \frac{\alpha}{\alpha^{'}} + \frac{\alpha^{'}}{\alpha} - 1 + 
    \cos{^2\theta} \right] S(y,Z)
\end{equation}
The later form of the differential incoherent cross section is made possible 
by the one-to-one correspondance between the outgoing energy and outgoing 
direction, which can be found using conservation of energy and momentum and the 
assumption that the electron is free and at rest (see Appendix 
\ref{ch:appendix_B}). 
\begin{equation}
  E = \frac{E^{'}}{1 + \frac{E^{'}}{m_ec^2}(1 - \cos{\theta})}
\end{equation}
When binding effects cannot be neglected, which typically occurs when the 
incident photon energies are on the order of the electron binding energy, there 
will no longer be a one-to-one correspondance between the outgoing energy and 
the outgoing direction. Instead, there will be a distribution of outgoing 
energies that correspond to each outgoing direction. This phenomenon will be 
discussed more in the next section.

A PDF for the outgoing photon angle can be created by dividing the differential
incoherent cross section by the total incoherent cross section at the incoming
photon energy. Then, by reorganizing the PDF, the procedure for sampling an
outgoing photon direction and energy can be determined 
\citep{persliden_monte_1983}.
\begin{align}
  p(\alpha^{'},\theta,Z) & = \frac{1}{\sigma_{i.s.}(\alpha^{'},Z)}
  \frac{d\sigma_{i.s.}(\alpha^{'},\theta,Z)}{d\Omega} \nonumber \\
  & = \frac{S_{max}(y,Z) \sigma_{K.N.}(\alpha^{'})}{\sigma_{i.s.}(\alpha^{'},Z)}
  \left[ \frac{S(y,Z)}{S_{max}(y,Z)} \right]
  \left[ \frac{1}{\sigma_{K.N.}(\alpha^{'})} 
    \frac{d\sigma_{K.N.}(\alpha^{'},\theta)}
         {d\Omega} \right] \nonumber \\
  & = C(\alpha^{'},Z) R(y,Z) p_{K.N.}(\alpha^{'},\theta) 
\end{align}
The scattering function increases monotonically from 0 when $y=0$ to Z when
$y=\infty$ and therefore,
\begin{equation*} 
S_{max}(y,Z) = S(y_{max},Z).
\end{equation*}
The maximum value of $y$, which occurs when $\theta = \pi$ (corresponding to
back scattering), is simply the inverse wavelength of the incoming 
particle (usually in $cm^{-1}$). Now, to sample the outgoing direction, one 
first samples an angle from the PDF for Compton Scattering off of a free 
electron, $p_{K.N.}(\alpha,\theta)$. Then one uses the rejection function R(y,Z)
with the sampled angle to determine if it should be accepted or rejected. 
Several sampling techniques that can be used with the Klein-Nishina cross 
section will now be discussed.

For the purpose of sampling, it is useful to write the differential 
Klein-Nishina cross section in terms of a new variable whose inverse is the 
energy loss ratio.
\begin{align}
  \frac{1}{x} & = \frac{\alpha}{\alpha^{'}} \\
  & = \frac{1}{1+\alpha^{'}(1-\cos{\theta})}
\end{align}
The differential Klein-Nishina cross section can then be written as follows 
\citep{lux_monte_1991}.
\begin{align}
  \frac{d\sigma_{K.N.}(\alpha^{'},x)}{dx} & = K \left[ A + \frac{B}{x} +
    \frac{C}{x^2} + \frac{D}{x^3} \right] \\
  K & = \frac{\pi r_e^2}{\alpha^{'}} \nonumber \\
  A & = \frac{1}{\alpha^{'2}} \nonumber \\
  B & = 1 - \frac{2(\alpha^{'} + 1)}{\alpha^{'2}} \nonumber \\
  C & = \frac{1 + 2\alpha^{'}}{\alpha^{'2}} \nonumber \\
  D & = 1 \nonumber
\end{align}
The change of variables is carried out in Appendix \ref{ch:appendix_B}.

Now, a PDF for $x$ can be created if the differential Klein-Nishina cross 
section is divided by the total Klein-Nishina cross section, which can be found 
by integrating the differential Klein-Nishina cross section from $x_{min} = 1$ to
$x_{max} = 1 + 2 \alpha^{'}$\footnote{The equation found in Lux and Koblinger's
text book contains an error which is described in Appendix \ref{ch:appendix_B}}.
\begin{equation}
  \sigma_{K.N.}(\alpha^{'}) = 2\pi r_e^2 \left( \frac{1+\alpha^{'}}{\alpha^{'2}}
  \left[\frac{2+2\alpha^{'}}{1+2\alpha^{'}} - 
    \frac{ln(1+2\alpha^{'})}{\alpha^{'}} \right]
  + \frac{ln(1+2\alpha^{'})}{2\alpha^{'}} - 
  \frac{1+3\alpha^{'}}{(1+2\alpha^{'})^2} \right)
\end{equation}
At low photon energies, the evaluation of this equation for the Klein-Nishina 
cross sections can lead to numerical errors due to the near-cancellation between
logarithmic and algebraic terms \citep{lux_monte_1991}. An empirical formula
was created that is correct to within $1.3\%$ up to 100 MeV
\citep{hastings_approximations_1955}. Due to the sampling techniques that will
be used, the total Klein-Nishina cross section will only need to be evaluated
when the incoming photon energy is above about $1.4$ MeV. Therefore there is no 
need to use the empirical formula.

The PDF for $x$ can be defined as follows.
\begin{align}
  p_{K.N.}(\alpha^{'},x) & = 
  \begin{cases}
    H \cdot \left[ A + \frac{B}{x} + \frac{C}{x^2} + \frac{D}{x^3} \right]
    & \text{if } 1 \leq x \leq 1 + 2 \alpha^{'} \\
    0 & \text{o.w.}
  \end{cases} \\
  H & = \frac{K}{\sigma_{K.N.}(\alpha^{'})} \nonumber 
\end{align}
A direct inversion of this PDF is not possible. However, this PDF can still be
sampled directly by using a combination of the probability mixing method and the
inverse CDF method \citep{koblinger_direct_1975}\footnote{For a description of 
all sampling methods that one can use, please refer to refs. \citep{koblinger_direct_1975, lux_monte_1991, spanier_monte_1969, blomquist_assessment_1983}.}. To
use these two sampling techniques, the PDF must be split into four terms. 
\begin{align}
  p_{K.N.}(\alpha^{'},x) = p_1(\alpha^{'},x) + p_2&(\alpha^{'},x) +
  p_3(\alpha^{'},x) + p_4(\alpha^{'},x) \\
  p_1(\alpha^{'},x) & = HA \nonumber \\
  p_2(\alpha^{'},x) & = \frac{HB}{x} \nonumber \\
  p_3(\alpha^{'},x) & = \frac{HC}{x^2} \nonumber \\
  p_4(\alpha^{'},x) & = \frac{HD}{x^3} \nonumber 
\end{align}
Now the $i^{th}$ term can be selected with the following probability.
\begin{equation}
  p_i = \int_1^{1+2\alpha^{'}} p_i(\alpha^{'},x) dx
\end{equation}
For the four functions above, the probabilities of being selected are shown
below. Take note that these probabilities contain the total Klein-Nishina
cross section.
\begin{equation}
  p_i = 
  \begin{cases}
    \frac{2H}{\alpha^{'}} & \text{if } i = 1 \\
    H \left[1 - \frac{2+2\alpha^{'}}{\alpha^{'2}} \right] ln(1 + 2\alpha^{'}) 
    & \text{if } i = 2 \\
    \frac{2H}{\alpha^{'}} & \text{if } i = 3 \\
    \frac{H}{2} \left[1 - \frac{1}{(1+2\alpha^{'})^2} \right] & \text{if } i = 4
    \end{cases}
\end{equation}
In order for this method to work, all of the probabilities must be positive.
Therefore, $p_2$ and $p_4$ impose limits on the values of $\alpha^{'}$ where this
method can be used. The probability $p_2$ is the more restrictive of the two.
When the value of $\alpha^{'}$ is less than $(1 + \sqrt{3})$, $p_2$ will be 
negative and the method cannot be used. This corresponds to a photon energy of
about $1.4$ MeV. Below this energy another method must be used, which will be 
discussed shortly. Once an $i$ has been selected, the value of $x$ is sampled
from the inverse CDF of $p_i(\alpha^{'},x)$. The inverse CDFs for each of the
four functions is shown below. The uniform random number in the interval (0,1)
required by the inverse CDF method is represented by the variable 
$\varepsilon$.
\begin{align}
  P_1^{-1}(\alpha^{'},\varepsilon) & = 1 + 2\alpha^{'}\varepsilon \\
  P_2^{-1}(\alpha^{'},\varepsilon) & = (1 + 2\alpha^{'})^{\varepsilon} \\
  P_3^{-1}(\alpha^{'},\varepsilon) & = \frac{1 + 2\alpha^{'}}
  {1 + 2\alpha^{'}\varepsilon} \\
  P_4^{-1}(\alpha^{'},\varepsilon) & = \left[1 - \varepsilon\left(1 - 
    \frac{1}{(1 + 2\alpha^{'})^2} \right) \right]^{-\frac{1}{2}}
\end{align}
Evaluating the inverse CDF with a random number yields a value of x from the
Klein-Nishina differential cross section. The value of x can then be used to
determine the new energy and direction (relative to the current direction) of 
the photon.
\begin{align}
  \alpha & = \frac{\alpha^{'}}{x} \\
  \cos{\theta} & = 1 + \frac{1-x}{\alpha^{'}} 
\end{align}

When the energy of the incoming photon drops below 1.4 MeV, the rejection
method can be used to sample values of x from the Klein-Nishina cross section.
Kahn developed a rejection sampling procedure that can be used for any incoming
photon energy. However the efficiency of the procedure is highest at lower 
energies \citep{lux_monte_1991, kahn_applications_1956}. This procedure is
shown in figure \ref{fig:kahn_rejection_sampling}.
\begin{figure}[t!]
  \begin{center}
    \def\svgwidth{300bp}
    \input{chapters/photon_interactions/Kahn_sampling_method.pdf_tex}
  \end{center}
  \caption{\textbf{Kahn's Rejection Sampling Procedure}.
    \textit{This sampling procedure is used to sample a value of x from the
    differential Klein-Nishina cross section. One can use this sampling
    procedure at any incoming particle energy, however the efficiency of the
    procedure degrades at higher enegies \citep{lux_monte_1991}.}}
  \label{fig:kahn_rejection_sampling}
\end{figure}

When the direct sampling method and the rejection method are used together, one 
can efficiently sample values from the Klein-Nishina cross section at any 
incoming photon energy. Figure \ref{} shows the efficiency of both Kahn's 
rejection sampling procedure and the combined sampling procedure. At around
1.4 MeV, the efficiency of the combined sampling procedure jumps to one because
of the switch to the direct sampling method. This is a considerable improvement
compared to only using Kahn's method. 

\section{Doppler Broadening of Incoherently Scattered Photons}
As mentioned in the previous section, the differential Klein-Nishina cross 
section assumes that the electron upon which the photon scatters is free
and at rest. Atomic electrons are bound and consequently cannot be at rest. 
When the energy of the incoming photon is much greater than the binding 
energies of the electrons in the atom, binding effects are negligable and the
incoherent cross section becomes simply the atomic number times the 
Klein-Nishina cross section. However, when the incoming photon energy is on the
order of a few hundred keV or lower, binding effects must be taken into account
\citep{namito_implementation_1994}. Taking binding effects into account results
in three significant changes to the simulated transport process. First, electron
binding results in a reduction in the total incoherent cross section, which
in turn results in larger mean free paths for the photon as it travels through
the medium. This occurs because part of the differential cross section is
suppressed because it would result in energetically impossible scattering 
events\footnote{Because the atomic system is a quantum-mechanical system, enough
energy must be given to the electron to free it. Otherwise a reaction will not
occur}. Second, the angular distribution of the scattered photon is 
modified, particurily in the forward direction. Finaly, the Compton-scattered
photon energy is broadened by the pre-collision motion of the electron 
\citep{namito_implementation_1994}. In other words, the one-to-one 
correspondence between the outgoing photon direction and energy is broken and 
for each outgoing direction, there is an associated distribution of outgoing 
photon energies. The first two changes were taken into account in the last
section by multiplying the differential Klein-Nishina cross section by the 
scattering function. In this section, the energy distribution will be dealt
with. 

To take into account the energy distribution, the double differential
incoherent scattering cross section must be used. Using the Born and impulse
approximations, Ribberfors was able to derive the double differential
incoherent scattering cross section, which is shown below 
\citep{ribberfors_x-ray_1983}. This double differential cross section is given 
for each subshell of the atom. The value $\alpha_c$ is called the Compton line, 
which is the outgoing photon energy (divided by the electron rest mass) 
corresponding to the outgoing scattering angle assuming that the electron was 
stationary and free. The variable $\alpha$ is the true outgoing photon 
energy resulting from the collision with a bound, moving electron.
\begin{equation}
  \left(\frac{d^2\sigma(\alpha,\alpha^{'},\theta,Z)}{d\Omega dE}\right)_i = 
  \frac{r_e^2}{2c\left|\vec{\alpha^{'}} - 
    \vec{\alpha}\right|} \left(\frac{\alpha}{\alpha^{'}}\right) 
  \left(\frac{\alpha^{'}}{\alpha_c} + \frac{\alpha_c}{\alpha^{'}} - 1 + 
  cos^2\theta \right) J_i(p_z,Z)
\end{equation}
The function $J_i(p_z,Z)$ is the Compton profile for the $i^{th}$ electron 
subshell for element $Z$. The argument $p_z$ is the projection of the 
electron's initial momentum on the scattering vector 
$\vec{\alpha^{'}} - \vec{\alpha}$. Because of the quantum mechanical nature of 
the atomic system, the electrons in each subshell have an associated 
probability distribution in momentum space, which is often denoted 
$n_i(\vec{p},Z)$. The Compton profile is simply the projection of this 
probability distribution along the scattering vector 
\citep{cooper_compton_1985}.
\begin{equation}
  J_i(p_z,Z) = \int_{p_x} \int_{p_y} n_i(p_x,p_y,p_z,Z)dp_xdp_y
\end{equation}

For the purposes of sampling from this double differential cross section it
will be more useful to do a change of variables from outgoing energy to 
electron momentum projection $p_z$. Using conservation of energy and momentum,
an equation for the electron momentum projection can be determined. This
derivation is shown in Appendix \ref{ch:appendix_B}. 
\begin{align}
  p_z & = m_ec \frac{\alpha - \alpha^{'} + \alpha^{'}\alpha(1 - \cos{\theta})}
  {\left|\vec{\alpha^{'}} - \vec{\alpha}\right|} \nonumber \\
  & = m_ec \frac{\alpha - \alpha^{'} + \alpha^{'}\alpha(1 - \cos{\theta})}
  {\sqrt{\alpha^{'2} + \alpha^{2} - 2\alpha^{'}\alpha^{'2}cos{\theta}}}
  \label{eq:pz}
\end{align}
The derivative of this equation with respect to the outgoing photon energy
is shown below.
\begin{equation}
  \frac{dp_z}{d\alpha} = m_ec \frac{1 + \alpha^{'}(1-\cos{\theta})}
  {\left|\vec{\alpha^{'}} - \vec{\alpha} \right|} - 
  p_z \frac{\left(\alpha - \alpha^{'}\cos{\theta} \right)}
  {\left|\vec{\alpha^{'}} - \vec{\alpha} \right|^2}
  \label{eq:diff_pz}
\end{equation}
Ribborfors suggested a simplification to this derivative based on some 
observations about the Compton profiles \citep{ribberfors_x-ray_1983}. First, 
the Compton profiles are even functions. Therefore, the odd moments of $p_z$ 
will be zero. Second, the Compton profiles drop off fairly rapidly so that
the first moment can be approximated as follows. The limit of integration
$p_{i,max}$ will be explained shortly.
\begin{equation}
  \langle p_z \rangle \approx \int_{-\infty}^{p_{i,max}} p_z J(p_z)dp_z = 0
\end{equation}
Therefore, the second term in equation \ref{eq:diff_pz} can be ignored because
it will not contribute significantly to the total incoherent cross section\footnote{In the change of variables $\frac{d\alpha}{dp_z}$ is used. However, 
constructing a Maclaurin series in $p_z$ to first order will give the same result.}.
\begin{equation}
  \frac{dp_z}{d\alpha} = m_ec \frac{1 + \alpha^{'}(1-\cos{\theta})}
  {\left|\vec{\alpha^{'}} - \vec{\alpha} \right|}
  \label{eq:diff_pz_simp}
\end{equation}

Now the change of variables in the double differential cross section can be
completed \citep{ribberfors_x-ray_1983}.
\begin{align}
  \left(\frac{d^2\sigma(p_z,\theta,Z)}{d\Omega dp_z}\right)_i & = 
  \left(\frac{d^2\sigma(\alpha,\alpha^{'},\theta,Z)}{d\Omega dE}\right)_i 
  \frac{dE}{d\alpha}
  \frac{d\alpha}{dp_z} \nonumber \\ 
  & =  \frac{r_e^2}{2} \left(\frac{\alpha\alpha_c}{\alpha^{'2}}\right) 
  \left(\frac{\alpha^{'}}{\alpha_c} + \frac{\alpha_c}{\alpha^{'}} - 1 + 
  cos^2\theta \right) J_i(p_z,Z)
\end{align}

The differential incoherent scattering cross section from the previous section
can be recovered by integrating over all possible electron momentum projections
and by summing up the resulting differential cross section for each shell.
Because $\alpha$ and $p_z$ are related by equation \ref{eq:pz}, this integral
will be quite complicated. Fortunately, Ribberfors has shown that using the
approximation $\alpha_c \approx \alpha$ results in negligable errors in the 
total incoherent scattering cross section \citep{ribberfors_x-ray_1983}. The
limit of integration $p_{i,max}$ is the maximum electron momentum projection
allowed for a compton scattering reaction to occur. This maximum occurs when
the outgoing photon energy is equal to $E - E_{i,b}$, where $E_{i,b}$ is the 
binding energy for the particular subshell. Unfortunately, $p_{i,max}$ is a 
function of $\theta$ so this integral will still be complicated. 
\begin{align}
  \frac{d\sigma(\alpha^{'},\theta,Z)}{d\Omega} & = 
  \frac{r_e^2}{2} \left(\frac{\alpha_c}{\alpha^{'}}\right)^2 
  \left(\frac{\alpha^{'}}{\alpha_c} + \frac{\alpha_c}{\alpha^{'}} - 1 + 
  cos^2\theta \right) \sum_i \int_{-\infty}^{p_{i,max}} J_i(p_z,Z)dp_z \nonumber \\
  & = \left(\frac{d\sigma}{d\Omega}\right)_{K.N.} S^I(y,Z) \nonumber \\
  & = \left(\frac{d\sigma}{d\Omega}\right)_{K.N.} S(y,Z) \nonumber
\end{align}

The scattering function $S^I(y,Z)$ is the scattering function from the
impulse approximation. The scattering function that is given in most tables
and is recommended for use in Monte Carlo codes is based on the Waller-Hartree
theory. Both of these scattering functions have been shown to be in close 
agreement for many elements though, which is why a direct substitution is 
justified \citep{namito_implementation_1994}. 

The method proposed by Namito et al. for sampling an outgoing photon energy
from the double differential incoherent cross section will now be discussed
\citep{namito_implementation_1994}. The first step is to sample an outgoing
scattering angle from the differential incoherent cross section using the 
methods from the previous section. Next, the subshell containing the electron 
upon which the photon will scatter must be sampled. The most accurate
way to sample the subshell would be to create a discrete PDF based on the total
incoherent cross section for each subshell. This data isn't readily available in
the popular tables \citep{cullen_epdl97_1997}. The approximation used by Namito 
et al., which is only truely applicable when the incoming photon energy is much 
greater than the binding energy of the electron, is to sample the electron shell
based on the number of electrons present in each shell (given as $n_i$).
\begin{align}
  p(i) & = \frac{(\sigma_{i.c.})_i}{\sigma_{i.c.}} \nonumber \\
  & \approx \frac{\sigma_{K.N.}n_i}{\sigma_{K.N.}Z} \nonumber \\
  & \approx \frac{n_i}{Z} 
\end{align}
Finally, an outgoing photon energy must be sampled from the double differential 
incoherent cross section. A conditional PDF for the outgoing photon energy can 
be created by dividing the double differential incoherent cross section by the 
differential coherent cross section evaluated at a particular angle. The value
of $p_{i,max}$ is calculated using the value of $\theta$ that was sampled and
the substitution $\alpha = \alpha^{'}-\frac{E_{i,b}}{m_ec^2}$.
\begin{align}
  p_i(p_z,Z \text{ | } \theta) & = 
  \left(\frac{d\sigma_{i.s.}(\alpha^{'},\theta,Z)}{d\Omega} \right)_i^{-1}
  \left(\frac{d^2\sigma(\alpha,\alpha^{'},\theta,Z)}{d\Omega dp_z}\right)_i 
  \nonumber \\
  & = \left(\frac{\alpha_c\alpha}{\alpha^{'2}}\right)
  \left(\frac{\alpha^{'}}{\alpha_c}\right)^2
  \frac{J_i(p_z,Z)}{S_i(y,Z)} \nonumber \\
  & = (1 + \alpha^{'}\cos{\theta})\left(\frac{\alpha}{\alpha^{'}}\right)
  \left(\frac{J_i(p_z,Z)}{\int_{-\infty}^{p_{i,max}}J_i(p_z,Z)dp_z}\right) \nonumber\\
  & = C(\alpha^{'},\theta)R(\alpha,\alpha^{'})p(p_z,Z)
\end{align}
One must sample a value of $p_z$ from the PDF $p(p_z,Z)$ to determine the
outgoing photon energy. Once the outgoing photon energy has been determined
the rejection function $R(\alpha,\alpha^{'})$ is used to determine if the value 
should be accepted or rejected. The CDF corresponding to the PDF for $p_z$ is 
the following\footnote{Namito et al. actually recommend the CDF $\frac{\int_{0}^{p_z}J_i(x,Z)dx}{\int_{0}^{p_{i,max}}J_i(x,Z)dx}$, which appears to be an error.}.
\begin{equation}
P(p_z,Z) = \frac{\int_{-\infty}^{p_z}J_i(x,Z)dx}{\int_{-\infty}^{p_{i,max}}J_i(x,Z)dx}
\end{equation}
Because the Compton profiles are most commonly found in tabular form\footnote{Biggs et al. have compiled the Compton profiles for atomic numbers 1-102\citep{biggs_hartree-fock_1975}.}, a table method must be used to sample the value of 
$p_z$. One first finds the value of the CDF corresponding to the momentum 
projection $p_{i,max}$. Then one finds the momentum projection associated with 
the CDF value $\varepsilon\int_{-\infty}^{p_{i,max}}J_i(x,Z)dx$, where $\varepsilon$ 
is again a uniform random number. Using a standard binary search algorithm, 
this process can be done very efficiently.

The equation for the outgoing photon energy in terms of the momentum projection
is shown below. When both values are energetically possible, one of the values
must be randomly selected (with probability one half).
\begin{align}
  \alpha & = \frac{-b}{2a} \pm \frac{\sqrt{b^2 - 4ac}}{2a} \\
  a & = \left(\frac{p_z}{m_ec}\right)^2 - 
  \left(\frac{\alpha^{'}}{\alpha_c}\right)^2
  \nonumber \\
  b & = -2\alpha^{'}\left[\left(\frac{p_z}{m_ec}\right)^2\cos{\theta} + 
  \frac{\alpha^{'}}{\alpha_c}\right] \nonumber \\
  c & = \left(\frac{p_z}{m_ec}\right)^2 - 1 \nonumber
\end{align}


\section{Coherent Scattering}
Coherent or Rayleigh scattering occurs when a photon scatters off of an atom 
with negligable energy loss. It occurs rarely except for when low energy (keV)
photons pass through a high atomic number material 
\citep{lux_monte_1991}. The differential coherent cross section (per atom) is 
usually expressed as a product of the classical Thompson differential 
cross section per electron and the atomic form factor squared. The differential 
coherent cross section is shown below \citep{lux_monte_1991}. The value $r_e$ 
is the classical radius of the electron. The arguments of the atomic form factor
are $y = \sin{\frac{\theta}{2}}/\lambda^{'}$ and Z, the atomic number. 
\begin{align}
  \frac{d\sigma_{c.s.}(\alpha^{'},\theta,Z)}{d\Omega} & = 
  \frac{d\sigma_{Th.}(\theta)}{d\Omega}F^2(y,Z) \nonumber \\
  & = \frac{r_e^2}{2}(1 + cos^2\theta)F^2(y,Z)
\end{align}

A PDF for the outgoing photon angle can be created by dividing the differential
coherent cross section by the total coherent cross section at the incoming
photon energy. This PDF can be reorganized in an analogous way to the PDF for 
the incoherent scattering cross section. 
\begin{align}
  p(\alpha^{'},\theta,Z) & = \frac{1}{\sigma_{c.s.}(\alpha^{'},Z)}
  \frac{d\sigma_{c.s.}(\alpha^{'},\theta,Z)}{d\Omega} \nonumber \\
  & = \frac{F^2_{max}(y,Z) \sigma_{Th.}(\theta)}{\sigma_{c.s.}(\alpha^{'},Z)}
  \left[\frac{F^2(y,Z)}{F^2_{max}(y,Z)}\right]
  \left[\frac{1}{\sigma_{Th.}(\theta)} \frac{d\sigma_{Th.}(\theta)}{d\Omega}
  \right] \nonumber \\
  & = C(\alpha^{'},Z)R(y,Z)p_{Th.}(\theta) \nonumber
\end{align}
Values of the scattering angle can be sampled from the PDF of the classical 
Thompson differential scattering cross section using a combination of the 
probability mixing and inverse CDF methods. Unfortunately, the rejection
function R(y,Z) created from the atomic form factor that would be used with this
method has a very low efficiency. Figure \ref{} shows the efficiency of this 
sampling procedure for aluminum and lead. 

Another sampling procedure, which is much more efficient, exists
\citep{persliden_monte_1983}. This procedure requires that the PDF for the 
outgoing photon angle be changed to a PDF in terms of the atomic form factor 
argument squared.
\begin{align}
  y^2 & = \frac{sin^2\left(\frac{\theta}{2} \right)}{\lambda^{'2}} \\
  dy^2 & = \frac{\sin{\theta}}{2\lambda^{'2}} d\theta \nonumber
\end{align}
\begin{align}
  p(\alpha^{'},\theta,Z)d\theta & = p(\alpha^{'},\theta,Z)d\Omega \nonumber \\
  & = \frac{\pi r_e^2}{\sigma_{c.s.}(\alpha^{'},Z)}(1 + cos^2\theta) F^2(y,Z) 
  \sin{\theta} d\theta \nonumber \\
  p(\alpha^{'},y^2,Z)dy^2 & = p(\alpha^{'},\theta,Z)d\theta \nonumber \\
  & = \frac{2\pi r_e^2 \lambda^{'2}}{\sigma_{c.s.}(\alpha^{'},Z)}
  (1 + cos^2\theta)F^2(y,Z) dy^2 
\end{align}
Now the PDF in terms of the atomic form factor argument squared can be 
reorganized into a PDF for sampling the atomic form factor argument squared
and a rejection function.
\begin{align}
  p(\alpha^{'},y^2,Z) & = 
  \frac{4\pi r_e^2 \lambda^{'2} \int_0^{y_{max}^2}F^2(y,Z)dy^2}
  {\sigma_{c.s.}(\alpha^{'},Z)} \left[\frac{1+cos^2\theta}{2} \right]
  \left[\frac{F^2(y,Z)}{\int_0^{y_{max}^2}F^2(y,Z)dy^2} \right] \nonumber \\
  & = C(\alpha^{'},Z)R(\theta)p(y^2,Z)
\end{align}
Now, to sample the outgoing photon direction, one first samples a squared
argument from the PDF $p(y^2,Z)$. Then one uses the rejection function 
$R(\theta)$ with the angle corresponding to the squared argument
\begin{equation}
  \cos{\theta} = 1 - 2\lambda^{'2}y^2,
\end{equation}
to determine if it should be accepted or rejected.

The CDF that would be used to sample a squared argument is simply
\begin{equation}
  P(y^2,Z) = \frac{\int_0^{y^2}F^2(x,Z)dx^2}{\int_0^{y_{max}^2}F^2(x,Z)dx^2}.
\end{equation}
Persliden recommended approximating the corresponding inverse CDF with 
polynomials and to then use the polynomials for sampling 
\citep{persliden_monte_1983}. Another method that can be used is to create a 
new data table for each element that gets stored along with all of the other 
cross section tables. The independent variable of this table is $y^2$ and the
dependent variable is $\int_0^{y^2}F^2(y,Z)dy^2$. To sample from this table, one
finds the dependent value associated with $y_{max}^2$. Then one finds the 
independent value associated with the value 
$\varepsilon \int_0^{y_{max}^2}F^2(y,Z)dy^2$, where $\varepsilon$ is a random 
number. By using a standard binary search, this method can be done very 
efficiently. In addition, the computation of the table only needs to be done 
once for each element.

Figure \ref{} shows the efficiency of Persliden's sampling method and 
the efficiency of the naive method that was discussed first. 

\section{Pair Production}
Pair production occurs when the photon interacts with the electric field of an
atom (screened nuclear field) producing an electron-positron pair. It can be
subdivided into two separate phenomemon depending on the state of the atom
after the interaction \citep{hubbell_pair_1980}. Coherent pair production occurs
when the entire atom recoils from the interaction without any internal 
excitation. Incoherent pair production occurs when the atom is left in an 
excited or ionized state. Pair production resulting in the ionization of an 
atom is often called triplet production because three outgoing particles (two 
electron and on positron) result from the interaction. However, triplet 
production is often used to refer to incoherent pair production (excitation and
ionization). 

The threshold for coherent pair production is $2m_ec^2$. The derivation of this
threshold can be found in Appendix \ref{ch:appendix_B}. The threshold for 
incoherent pair production is only slightly higher than the threshold for 
coherent pair production \citep{hubbell_pair_1980}. This is significantly 
different than the threshold for pair production in the field of a free 
electron (triplet production), which is $4m_ec^2$. However, the cross section 
for incoherent pair production below $4m_ec^2$ is very small and is usually 
neglected in data tables \citep{hubbell_pair_1980}. 

When one is conducting coupled photon-electron transport, the energy and 
direction of the outgoing electron and positron must be sampled. This can
be done using the Bethe-Hetler expression for the coherent pair production
cross section \citep{mukhin_experimental_1987, salvat_physics_2001}. This
cross section and the methods for sampling from it will be discussed in 
section \ref{}. When one is only interested in the transport of photons in 
the system, the pair production cross section can be formulated differently. 
Several assumptions will be made in the formulation of this approximate
cross section. The first is that the electron and positron deposit all of 
their energy locally (they stop immediately after being created). Since the 
mean free paths of electrons is generally much smaller than photons, this
assumption is usually acceptable. The second is that the positron will only
annihilate once its kinetic energy is essentially zero. In other words, the
annihilation of positrons in flight will be neglected. This results in a very
simple (isotropic) angular distribution of the annihilation photons. The final 
approximation is that the directions of the resulting annihilation photons are 
uncorrelated with the initial photon direction. Because of the tortorous path 
taken by a typical electron or positron, this approximation is usually 
acceptable as well. The simplified treatment allows the pair production and 
triplet production cross sections to be combined into a single cross section. 
The resulting double differential pair production is as follows 
\citep{gabler_amos_2006}.
\begin{equation}
  \frac{d^2\sigma_{p.p.}(E^{'},Z)}{d\Omega dE} = \frac{2 [\sigma_{p.p.}(E^{'},Z) 
    + \sigma_{t.p.}(E^{'},Z)] \delta(E - m_ec^2)}{4\pi}
\end{equation}
Note that the double differential pair production cross section contains the 
factor two to denote that two photons are created from the reaction. 

As discuss in chapter \ref{ch:particle_transport} a single outgoing photon can
be followed as long as its weight is multiplied by two. Otherwise two photons
must be followed. Both outgoing photons will have an outgoing energy equal to
$m_ec^2$ as indicated by the delta function in the pair production cross 
section. The first photon will then have its outgoing direction sampled from
an isotropic distribution. The direction of the second photon will then simply
be the opposite direction of the first photon in accordance with conservation
of momentum (and under the assumption that the positron-electron pair that
annihilated had negligable momentum).

\section{The Photoelectric Effect}
The photoelectric effect occurs when a photon of energy E is absorbed by a
target atom, which makes a transition to an excited state or becomes ionized.
When ionization occurs, the ejected electron will have energy equal to the 
absorbed photon minus the binding energy of the electron. Atomic relaxation
will occur after the vacancy has been created. This process results in the 
emission of Auger electrons and x-rays. The x-rays may be of significance to
low energy photon simulations, especially when high atomic number elements are
present. It is therefore necessary to conduct a more detailed treatment of the
photoelectric effect in photon transport codes. Salvat et al. recommended a 
procedure in which only the K and L shells are treated in a detailed way
\citep{salvat_physics_2001}. The outer shells are treated together. The PDF
for selecting the shell containing the electron that is ejected is as follows.
\begin{align}
  p(i) = 
  \begin{cases}
    \frac{\sigma_{i,pe}(E^{'},Z)}{\sigma_{pe}(E^{'},Z)} & \text{for K, L1, L2 or L3 shells} \\
    1 - p_K - p_{L1} - p_{L2} - p_{L3} & \text{o.w.}
  \end{cases}
\end{align}

When ionization occurs in the K or L shell, the energy of the ejected electron
is equal to the energy of the photon minus the binding energy of the particular
shell. While this is also true for the outer shells, the binding energies are
small enough that they can be neglected. If coupled photon-electron transport
is not being conducted than all emitted electrons are ignored. When the K or
L shells become ionized, the subsequent atomic relaxation has the possibility
to result in the emission of x-rays with energy that can be significant
depending on the calculation being done and must therefore be simulated. The
emission of particles from atomic relaxation resulting from vacancies in the
outer shells will always be ignored because of the low energy of these 
particles. The direction of all particles emitted from the relaxation process
should be sampled from an isotropic distribution \citep{salvat_physics_2001}.

\section{Photonuclear Absorption}
Photonuclear absorption occurs when a gamma ray is absorbed by the nucleus of
an atom resulting in the emission of one or more neutrons, charged particles or
additional gamma rays. These reactions generally have small cross-sections
and are usually neglected in photon transport codes. However, the cross sections
do exhibit broad resonances in the 12 to 24 MeV energy range 
\citep{lux_monte_1991}. For high energy coupled neutron-photon transport this
interaction can be important to include. 

\section{Other Interaction}
There are many more interactions that can occur between a photon and an atom.
These processes include Delbr\"{u}ck scattering, Raman scattering\footnote{Raman scattering is an inelastic scattering process that leaves the target atom in an excited state instead of ionized \citep{cullen_epdl97_1997}.}, and 
photomeson production. All of these interactions contribute less than 1\% to the
total cross section and are therefore neglected \citep{lux_monte_1991}.

\section{Adjoint Incoherent Scattering}

\section{Doppler Broadening of Incoherently Scattered Adjoint Photons}

\section{Adjoint Coherent Scattering}

\section{Adjoint Pair Production}

\section{Adjoint Triplet Production}
