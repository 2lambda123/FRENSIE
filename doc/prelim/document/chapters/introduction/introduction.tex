\chapter{Introduction}
\label{ch:introduction}
The Monte Carlo method has a rich history going back as far back as Babylonian 
times. However, its use in the field of radiation transport began in the 1940s 
and can be attributed to the work of von Neumann, Ulam, Metropolis, Kahn, Fermi 
and their collaborators \citep{halton_retrospective_1970}. The first successful 
application of the method in the field of radiation transport coincided with 
the construction of the first digital computers \citep{lux_monte_1991}. Because 
computational resources were relatively scarce and expensive, the computer codes
implementing the Monte Carlo method to solve radiation transport problems were
full of approximations to both the physical models and the cross section data.
As the availability of computer resources increased in the 90s, it became 
feasible to do high fidelity Monte Carlo simulations with regard to both the 
physical models and the cross section data \citep{chucas_preparing_1994}. Today,
the Monte Carlo method is regarded as the gold standard of computational 
methods for solving radiation transport problems because all variables of 
interest (i.e. energy, direction and position) can be treated on a continuous
scale and because the problem geometry can be modeled completely. As 
computers continue to grow in size and speed, the Monte Carlo method will 
continue to be used for more and more challenging problems \footnote{Already, 
the Monte Carlo method is appearing in full reactor core simulation codes 
where it was once deemed too costly and inefficient to use 
\citep{hoogenboom_monte_2011}.}.

\section{The Monte Carlo Method}
\label{sec:monte_carlo_method}
The Monte Carlo method is a stochastic method in which samples are drawn from 
a parent population through sampling procedures governed by a set of 
probability laws. From the samples, statistical data is acquired and analyzed 
to make inferences about the parent population. 

In radiation transport problems, the system of interest is a collection of 
bounded regions which can contain one or more of the following: a material, a 
vacuum, a source, a detector. The parent population is the set of all possible 
radiation histories and the samples are histories drawn from this set. The 
particle history can be regarded as a random walk from a source region to a 
problem domain boundary or some other terminating location (i.e. absorption 
point). Each phase of the random walk is governed by a set of probability laws 
that are all related to the material interaction cross sections of the 
particular form of radiation. The portion of a random walk that passes through 
a finite detector region is recorded or scored. It must be noted that in the 
context of the Monte Carlo method, the terms source region and detector region 
do not necessarily refer to the physical analog of a radiation source and 
radiation detector. Any region where a random walk is started is referred to 
as a source region and any region where some portion of a random walk is 
recorded is refereed to as a detector region. The necessity of this generality 
is an important point of discussion. 

While radiation transport problems are typically solved by sampling radiation
histories that start in what can be regarded as a model of the physical source
and recorded in what can be regarded as a model of the physical detector, the
opposite can also be true. The process of sampling the starting point of a 
radiation history in the region analogous to a physical source is often called
a forward process. The probability laws used in a forward process can be 
derived from the radiation transport equation. The forward process is most 
effective when the detector region is large relative to the source region. As 
the detector region decreases in size, the probability of any given history 
passing through the detector region decreases until, for a point detector, the 
probability goes to zero \citep{spanier_monte_1969}. Fortunately, there is 
another process of sampling radiation histories where the starting point of a 
history is sampled in the region analogous to a physical detector. This process 
is refer-ed to in the literature as an adjoint or reverse process 
\citep{spanier_monte_1969, desorgher_implementation_2010}. In this process 
histories are instead recorded when they pass through the region analogous to a 
physical source. The same logic applies to this process in the sense that as 
the Monte Carlo detector region is large relative to the Monte Carlo source
region, the process will be more effective. The difference is that the 
Monte Carlo source now refers to the region analogous to the physical detector 
and the Monte Carlo detector now refers to the region analogous to the physical 
source. The probability laws that govern this process can be derived from the 
adjoint transport equation. The derivation of these probability laws will be a 
major focus of this report.

\section{Monte Carlo Codes Available Today}
\label{sec:monte_carlo_codes}
Most Monte Carlo codes available today focus on the forward process described
before. The forward process has been developed to a level where very few 
approximations are used. For instance, it is very common to treat radiation
histories on a continuous energy scale. This is also made possible by the very
accurate cross section data that is available. The reverse process has not been 
developed to the same level yet. Only a few Monte Carlo codes have implemented
the reverse process in a way that is relatively free of approximation. The
GEANT4 toolkit has implemented the reverse process on a continuous energy 
scale for electromagnetic radiation and charged particles. In this implementation there are still some approximations that lead to discrepancies in results 
compared to results from the forward process 
\citep{desorgher_implementation_2010}. FOCUS, a research code written by 
Hoogenboom, was the first code to implement the reverse process for neutrons
on a continuous energy scale. This code was not able to model the coupled 
reverse process for neutrons and photons \citep{hoogenboom_adjoint_1977}. Today 
only the commercial United Kingdom code MCBEND has implemented the reverse
process for neutrons \citep{grimstone_extension_1998}. The implementation in
MCBEND has some approximations that can be eliminated as well. Like FOCUS, 
MCBEND can not model the coupled reverse process for neutrons and photons. 
Table \ref{table:monte_carlo_codes_today} summarizes the continuous energy 
modeling capabilities of most Monte Carlo codes available today. Please note 
that two of the most powerful and popular codes, MCNP5 and MCBEND are not open 
source codes. GEANT4, though a software development kit and not a true code,
is the only open source software that has some continuous energy reverse 
capabilities. Several codes, such as MCNP5 and MORSE have implemented the 
reverse process on a discrete or multigroup energy scale.

\begin{table}[ht]
\label{table:monte_carlo_codes_today}
  \caption{\textbf{Continuous Energy Capabilities of Monte Carlo Codes Available
      Today}.\textit{The final column shows the proposed capabilities of the 
      Forward-Adjoint Continuous Energy Monte Carlo (FACEMC) code}.}
  \centering
  \begin{tabular}{c c c c c c c c }
    \hline\hline
    Code & $n$ & $\gamma$ & $e^-$ & $p$ & $n^{\dagger}$ & $\gamma^{\dagger}$ & $e^{-\dagger}$ \\ [0.5ex]
    \hline
    EGS4 & - & $\surd$ & $\surd$ & - & - & - & -  \\
    EGSnrc & - & $\surd$ & $\surd$ & - & - & - & - \\
    ITS6 & - & $\surd$ & $\surd$ & - & - & - & - \\
    PENELOPE & - & $\surd$ & $\surd$ & - & - & - & - \\
    MORSE & - & - & - & - & - & - & - \\
    TART2005 & $\surd$ & $\surd$ & - & - & - & - & - \\
    MCNP5/6 & $\surd$ & $\surd$ & $\surd$ & - & - & - & - \\
    MCNPX & $\surd$ & $\surd$ & $\surd$ & $\surd$ & - & - & - \\
    GEANT4 & $\surd$ & $\surd$ & $\surd$ & $\surd$ & - & $\surd$ & $\surd$ \\
    MCBEND & $\surd$ & $\surd$ & $\surd$ & - & $\surd$ & - & - \\ [1ex]
    \hline
    FACEMC & $\surd$ & $\surd$ & $\surd$ & $\surd$ & $\surd$ & $\surd$ & $\surd$ \\ [1ex]
    \hline
  \end{tabular}
  \label{table:mccodes}
\end{table}

The main reason for the apparent lack of 
codes that have implemented the reverse process on a continuous energy scale is
the lack of available adjoint cross section data necessary for the reverse 
process. The popular ENDF libraries only supply cross section data for the 
forward process. In addition, most of the literature only discusses sampling 
procedures for the forward process based on differential cross sections. 
Fortunately, Hoogenboom has shown that both the total and 
differential adjoint cross sections can be derived from the forward cross 
sections. The calculation of these cross sections is costly, but only needs to 
be done once and can be done in the popular ENDF format
\citep{hoogenboom_adjoint_1977}. 

\section{The FACEMC Code}
\label{sec:research_outline}
To address the limitations of current Monte Carlo codes and to bring the adjoint
process up to the level of the forward process, the Forward-Adjoint Continuous 
Energy Monte Carlo (FACEMC) code will be developed along with any adjoint
Monte Carlo cross sections and sampling techniques that are currently lacking. 
This code will be open source to foster adoption by other researchers. As 
mentioned previously, the scope of this code will only encompass fixed source 
problems. The energy range over which the forward and adjoint neutron processes 
will be explored is $10^{-5}$ eV to 20.0 MeV. For the forward and adjoint photon
and electron processes, the energy range that will be explored is 1.0 keV to 
20.0 MeV. These energy ranges will be sufficient to model a large number of 
problems important to both the nuclear engineering community and the medical 
physics community. Several such problems will be used to test the final version 
of the code. 

Most of this report will focus on the theory behind the Monte Carlo random 
walk process used to simulate the transport of the above types of radiation 
through a problem model. The first chapters will discuss the Monte Carlo random
walk process very generally. The later chapters will discuss the specifics 
associated with each type of radiation and its adjoint. Because some work has 
already been completed, specifically with adjoint photon transport, the final
chapter will discuss some preliminary results as well as the current state of
the FACEMC code.  

