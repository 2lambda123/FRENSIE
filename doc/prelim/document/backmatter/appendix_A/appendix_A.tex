\chapter{Alternative Derivation of the Adjoint Emission and Collision Densities}
\label{ch:appendix_a}

In chapter \ref{ch:adjoint_neutral_particle_transport} a FIESK for the adjoint
of the emission density was created. However, it was pointed out that the 
adjoint of the emission density was ``flux-like'' and therefore not ideal to
estimate via a Monte Carlo random walk process. It was also briefly mentioned
that a better function could be constructed if the adjoint of the emission
density was multiplied by some function $\Sigma(\vec{r},E)$, whose only 
necessary properties are that it is strictly positive and goes to zero in a 
vacuum. In the following sections, two functions will be used to modify the 
adjoint of the emission density. In addition, the adjoint of the collision 
density, which is also ``flux-like,'' will be modified in a similar way.

\section{The Adjoint Collision Density}
\label{sec:adjoint_collision_density}
To derive the adjoint collision density, the adjoint of the emission density
will be multiplied by $\Sigma(\vec{r},E) = \Sigma_T(\vec{r},E)$, which is the
total cross section. In the following equation,
$\xi^{\dagger}(\vec{r},E,\hat{\Omega})$ is the adjoint collision density.
\begin{equation}
  \xi^{\dagger}(\vec{r},E,\hat{\Omega}) = \Sigma_T(\vec{r},E)
  \chi^{\dagger}(\vec{r},E,\hat{\Omega})
  \label{eq:adj_collision_dens_to_adjoint_of_emission_dens}
\end{equation}

\begin{equation*}
  \begin{split}
  \xi^{\dagger}(\vec{r},&E,\hat{\Omega}) = \int a(\vec{r^{'}},E,\hat{\Omega}) 
    \Sigma_T(\vec{r},E) \tau(\vec{r^{'}},\vec{r},E,-\hat{\Omega}) dV^{'} + \\
    & \int\int\int  \frac{\Sigma_T(\vec{r},E)}{\Sigma_T(\vec{r^{'}},E^{'})}
      \tau(\vec{r^{'}},\vec{r},E,-\hat{\Omega}) 
    \Sigma_S(\vec{r^{'}},E \to E^{'},\hat{\Omega} \to \hat{\Omega^{'}})
    \xi^{\dagger}(\vec{r^{'}},E^{'},\hat{\Omega^{'}}) dE^{'}d\hat{\Omega^{'}}dV^{'}
  \end{split}
\end{equation*}
\begin{equation*}
  \begin{split}
  &\text{} \qquad = \int a(\vec{r^{'}},E,\hat{\Omega}) 
    T(\vec{r^{'}} \to \vec{r},E,-\hat{\Omega}) dV^{'} + \\
    & \int\int\int
      T(\vec{r^{'}} \to \vec{r},E,-\hat{\Omega})
      \frac{\Sigma_S(\vec{r^{'}},E \to E^{'},\hat{\Omega} \to \hat{\Omega^{'}})}
           {\Sigma_T(\vec{r^{'}},E^{'})}
    \xi^{\dagger}(\vec{r^{'}},E^{'},\hat{\Omega^{'}}) dE^{'}d\hat{\Omega^{'}}dV^{'}
  \end{split}
\end{equation*}

As mentioned in chapter \ref{ch:neutral_particle_transport}, the transport 
kernel is normalized, assuming that random walks that exit the domain of 
interest are terminated. Interestingly, the particles characterized by the 
adjoint collision density FIESK travel in the direction opposite of the 
variable $\hat{\Omega}$, which is indicated by the negative sign in front of
this variable in the transport kernel. The adjoint transport operator will be
defined as follows.
\begin{equation}
  T^{\dagger}(\vec{r^{'}} \to \vec{r},E,\hat{\Omega}) = 
  T(\vec{r^{'}} \to \vec{r},E,-\hat{\Omega})
  \label{eq:adjoint_transport_kernel}
\end{equation}

Unfortunately, the part of the state transition kernel that describes the 
collisions is still poorly behaved. However, it can be normalized by 
introducing the factor $P^{\dagger}(\vec{r^{'}},E^{'},\hat{\Omega^{'}})$.
\begin{equation}
  P^{\dagger}(\vec{r^{'}},E^{'},\hat{\Omega^{'}}) = 
  \frac{\Sigma^{\dagger}(\vec{r^{'}},E^{'})}{\Sigma_T(\vec{r^{'}},E^{'})}
  \label{eq:adjoint_weight_factor}
\end{equation}
This factor will be called the adjoint weight factor and the function 
$\Sigma^{\dagger}(\vec{r^{'}},E^{'})$ will be called the total adjoint cross 
section.
\begin{equation}
  \Sigma^{\dagger}(\vec{r^{'}},E^{'}) = \int\int \sum_j \Sigma_{S,j}(\vec{r^{'}},E) 
      f_j(E \to E^{'}, \hat{\Omega} \to \hat{\Omega^{'}}) dE d\Omega
  \label{eq:adjoint_total_cross_section}
\end{equation}
Now the adjoint collision kernel, which is properly normalized, can be 
defined.
\begin{equation}
  C^{\dagger}(\vec{r^{'}},E^{'} \to E,\hat{\Omega^{'}} \to \hat{\Omega}) = 
  \sum_j \frac{\Sigma_{S,j}(\vec{r^{'}},E)}{\Sigma^{\dagger}(\vec{r^{'}},E^{'})}
  f_j(E \to E^{'},\hat{\Omega} \to \hat{\Omega^{'}})
  \label{eq:adjoint_collision_kernel}
\end{equation}

Using the adjoint transport kernel, adjoint collision kernel and adjoint
weight factor defined, the adjoint collision density state transition kernel
can be defined.
\begin{equation}
  \begin{split}
    N^{\dagger}(\vec{r^{'}} \to \vec{r},E^{'} \to E,\hat{\Omega^{'}} \to \hat{\Omega})
    = T^{\dagger}(\vec{r^{'}} \to \vec{r},E,\hat{\Omega})
    C^{\dagger}(\vec{r^{'}},&E^{'} \to E,\hat{\Omega^{'}} \to \hat{\Omega}) \\
    & \cdot P^{\dagger}(\vec{r^{'}},E^{'},\hat{\Omega^{'}})
  \end{split}
\end{equation}

The FIESK for the adjoint collision density can now be written in terms of 
these kernels. In this equation $S_c^{\dagger}(\vec{r},E,\hat{\Omega})$ is the
adjoint first collided source.
\begin{equation*}
  \begin{split}
    \xi^{\dagger}(\vec{r},&E,\hat{\Omega}) = \int a(\vec{r^{'}},E,\hat{\Omega}) 
    T^{\dagger}(\vec{r^{'}} \to \vec{r},E,\hat{\Omega}) dV^{'} + \\
    & \int\int\int
    N^{\dagger}(\vec{r^{'}} \to \vec{r},E^{'} \to E,\hat{\Omega^{'}} \to \hat{\Omega})
      \xi^{\dagger}(\vec{r^{'}},E^{'},\hat{\Omega^{'}}) dE^{'}d\hat{\Omega^{'}}dV^{'}
  \end{split}
\end{equation*}
\begin{equation*}
  \begin{split}
    \qquad & \qquad = S_c^{\dagger}(\vec{r},E,\hat{\Omega}) + \\
    & \int\int\int
    N^{\dagger}(\vec{r^{'}} \to \vec{r},E^{'} \to E,\hat{\Omega^{'}} \to \hat{\Omega})
      \xi^{\dagger}(\vec{r^{'}},E^{'},\hat{\Omega^{'}}) dE^{'}d\hat{\Omega^{'}}dV^{'}
  \end{split}
\end{equation*}

Based on the relationship between the adjoint collision density and the adjoint
of the emission density shown in equation 
\ref{eq:adj_collision_dens_to_adjoint_of_emission_dens}, the inner product must
now be defined as follows.
\begin{equation}
  I = \int\int\int \xi^{\dagger}(\vec{r},E,\hat{\Omega})
  \frac{S(\vec{r},E,\hat{\Omega})}{\Sigma_T(\vec{r},E,\hat{\Omega})}
  dV dE d\hat{\Omega}
\end{equation}
 
\section{The Adjoint ``Collision-Like'' Density}
Another adjoint event density, which is collision-like, can be found in the 
literature. Instead of multiplying the adjoint of the emission density by the
total cross section, it is multiplied by the total adjoint cross section.
\begin{equation}
  \mu^{\dagger}(\vec{r},E,\hat{\Omega}) = \Sigma^{\dagger}(\vec{r},E)
  \chi^{\dagger}(\vec{r},E,\hat{\Omega})
  \label{eq:adj_collision_like_dens_to_adjoint_of_emission_dens}
\end{equation}
The resulting FIESK for this adjoint collision-like density is shown below.
In this equation, the definition of the adjoint weight factor, adjoint
transport kernel, adjoint collision kernel and adjoint first collided 
source from the previous section will be used.
\begin{equation*}
  \begin{split}
  \mu^{\dagger}(\vec{r},&E,\hat{\Omega}) = \int a(\vec{r^{'}},E,\hat{\Omega}) 
    \Sigma^{\dagger}(\vec{r},E) \tau(\vec{r^{'}},\vec{r},E,-\hat{\Omega}) 
    dV^{'} + \\
    & \int\int\int  \frac{\Sigma^{\dagger}(\vec{r},E)}
    {\Sigma^{\dagger}(\vec{r^{'}},E^{'})}
      \tau(\vec{r^{'}},\vec{r},E,-\hat{\Omega}) 
    \Sigma_S(\vec{r^{'}},E \to E^{'},\hat{\Omega} \to \hat{\Omega^{'}})
    \mu^{\dagger}(\vec{r^{'}},E^{'},\hat{\Omega^{'}}) dE^{'}d\hat{\Omega^{'}}dV^{'}
  \end{split}
\end{equation*}
\begin{equation*}
  \begin{split}
  \text{} & \qquad \qquad = P^{\dagger}(\vec{r},E,\hat{\Omega}) \int
    a(\vec{r^{'}},E,\hat{\Omega}) 
    T(\vec{r^{'}} \to \vec{r},E,-\hat{\Omega}) dV^{'} + \\
    & \int\int\int P^{\dagger}(\vec{r},E,\hat{\Omega}) 
    T(\vec{r^{'}} \to \vec{r},E,-\hat{\Omega})
      \frac{\Sigma_S(\vec{r^{'}},E \to E^{'},\hat{\Omega} \to \hat{\Omega^{'}})}
           {\Sigma^{\dagger}(\vec{r^{'}},E^{'})}
    \mu^{\dagger}(\vec{r^{'}},E^{'},\hat{\Omega^{'}}) dE^{'}d\hat{\Omega^{'}}dV^{'}
  \end{split}
\end{equation*}
\begin{equation*}
  \begin{split}
    \text{} & \qquad \qquad = P^{\dagger}(\vec{r},E,\hat{\Omega}) 
    S_c^{\dagger}(\vec{r},E,\hat{\Omega}) +\\
    & \int\int\int P^{\dagger}(\vec{r},E,\hat{\Omega}) 
    T^{\dagger}(\vec{r^{'}} \to \vec{r},E,\hat{\Omega})
    C^{\dagger}(\vec{r^{'}},E^{'} \to E,\hat{\Omega^{'}} \to \hat{\Omega})
    \mu^{\dagger}(\vec{r^{'}},E^{'},\hat{\Omega^{'}}) dE^{'}d\hat{\Omega^{'}}dV^{'}
  \end{split}
\end{equation*}

For this adjoint collision-like density, the state transition kernel is the 
following. Interestingly, the only difference between this state transition
kernel and the state transition kernel for the adjoint collision density is
the placement of the adjoint weight factor. In this kernel, the adjoint
weight factor is a function of the particle's state after the collision where
it is a function of the particle's state before the collision in the adjoint
collision density state transition kernel.
\begin{equation}
  \begin{split}
    O^{\dagger}(\vec{r^{'}} \to \vec{r},E^{'} \to E,\hat{\Omega^{'}} \to \hat{\Omega})
    = P^{\dagger}(\vec{r},E,&\hat{\Omega}) 
    T^{\dagger}(\vec{r^{'}} \to \vec{r},E,\hat{\Omega}) \\
    & \cdot C^{\dagger}(\vec{r^{'}},E^{'} \to E,\hat{\Omega^{'}} \to \hat{\Omega})
  \end{split}
\end{equation}

Based on the relationship between the adjoint collision-like density and the
adjoint of the emission density shown in equation 
\ref{eq:adj_collision_like_dens_to_adjoint_of_emission_dens}, the inner product
must now be defined as follows.
\begin{equation}
  I = \int\int\int \mu^{\dagger}(\vec{r},E,\hat{\Omega})
  \frac{S(\vec{r},E,\hat{\Omega})}{\Sigma^{\dagger}(\vec{r},E,\hat{\Omega})}
  dV dE d\hat{\Omega}
\end{equation}

\section{The Adjoint Emission Density}
Using the procedure outlined in chapter \ref{ch:mc_methods}, the adjoint of the
collision density FIESK can be derived. The steps will not be shown, but the 
resulting equation can be seen in the next equation. The function 
$\psi^{\dagger}(\vec{r},E,\hat{\Omega})$ is the adjoint of the collision density.
\begin{equation*}
  \begin{split}
    \psi^{\dagger}(\vec{r},&E,\hat{\Omega}) = b(\vec{r},E,\hat{\Omega}) + \\
    & \int\int\int C(\vec{r},E \to E^{'},\hat{\Omega} \to \hat{\Omega^{'}})
    T(\vec{r} \to \vec{r^{'}},E^{'},\hat{\Omega^{'}})
    \psi^{\dagger}(\vec{r^{'}},E^{'},\hat{\Omega^{'}}) dV^{'}dE^{'}d\hat{\Omega^{'}}
  \end{split}
\end{equation*}
\begin{equation}
  \begin{split}
    \qquad \quad \text{  } & \qquad = \frac{a(\vec{r},E,\hat{\Omega})}
    {\Sigma_T(\vec{r},E)}+ \\
    & \int\int\int C(\vec{r},E \to E^{'},\hat{\Omega} \to \hat{\Omega^{'}})
    T(\vec{r} \to \vec{r^{'}},E^{'},\hat{\Omega^{'}})
    \psi^{\dagger}(\vec{r^{'}},E^{'},\hat{\Omega^{'}}) dV^{'}dE^{'}d\hat{\Omega^{'}}
  \end{split}
\end{equation}

The adjoint of the collision density also exhibits flux-like behavior in that
it will not got to zero in a vacuum. To derive the adjoint emission density,
which will have the desirable property that it goes to zero in a vacuum, the
adjoint of the collision density will be multiplied by the total cross section.
In the following equation $\theta^{\dagger}(\vec{r},E,\hat{\Omega})$ is the
adjoint emission density.
\begin{equation}
  \theta^{\dagger}(\vec{r},E,\hat{\Omega}) = \Sigma_T(\vec{r},E)
  \psi^{\dagger}(\vec{r},E,\hat{\Omega})
  \label{eq:adj_emission_dens_to_adj_of_collision_dens}
\end{equation}
The resulting FIESK for the adjoint emission density is shown below. In this
equation, the definition of the adjoint weight factor, adjoint transport 
kernel and adjoint collision kernel from section 
\ref{sec:adjoint_collision_density} will be used.
\begin{equation*}
  \begin{split}
    \theta^{\dagger}(\vec{r},E,\hat{\Omega}) = a(\vec{r},E,&\hat{\Omega}) + 
    \int\int\int \frac{\Sigma_T(\vec{r},E)}{\Sigma_T(\vec{r^{'}},E^{'})}
    C(\vec{r},E \to E^{'},\hat{\Omega} \to \hat{\Omega^{'}}) \\
    & \cdot \Sigma_T(\vec{r^{'}},E^{'})
    \frac{T(\vec{r} \to \vec{r^{'}},E^{'},\hat{\Omega^{'}})}
         {\Sigma_T(\vec{r^{'}},E^{'})}
    \psi^{\dagger}(\vec{r^{'}},E^{'},\hat{\Omega^{'}}) dV^{'}dE^{'}d\hat{\Omega^{'}}
  \end{split}
\end{equation*}
\begin{equation*}
  \begin{split}
    \quad \quad = a(\vec{r},E,&\hat{\Omega}) + 
    \int\int\int \Sigma_T(\vec{r},E) 
    C(\vec{r},E \to E^{'},\hat{\Omega} \to \hat{\Omega^{'}}) \\
    & \frac{T(\vec{r^{'}} \to \vec{r},E^{'},-\hat{\Omega^{'}})}
           {\Sigma_T(\vec{r},E^{'})}
    \psi^{\dagger}(\vec{r^{'}},E^{'},\hat{\Omega^{'}}) dV^{'}dE^{'}d\hat{\Omega^{'}}
  \end{split}
\end{equation*}
\begin{equation*}
  \begin{split}
    \quad \quad = a(\vec{r},E,&\hat{\Omega}) + 
    \int\int\int \sum_j \Sigma_{S,j}(\vec{r},E) 
    f_j(E \to E^{'},\hat{\Omega} \to \hat{\Omega^{'}}) \\
    & \frac{T(\vec{r^{'}} \to \vec{r},E^{'},-\hat{\Omega^{'}})}
           {\Sigma_T(\vec{r},E^{'})}
    \psi^{\dagger}(\vec{r^{'}},E^{'},\hat{\Omega^{'}}) dV^{'}dE^{'}d\hat{\Omega^{'}}
  \end{split}
\end{equation*}
\begin{equation*}
  \begin{split}
    & \qquad \qquad \qquad \quad = a(\vec{r},E,\hat{\Omega}) + \\
    & \int\int\int 
    C^{\dagger}(\vec{r},E^{'} \to E,\hat{\Omega^{'}} \to \hat{\Omega})
    P^{\dagger}(\vec{r},E^{'},\hat{\Omega^{'}})
    T^{\dagger}(\vec{r^{'}} \to \vec{r},E^{'},\hat{\Omega^{'}})
    \psi^{\dagger}(\vec{r^{'}},E^{'},\hat{\Omega^{'}}) dV^{'}dE^{'}d\hat{\Omega^{'}}
  \end{split}
\end{equation*}

For the adjoint emission density, the state transition kernel is the following.
Again, the only difference between this state transition kernel and the 
previous two is the placement of the adjoint weight factor. In this kernel,
the adjoint weight factor is a function of the particle's state before and
after the collision (specifically, the particle's final position and initial
energy and direction).
\begin{equation}
  \begin{split}
    M^{\dagger}(\vec{r^{'}} \to \vec{r},E^{'} \to E,\hat{\Omega^{'}} \to \hat{\Omega})
    = C^{\dagger}(\vec{r},E^{'} \to E,&\hat{\Omega^{'}} \to \hat{\Omega})
    P^{\dagger}(\vec{r},E^{'},\hat{\Omega^{'}})  \\
    & \cdot T^{\dagger}(\vec{r^{'}} \to \vec{r},E^{'},\hat{\Omega^{'}}) 
  \end{split}
\end{equation}

Based on the relationship between the adjoint emission density and the adjoint
of the collision density shown in equation 
\ref{eq:adj_emission_dens_to_adj_of_collision_dens}, the inner product must now
be defined as follows, where $S_c(\vec{r},E,\hat{\Omega})$ is the first
collided source.
\begin{equation}
  I = \int\int\int \theta^{\dagger}(\vec{r},E,\hat{\Omega})
  \frac{S_c(\vec{r},E,\hat{\Omega})}{\Sigma_T(\vec{r},E,\hat{\Omega})}
  dV dE d\hat{\Omega}
\end{equation}
