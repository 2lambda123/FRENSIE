\chapter{Conclusion}
The primary goals of this report were:
\begin{enumerate}
  \item to present a detailed background of the Monte Carlo method beyond what 
    is typically found in the literature,
  \item to use this background to develop the Monte Carlo random walk process
    for adjoint particles,
  \item to describe the adjoint cross sections and sampling procedures that 
    will be used to conduct random walks with adjoint photons.
  \item and to describe how all of this information will be incorporated into
    the FACEMC code. 
\end{enumerate}

While working to achieve these goals in this report several important points 
were made which must be summarized. In the following sections, the main points 
from each chapter will be given. Before finishing, all proposed future work will
also be enumerated. 

\section{Monte Carlo Methods for Fredholm Integral Equations}
The primary goal of chapter \ref{ch:mc_methods} was to give a very general
description of the Monte Carlo random walk process and to show how one can use
it to estimate solutions to a particular class of equations called FIESKs. 
The FIESK is an important equation with applications in radiation transport
problems. 

The Monte Carlo random walk process was shown to be governed by a set of PDFs 
that must be derived from the FIESK of interest. The process is very general 
and can in fact be applied to any FIESK as long as the PDFs derived from the 
FIESK have certain properties. 

The concept of an estimator was also discussed in this chapter. To summarize 
briefly, the estimator accumulates information from a random walk and uses
it to estimate the inner product of the solution of a FIESK and some known
function. The two estimators that were discussed were the termination estimator 
and the event estimator. The termination estimator estimates the solution of a
FIESK every time a random walk ends in the region of interest. The event
estimator estimates the solution of a FIESK every time a random walk has an
event in the region of interest. In general, these estimators become ineffective
when the region of interest is reduced to a single point of the phases space.

Finally, the estimation of the solution of a FIESK at a point was discussed. 
When this kind of estimate is desired the Dual FIESK and its associated Monte 
Carlo random walk process must be used. In the random walk process associated
with the Dual FIESK random walks start in a detector region (or region of
interest as its is sometimes referred to) and estimators accumulate information
about the random walk in the source region. There are also special estimators 
that exist which can be used to estimate the value of certain FIESKs at a point,
however these estimators are not applicable in general and were therefore not 
discussed. In general, if the size of the source region is larger than the size 
of the detector region, the Dual FIESK and its associated Monte Carlo random 
walk process should be used. This claim is based on the simple fact that a 
random walk will have a higher probability of entering a large region than a 
small region. 

\section{The Monte Carlo Random Walk Process for Radiation Transport}
The primary goal of chapter \ref{ch:particle_transport} was to show how to 
create a FIESK from the integro-differental form of the transport equation and 
to then construct a Monte Carlo random walk process using the general theory 
from chapter \ref{ch:mc_methods}. The idea was to take a very thorough approach 
to deriving the Monte Carlo random walk process for radiation transport even 
though most people are likely familiar with the process. Then, the same thorough
approach could be used to derive the Monte Carlo random walk process for 
adjoint radiation, which is much less familiar. 

Through a series of manipulations, it was shown in this chapter how to take
the integro-differential tranport equation and create a FIESK that describes
either the flux, the emission density or the collision density of a system. 
The flux FIESK was examined first and an associated Monte Carlo random walk
process was created. This process was shown to have several unfavorable
properties. By far the most unseemly was the potential for event locations
to be sampled in a vacuum. This property arose from the fact that the flux
does not go to zero in a vacuum. Because of these unfavorable properties
the flux FIESK and its associated random walk process are rarely used in
practice. 

The emission density FIESK and collision density FIESK were examined next. It
was shown that the state transition kernels that appear in the emission density
FIESK and collision density FIESK can be decomposed into two kernels: the
collision kernel, which describes the movement of particles through energy
and direction, and the transport kernel, which describes the movement of 
particles through space. It was also shown that the collision kernel can be
further decomposed into its constituent reactions. When one wants to handle
particle multiplication explicitly, which is usually done, this expansion of
the collision kernel must be done. A sampling procedure was also given for
sampling a particular reaction from the collision kernel.

Finally, the Monte Carlo random walk process for radiation transport was
derived. Due to the similarities between the emission density FIESK and the
collision density FIESK, solutions to both FIESKs can actually be estimated
using the same random walk process. To quickly summarize this combined process,
a random walk is always initiated in the problem source. The next collision 
point is then sampled from the transport kernel. At this new collision point
a new energy and direction is sampled from the collision kernel. If a
multiplying reaction was sampled additional random walks will start from this
collision point. A new collision point will then be sampled for the original
random walk and the process will continue until either an absorption reaction
is chosen or the random walk exits the problem domain. During this process
the emission density will always be estimated after a birth or after a particle
exits a collision. The collision density will always be estimated upon entering
a collision. 

One final note was also given regarding the calculation of material responses.
The material response is usually given as the inner product of the flux and a
material response function. The emission density or the collision density
can be used instead of the flux if a modified material response function is
used. 

\section{The Monte Carlo Random Walk Process for Adjoint Radiation Transport}
In chapter \ref{ch:adjoint_particle_transport} all of the theories and 
procedures from chapters \ref{ch:mc_methods} and \ref{ch:particle_transport}
were utilized to construct a Monte Carlo random walk process for adjoint
radiation transport. The thorough approach to deriving the Monte Carlo
random walk process for radiation transport in chapter 
\ref{ch:particle_transport} was particularly useful. Before this approach
was mimicked for adjoint radiation transport, the dual emission density
FIESK was constructed and examined.

Upon examination of the dual emission density FIESK it became clear that the 
\textit{adjoint of the emission density}, which the dual emission density
FIESK describes, was a ``flux-like'' quantity. Based on the observations
made in chapter \ref{ch:particle_transport} about the Monte Carlo random
walk process for the flux, it was determined that the dual emission density
FIESK should be avoided. A modification to the dual emission density FIESK was
also suggested to ameliorate its ``flux-like'' nature. It turns out that this
modification results in the adjoint emission density FIESK, however this idea
was not pursued further because it was desired to treat adjoint transport more
generally. 

As mentioned in chapter \ref{ch:particle_transport} a material response can
be calculated from the inner product of the flux and the material response
function. This material response function can also be calculated from the 
inner product of the source function and another quantity called the adjoint
flux, which is described by the adjoint transport equation. The exact 
derivation of the adjoint transport equation was not given, though it is 
actually very similar to the derivation of the dual FIESK from a FIESK given
in chapter \ref{ch:mc_methods}. 

Once the adjoint transport equation in integro-differential form was 
obtained, a set of manipulations similar to the ones shown in chapter
\ref{ch:particle_transport} were done to obtain a FIESK that described the
adjoint emission density or the adjoint collision density. The adjoint flux
FIESK was avoided due to the negative properties of FIESKs that describe 
``flux-like'' quantities. The state transition kernels that appeared in the
adjoint emission density FIESK and the adjoint collision density FIESK were
then examined.

Unlike the state transition kernels from the emission density FIESK and
collision density FIESK, the state transition kernels from the adjoint
emission density FIESK and adjoint collision density FIESK can actually be
decomposed into three terms: the adjoint collision kernel, which describes
the movement of adjoint particles through energy and direction, the adjoint
transport kernel, which describes the movement of adjoint particles through
space, and the adjoint weight factor.

The adjoint weight factor, which is simply the ratio of the total macroscopic
adjoint cross section to the total macroscopic cross section, is one of the 
very interesting and surprising properties of these state transition kernels. It
comes about from the construction of the adjoint collision kernel and the 
adjoint transport kernel, both of which are normalized to unity. The adjoint 
collision kernel is constructed with the total macroscopic adjoint cross
section while the adjoint transport kernel is constructed with the total
macroscopic cross section, which allows for their normalization to unity. The
total macroscopic adjoint cross section and the total macroscopic cross section
are in general not equal, which gives rise to the adjoint weight factor. 

An expansion of the adjoint collision kernel into its constituent adjoint
reactions and a sampling procedure for sampling a particular reaction from this
expanded adjoint collision kernel were also shown. This expansion and the 
associated sampling procedure is useful if one wants to model every
adjoint reaction separately, which will usually be the case. Unfortunately, 
the adjoint cross section for each individual reaction is often unknown and 
must be constructed from the associated forward cross section. The equations
for calculating the adjoint cross sections are an extremely important take-away 
from this report and will therefore be shown again:
\begin{equation*}
  \sigma^{\dagger}(E^{'}) = \int\int \sigma(E)c(E)
  p(E \to E^{'}, \hat{\Omega} \to \hat{\Omega}^{'})dEd\hat{\Omega}
\end{equation*}
\begin{equation*}
  p^{\dagger}(E^{'} \to E, \hat{\Omega}^{'} \to \hat{\Omega}) = 
  \frac{\sigma(E)c(E)p(E \to E^{'}, \hat{\Omega} \to \hat{\Omega}^{'})}
  {\sigma^{\dagger}(E^{'})}
\end{equation*}

An interesting property of adjoint radiation, which is apparent from both
the normalization of the adjoint collision kernel and the definition of the
adjoint cross section, is the lack of absorption reactions or multiplying
reactions. For absorption reactions $c(E)$ is zero and the resulting adjoint
cross section is zero. For multiplying reactions $c(E)$ gets integrated into
the adjoint cross section. These properties highlight the necessity of the
rigorous mathematical approach to deriving the Monte Carlo random walk process
for adjoint radiation since they are very unintuitive. 

Finally, the Monte Carlo random walk process for radiation transport was 
derived. Due to the similarities between the adjoint emission density FIESK 
and the adjoint collision density FIESK, solutions to both FIESKs can 
actually be estimated using the same random walk process. To quickly 
summarize this combined process, a random walk is always initiated in the
adjoint source (or detector). The next collision point is then sampled from
the adjoint transport kernel. At this new collision point a new energy and
direction is sampled from the adjoint collision kernel. A new collision point
will then be sampled and the process will continue. Due to the lack of an
absorption cross section, Russian roulette must be used to terminate these
random walks. During this process the adjoint emission density will always be
estimated after a birth or after an adjoint particle exits a collision. The
adoint collision density will always be estimated upon entering a collision. 

One final note was also given in this chapter regarding the calculation of 
material responses. The material response defined as the inner product of
the adjoint flux and the source can also be defined in terms of the adjoint
emission density or the adjoint collision density and a modified source
function.

\section{Photon Interaction Cross Sections and Sampling Techniques}
Chapter \ref{ch:photon_interactions} was intended to described completely
all photon and adjoint photon interactions and associated sampling procedures. 
The primary interactions that were discussed for photons were incoherent 
scattering (with and without Doppler broadening), coherent scattering, pair 
production and the photoelectric effect. For adjoint photons the interactions
that were discussed were adjoint incoherent scattering (with and without
Doppler broadening), adjoint coherent scattering and adjoint pair production.

The main take-away from this chapter is that nearly all photon phenomena 
can be modeled with either photons or adjoint photons with the same degree of
accuracy. The one phenomena that can't currently be accounted for by adjoint
photons is the release of low energy x-rays due to atomic relaxation after
a photoelectric effect event. In applications where low energy photons are
important this limitation could be a problem. Solutions to this problem will
be part of the future work. 

It must also be noted that this chapter presented some genuinely novel work:
the Doppler broadening of incoherently scattered adjoint photons. The
theoretical work that was presented still needs to be tested. However, if it
is indeed correct, the ability to model the Doppler broadening of incoherently 
scattered adjoint photons will be a capability unique to FACEMC. 

\section{Overview of FACEMC}
The main take-away from chapter \ref{ch:code_overview} is that FACEMC is
already in a state where problems can be modeled and data can be acquired. 
The first version of the code is currently being updated to make it faster
and easier to use. Adjoint neutron capabilities are also being added to the
code. 

An extensive validation plan was also presented in this chapter. Completion
of this plan should ensure that FACEMC is working properly.

\section{Future Work}
Though much work has been completed towards the development of FACEMC both in
terms of theory and coding, there is still much work to be done. In terms of
theory, the following items have been identified as needing more work:
\begin{itemize}
  \item The low energy x-ray emission problem for adjoint photons.
  \item The background work on neutron and adjoint neutron transport cross
    sections and sampling techniques.
\end{itemize}
In terms of coding, the following items must still be completed:
\begin{itemize}
  \item The coding of the second version of all major FACEMC components 
    outlined in chapter \ref{ch:code_overview}.
  \item The computation of adjoint neutron cross sections and storage in
    an HDF5 format library.
\end{itemize}
In terms of analysis work, the following items will be proposed:
\begin{itemize}
  \item Complete the FACEMC validation plan.
  \item Calculate the adjoint data required for brachytherapy treatment planning
    optimization using data from a patient.
  \item Calculate the adjoint data required for external beam treatment planning
    using a standard phantom.
  \item Run a full scale shutdown dose calculation for a fusion device using
    the R2SA method.
  \item Run a shielding problem using the adjoint neutron transport capabilities
    of FACEMC.
\end{itemize}
These final items should be challenging enough to test the capabilities of 
FACEMC.

To complete the items listed above in a timely fashion, the following timeline
has been created:
\begin{enumerate}
  \item Complete all theoretical work by September 1st, 2013.
  \item Complete all coding work by March 1st, 2014.
  \item Complete the validation plan by April 1st, 2014.
  \item Complete all of the the challenge problems by June 1st, 2014.
  \item Ideal thesis completion date: August 1st, 2014 - September 1st, 2014
\end{enumerate}
Based on experience gained writing the first version of FACEMC, this timeline
should be feasible. 

