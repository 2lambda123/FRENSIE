\chapter{Monte Carlo Methods for Fredholm Integral Equations}
\label{ch:mc_methods}
Monte Carlo processes are often used to solve quadrature problems. This is 
because a probability interpretation of integrals is quite natural, which in
turn allows one to develop statistical estimators for integrals rather easily
\citep{spanier_monte_1969}.
In this report, the focus will fall solely on the specific Monte Carlo
process that is used to estimate the solutions of Fredholm integral
equations of the second kind because, as will be shown in the following 
chapters, the transport equation can be written this way. Equation
\ref{eq:fredholm_int_eqn} is an example of a Fredholm integral equation
of the second kind.
\begin{equation}
  F(x) = S(x) + \int K(y \to x) F(y)dy
\label{eq:fredholm_int_eqn}
\end{equation}

\section{Monte Carlo Random Walk Process}
\label{sec:mc_random_walk_process}
The Monte Carlo process that is used to determine the solution of a Fredholm
integral equation of the second kind simulates the movement of entities from
state to state in some phase space. The process of moving some entity from
state to state is commonly refered to as a random walk processes. However,
the random walk process can also refer to the set of probability functions
that govern the movement of the entity. The derivation of this Monte Carlo
process is outside the scope of this report. It will only be proven that this
process does indeed recover the solution to the equation of interest given an
appropriate random variable. For a detailed derivation of the process from a 
mathematical and probabilistic point of view, please refer to ref. 
\cite{spanier_monte_1969}. 

Assume that the the solution to equation \ref{eq:fredholm_int_eqn} is desired.
The random walk process for the solution of a Fredholm integral equation of
the second kind is completely specified by the following probability
distribution functions (PDFs) \citep{spanier_monte_1969}. The variable $x$ 
represents one point, or state, in continuous phase space $\Gamma$. The 
variable $\alpha = (x_1,\ldots,x_{k-1},x_k)$ represents a typical random walk 
with $k$ equal to the number of events that have occured to termination. The 
PDF $p^1(x)$ characterizes the probability that the first event of a random 
walk will occur in state $x$. The PDF $p(y \to x)$ characterizes the probability
of a transition from state $y$ to state $x$. Finally, the probability $p(x)$
characterizes the probability of termination in state $x$. 
\begin{equation}
  \text{Random Walk: }
  \begin{cases}
    p^1(x) & = P(x_1 = x) \\
    p(y \to x) & = P(x_{n+1} = x \quad|\quad x_n = y, k > n)  \\
    p(x) & = P(k = n \quad|\quad x_n = x) 
  \end{cases}
  \label{eq:mc_random_walk_pdfs}
\end{equation}
The PDFs that define the random walk process have the following properties:
\begin{enumerate}
  \item $p^1(x) \geq 0$
  \item $\int_{\Gamma} p^1(x)dx = 1$
  \item $p(y \to x) \geq 0$
  \item $\int_{\Gamma} p(y \to x)dx = 1 - p(y) \leq 1 \quad \forall y \in\Gamma$
\end{enumerate}
The PDF $p^1(x)$ can be generalized to a PDF that characterizes the
probability of an entity having its $n^{th}$ event in state $x$. 
\begin{equation}
  P^n(x) = 
  \begin{cases} 
    P(x_n = x \quad|\quad k > n-1) & \text{if } n >= 2 \\
    P(x_1 = x) = p^1(x) & \text{if } n = 1 
  \end{cases}
  \label{eq:pn_pdf}
\end{equation}
To calculate the PDF in equation \ref{eq:pn_pdf}, the following recursion
relation may be used.
\begin{equation}
  P^n(x) = \int_{\Gamma} p(y \to x) P^{n-1}(y)dy
  \label{eq:pn_recursion_rel}
\end{equation}

A discrete random variable $X(x)$ can now be defined on the space of continuous
random walks, which represents the number density of events that happen in 
state $x$. The expected value of this discrete random variable can be calculated
easily.
\begin{eqnarray}
  E[X(x)] & = & 1 \cdot P^1(x) + 1 \cdot P^2(x) + \ldots \nonumber \\
  & = & \sum_{n=1}^{\infty} P^n(x)
  \label{eq:expec_coll_dens}
\end{eqnarray}

Finally, a continuous random variable $P(x)$ can be defined, which is equal to
$E[X(x)]$. This new random variable is called the continuous event density 
at $x$, though it is more commonly refered to simply as the event density.
Using this random variable for the event density, and equation
\ref{eq:pn_recursion_rel}, a proof will be shown that the Monte Carlo random
walk process outlined in equation \ref{eq:mc_random_walk_pdfs} will indeed
recover the solution of a Fredholm integral equation, which has the form shown
in equation \ref{eq:fredholm_int_eqn}. This proof assumes that $S(x)$ and
$K(y \to x)$ from equation \ref{eq:fredholm_int_eqn} have the same properties 
as $p^1(x)$ and $p(y \to x)$ respectively. 
\begin{eqnarray}
  P(x) & = & \sum_{n=1}^{\infty} P^n(x) \nonumber \\
  & = & p^1(x) + \sum_{n=2}^{\infty} \int_{\Gamma} p(y \to x) P^{n-1}(y)dy \nonumber\\
  & = & p^1(x) + \int_{\Gamma} p(y \to x) \sum_{n=2}^{\infty} P^{n-1}(y)dy \nonumber\\
  & = & p^1(x) + \int_{\Gamma} p(y \to x) \sum_{n=1}^{\infty} P^{n}(y)dy \nonumber\\
  & = & p^1(x) + \int_{\Gamma} p(y \to x) P(y)dy \nonumber \\
  & = & S(x) + \int_{\Gamma} K(y \to x) P(y)dy \nonumber \\
  P(x) & = & F(x) \nonumber
\end{eqnarray}

From the Fredholm integral equation of the second kind shown in equation
\ref{eq:fredholm_int_eqn}, it is clear that $S(x)$ acts as a source term.
However, the exact interpretation of this source term is ambiguous. The
source term can either characterize the birth states of entities or the
states where the first non-birth events occur. Depending on the interpretation 
of the source term, one must follow a set of steps to simulate a random walk.
Figure \ref{fig:random_walk_process_1} shows the steps that one would follow
to simulate a random walk if the source term is interpreted as characterizing
the birth states of entites, while figure \ref{fig:random_walk_process_2}
shows the steps that one would follow to simulate a random walk if the
source term is interpreted as characterizing the first non-birth states of
entities. Note that the only difference between the two procedures is the
ordering of the steps.
\begin{figure}[t!]
  \begin{center}
    \scalebox{0.5}{\includegraphics{chapters/monte_carlo_background/Monte_Carlo_Process_Emission.pdf} }
  \end{center}
  \caption{\textbf{Monte Carlo random walk simulation proceedure.} \textit{This
      procedure must be followed if the source term characterizes the birth
      states of entities.} }
  \label{fig:random_walk_process_1}
\end{figure}
\begin{figure}[t!]
  \begin{center}
    \scalebox{0.5}{
      \includegraphics{chapters/monte_carlo_background/Monte_Carlo_Process_Collision.pdf} }
  \end{center}
  \caption{\textbf{Monte Carlo random walk simulation procedure.} \textit{This
      procedure must be followed if the source term characterizes the first
      collision states of entities.} }
  \label{fig:random_walk_process_2}
\end{figure}

\section{Monte Carlo Inner Product Estimators}
\label{sec:mc_int_eqn_estimators}
In the previous section, it was mentioned that the Monte Carlo random walk
process will only recover the solution to a Fredholm integral equation of 
the second kind when the function $S(x)$ and $K(y \to x)$ have specific
properties. These properties can be relaxed if a new problem is defined. 
Instead of seeking a solution to a Fredholm integral equation of the second 
kind, the inner product $I$ defined in equation \ref{eq:inner_product} is 
sought. The function g(x) is a known function and the function F(x) is a 
solution to equation \ref{eq:fredholm_int_eqn}. 
\begin{equation}
  I = \int_{\Gamma} g(x)F(x)dx
  \label{eq:inner_product}
\end{equation}

The random walk procedure outlined in the previous section can be used to
estimate the value of this integral. The PDFs $p^1(x)$ and $p(y \to x)$ can be 
constructed by normalizing $S(x)$ and $K(y \to x)$ respectively. By using the
following random variable or estimator, the value of the inner product can 
be recovered. The variable $\alpha$ is a random walk beginning at $x_1$ and 
ending at $x_k$.
\begin{equation}
  \varepsilon(\alpha) = \frac{S(x_1)}{p^1(x_1)}w(x_1 \to x_w)\cdots 
  w(x_{k-1} \to w_k)\frac{g(x_k)}{p(x_k)}
  \label{eq:collision_estimator_1}
\end{equation}
The weight function $w(y \to x)$ is defined in the following equation.
\begin{equation}
  w(y \to x) = 
  \begin{cases}
    \frac{K(y \to x)}{p(y \to x)} & \text{if } p(y \to x) \neq 0 \\
    0 & \text{o.w.}
  \end{cases}
\end{equation}
This estimator reintroduces the magnitude of both the source and the transition
kernel. It will now be shown that the expected value of this estimator is 
equal to the value of the inner product. This will prove that this estimator is
unbiased \citep{spanier_monte_1969}.
\begin{eqnarray}
  E\left[\varepsilon(\alpha)\right] & = & \sum_{\alpha} 
  P(\alpha)\varepsilon(\alpha) \nonumber \\
  & = & \sum_{k=1}^{\infty} \int \cdots \int dx_1 \cdots dx_k p^1(x_1)
  p(x_1 \to x_2) \cdots p(x_{k-1} \to x_k)p(x_k) \cdot \nonumber \\
  & & \qquad \frac{S(x_1)}{p^1(x_1)}w(x_1 \to x_w)\cdots 
  w(x_{k-1} \to w_k)\frac{g(x_k)}{p(x_k)} \nonumber \\
  & = & \sum_{k=1}^{\infty} \int \cdots \int dx_1 \cdots dx_k S(x_1)K(x_1 \to x_2)
  \cdots K(x_{k-1} \to x_k)g(x_k) \nonumber \\
  & = & \int_{\Gamma} g(x)F(x)dx \nonumber  
\end{eqnarray}
If the function g(x) is only non-zero in some finite portion of the phase space
$\Gamma_p$, which will be the case for most practicle problems, only the random
walks that terminate in the phase space $\Gamma_p$ will contribute to the 
estimation of $I$. Another more efficient estimator exists, which estimates $I$
every time a random walk has an event in the phase space $\Gamma_p$. This 
estimator, defined in equation \ref{eq:collision_estimator_2}, is sometimes 
called the event estimator since every event can potentially contribute to the 
estimation of the value of interest. The value $W_m(\alpha)$ can be interpreted 
as the weight of the entity after the $m^{th}$ event of the random walk.
\begin{eqnarray}
  \eta(\alpha) & = & \sum_m^k \frac{S(x_1)}{p(x_1)}w(x_1 \to x_2) \cdots 
  w(x_{m-1} \to x_m) g(x_m) \nonumber \\
  & = & \sum_m^k W_m(\alpha) g(x_m)
  \label{eq:collision_estimator_2}
\end{eqnarray}
The proof that this estimator is an unbiased random variable is quite lengthy 
and requires a more in-depth knowledge of probability theory. Therefore, 
reference \cite{spanier_monte_1969} should be consulted for details on this 
proof. As the size of the phase space $\Gamma_p$ decreases, the event estimator 
will perform slightly better than the first estimator because each random walk 
will have more opportunities to contribute to the estimate of the inner product
$I$. 

One final estimator that will be discussed is the track length estimator.
Unlike the previous two estimators, contributions to the inner product $I$
are not made at event points but instead along a path between events points.
Therefore, this estimator will have the best performance of the three
estimators as the size of the phase space $\Gamma_p$ becomes small. In this 
estimator, defined in equation \ref{eq:track_length_estimator}, the function 
$W_m(\alpha)$ is the same that was defined in the event estimator. The function 
$h(x_m)$ is related to the function $g(x_m)$, however the exact relationship is 
problem dependent. The value $d_m$ is the length of the $m^{th}$ entity track 
(distance between states $x_m$ and $x_{m+1}$) that lies within the phase space 
$\Gamma_p$.
\begin{equation}
  \tau(\alpha) = \sum_{m=1}^k W_m(\alpha)h(x_m)d_m
  \label{eq:track_length_estimator}
\end{equation}

Note that as the size of the phase space $\Gamma_p$ decreases to a point, the 
function $g(x)$ becomes a delta function. All three of the previous estimators 
will be ineffective at estimating the value $I$ when that is the case. In the 
next section, this problem will be addressed specifically.

When the Monte Carlo random walk process is used to calculate the inner product
$I$, only a finite number of random walks can be simulated and therefore, only 
an estimate of the value $I$ can be obtained. The estimate will
correspond to the following equation, where $N$ is the number of random walks
conducted and $f(\alpha_i)$ is the value of the estimator used for random
walk $\alpha_i$.
\begin{equation}
  \bar{I} = \frac{1}{N} \sum_{i=1}^N f(\alpha_i)
  \label{eq:inner_product_estimate}
\end{equation}
Because there will only be a finite number of random walks simulated, there
will also be a standard deviation associated with this estimate of $I$. The 
standard deviation of the estimate is shown below
\begin{equation}
  s = \sqrt{\frac{1}{N-1}\sum_{i=1}^N f(\alpha_i)^2 - \frac{N}{N-1}\bar{I}}
  \label{eq:inner_product_stddev}
\end{equation}

One last quantity must be discussed, which characterizes the efficacy of
an estimator at obtaining estimates of the value $I$. This quantity, defined
in equation \ref{eq:fom}, is often refered to as the figure of merrit. The
basic theoretical problem of Monte Carlo methods is to select the best
estimators for a specific problem, which are unbiased and whose associated
variances are as small as possible \citep{spanier_monte_1969}. However, the 
computer time needed to evaluate the estimator is also an important 
consideration. The figure of merrit takes into account both of these factors
and can therefore be used to evaluate the efficacy of a particular estimator
with regard to a particular problem. Because the value $s^2/\bar{I}^2$ is
proportional to $N^{-1}$ and the computer time is proportional to $N$, the
figure of merrit is a constant.
\begin{equation}
  FOM = \frac{\bar{I}}{s^2t}
  \label{eq:fom}
\end{equation}

Before moving on, an assumption that was made, which has not been mentioned
until now, must be discussed. In order for any of the above estimators to 
converge to the value $I$, they must be bounded \citep{spanier_monte_1969}. This
means that the random walks must terminate after $k < \infty$ events with 
probability unity. If an integral operator $\hat{K} = \int_{\Gamma} K(y \to x)dy$
\footnote{$\hat{K} \cdot f = \int_{\Gamma} K(y \to x)f(y)dy$} is defined, then 
random walks will be finite if and only if the spectrum of $\hat{K}$ is 
contained on the open unit disk \citep{spanier_monte_1969}. In the context of 
neutron transport problems, this means that only subcritical problems can be 
handled. This is an acceptable limitation since this report will not address 
criticality problems.

\section{The Dual Problem for Calculating Inner Products Over Small Phase 
  Spaces}
\label{sec:dual_problems}
In the previous section, a problem arose when the value of an inner product
in a small portion of the phase space was desired. In the limit that this 
portion of the phase space becomes a point, the estimators become ineffective 
at obtaining the value of the inner product. In this section, a dual problem 
will be introduced in which the same Monte Carlo random walk process can be 
used but a different inner product is estimated. Because of the way that this
dual problem is formulated the value of this new inner product will have the 
same value as the original inner product. 

First consider an operator $H$ with the following property when operating
on the function P(x), where x again is a state in the continuous phase space
$\Gamma$. 
\begin{equation}
  H \cdot P(x) = P(x) - \int_{\Gamma} K(y \to x)P(y)dy
  \label{eq:forward_operator}
\end{equation}
Based on equation \ref{eq:fredholm_int_eqn}, $H \cdot P(x)$ is equal to $S(x)$.
If we assume that $P(x)$ is a real-valued function, the adjoint operator 
$H^{\dagger}$ is defined by the following identity 
\citep{lewis_computational_1993}. The function $P^{\dagger}(x)$ is the solution 
of a dual Fredholm integral equation of the second kind. 
\begin{equation}
  \langle P^{\dagger}H \cdot P \rangle = 
  \langle PH^{\dagger} \cdot P^{\dagger} \rangle
  \label{eq:forward_adjoint_ops}
\end{equation}
From this identity the dual Fredholm integral equation of the second kind can
be created.
\begin{eqnarray}
  \langle P^{\dagger}H \cdot P \rangle & = & \int_{\Gamma} \left[P^{\dagger}(x)P(x) -
  P^{\dagger}(x)\int_{\Gamma}K(y \to x)P(y)dy \right]dx \nonumber \\
  & = & \int_{\Gamma} \left[P^{\dagger}(x)P(x) -
  \int_{\Gamma}K(y \to x)P^{\dagger}(x)P(y)dy \right]dx \nonumber \\
  & = & \int_{\Gamma} P(x)P^{\dagger}(x)dx - 
  \int_{\Gamma}P(y)\int_{\Gamma}K(y \to x)P^{\dagger}(x)dxdy \nonumber \\
  & = & \langle PH^{\dagger} \cdot P^{\dagger} \rangle \nonumber
\end{eqnarray}
\begin{equation}
  H^{\dagger} \cdot P^{\dagger}(x) = P^{\dagger}(x) - 
  \int_{\Gamma}K(x \to y)P^{\dagger}(y)dy
  \label{eq:adjoint_operator}
\end{equation}
If the function $H^{\dagger} \cdot P^{\dagger}(x)$ is set equal to the function
$g(x)$ from equation \ref{eq:inner_product}, the dual Fredholm integral
equation can be defined.
\begin{equation}
  P^{\dagger}(x) = g(x) + \int_{\Gamma}K(x \to y)P^{\dagger}(y)dy
  \label{eq:dual_fredholm_int_eqn}
\end{equation}
With the dual equation defined in this way, the new inner product can also
be defined.
\begin{eqnarray}
  I & = & \langle PH^{\dagger} \cdot P^{\dagger} \rangle \nonumber \\
  & = & \langle Pg \rangle \nonumber \\
  & = & \langle P^{\dagger}H \cdot P \rangle \nonumber \\
  & = & \langle P^{\dagger}S \rangle
\end{eqnarray}
Now, in the limiting case where g(x) is a delta function, the Monte Carlo
random walk process for the dual equation, combined with any of the estimators 
described in the previous section, will be able to estimate the inner product. 
This is because delta functions can be handled easily as source terms but
are very challenging to handle with estimators. In general, if the portion of 
the phase space in which the source is defined is larger than the the portion 
of the phase space in which the function $g(x)$ is defined, the dual problem
will be be able to take advantage of estimators with higher figures of merrit. 
This is one of the primary motivations for studying the dual problem in this
report. It should also be noted that for the dual problem, the same state 
transition kernel is used, but the integration is over final states. This 
property of the dual problem will cause many interesting issues, which will be 
discussed in much more detail in the coming chapters.

