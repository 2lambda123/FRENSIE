\chapter{Deriving the Monte Carlo Random Walk Process for Adjoint Neutral Particle Transport Problems}
\label{ch:adjoint_neutral_particle_transport}
In chapter \ref{ch:mc_methods}, the dual problem for calculating inner products
over small sphase spaces was introduced. In this chapter, it will be shown how
the dual problem can be solved in the context of neutral particle transport 
problems. This will require both the derivation of the dual problems of interest
and their associated Monte Carlo random walk processes. Since, in the previous
chapter, the FEISKs that described the event densities were shown to have 
ideal random walk PDFs, their dual problems will be evaluated first. The 
random walk process for the more general dual problem, which is characterized
by the adjoint transport equation, will then be determined. 

\section{The Adjoint Emission Density FIESK}
Using the procedure outlined in chapter \ref{ch:mc_methods}, the dual emission
density FIESK will be constructed. The function that will solve this dual 
equation will be called the adjoint of the emission density. It is not called
the adjoint emission density because, as will be shown, this function does
not behave like an emission density (or in general an event density). First, 
a linear operator will be constructed from equation 
\ref{eq:emission_density_integral_eqn}, which is the emission density FIESK. 
\begin{equation}
  \begin{split}
    H_{\chi} \cdot &\chi(\vec{r},E,\hat{\Omega}) = 
    \chi(\vec{r},E,\hat{\Omega}) - \\
    & \int\int\int C(\vec{r},E^{'} \to E,\hat{\Omega^{'}} \to \hat{\Omega})
    T(\vec{r^{'}} \to \vec{r},E^{'},\hat{\Omega^{'}}) 
    \chi(\vec{r^{'}},E',\hat{\Omega^{'}}) dV^{'}dE^{'}E\hat{\Omega^{'}}
  \end{split}
\end{equation}
With the operator $H_{\chi}$ defined in this way it is clear from equation
\ref{eq:emission_density_integral_eqn} that 
$H_{\chi} = S(\vec{r},E,\hat{\Omega})$.

To define the adjoint operator, the following equality must hold. Where the 
brackets indicate integration over all phase space.
\begin{equation*}
  \langle \chi^{\dagger}H_{\chi} \cdot \chi \rangle = 
  \langle \chi H_{\chi}^{\dagger} \cdot \chi^{\dagger} \rangle
\end{equation*}
Since the operator $H_{\chi}$ is known, the left bracket will be expanded.
\begin{equation*}
  \begin{split}
    \langle &\chi^{\dagger}H_{\chi} \cdot \chi \rangle =
    \int\int\int \
    \chi^{\dagger}(\vec{r},E,\hat{\Omega}) \chi(\vec{r},E,\hat{\Omega})
    dV dE d\hat{\Omega} - \int\int\int \chi^{\dagger}(\vec{r},E,\hat{\Omega}) \\
    & \cdot \int\int\int 
    C(\vec{r},E^{'} \to E, \hat{\Omega^{'}} \to \hat{\Omega^{'}})
    T(\vec{r^{'}} \to \vec{r},E^{'},\hat{\Omega^{'}})
    \chi(\vec{r^{'}},E^{'},\hat{\Omega^{'}}) dV^{'}dE^{'}d\hat{\Omega^{'}}
    dV dE d\hat{\Omega}
  \end{split}
\end{equation*}
\begin{equation*}
  \begin{split}
    & \qquad \qquad = \int\int\int \
    \chi^{\dagger}(\vec{r},E,\hat{\Omega}) \chi(\vec{r},E,\hat{\Omega})
    dV dE d\hat{\Omega} - \int\int\int \chi(\vec{r^{'}},E^{'},\hat{\Omega^{'}}) \\
    & \cdot \int\int\int 
    C(\vec{r},E^{'} \to E, \hat{\Omega^{'}} \to \hat{\Omega^{'}})
    T(\vec{r^{'}} \to \vec{r},E^{'},\hat{\Omega^{'}})
    \chi^{\dagger}(\vec{r},E,\hat{\Omega}) dV dE d\hat{\Omega}
    dV^{'}dE^{'}d\hat{\Omega^{'}}
  \end{split}
\end{equation*}
From this last manipulation, the adjoint operator can be deduced, which is 
shown in the following equation.
\begin{equation}
  \begin{split}
    H_{\chi}^{\dagger} \cdot &\chi^{\dagger}(\vec{r},E,\hat{\Omega}) = 
    \chi^{\dagger}(\vec{r},E,\hat{\Omega}) - \\
    & \int\int\int T(\vec{r} \to \vec{r^{'}},E,\hat{\Omega}) 
    C(\vec{r^{'}},E \to E^{'},\hat{\Omega} \to \hat{\Omega^{'}})
    \chi(\vec{r^{'}},E',\hat{\Omega^{'}}) dE^{'}E\hat{\Omega^{'}}dV^{'}
  \end{split}
\end{equation}

In the previous chapter, the function $c(\vec{r},E,\hat{\Omega})$ was defined 
in equation \ref{eq:emission_response_function}, which should be used with 
the collision density to calculate a response. By forcing the adjoint 
operator acting on the adjoint of the emission density to equal the 
function $c(\vec{r},E,\hat{\Omega})$, the dual equation, or simply the adjoint
of the emission density FIESK can be created.
\begin{equation}
  \begin{split}
    \chi^{\dagger}(\vec{r},&E,\hat{\Omega}) = c(\vec{r},E,\hat{\Omega}) + \\
    &\int\int\int T(\vec{r} \to \vec{r^{'}},E,\hat{\Omega}) 
    C(\vec{r^{'}},E \to E^{'},\hat{\Omega} \to \hat{\Omega^{'}})
    \chi^{\dagger}(\vec{r^{'}},E',\hat{\Omega^{'}}) dE^{'}E\hat{\Omega^{'}}dV^{'}
  \end{split}
  \label{eq:adjoint_of_emission_density_integral_eqn}
\end{equation}

Unfortunately, in the integral in equation 
\ref{eq:adjoint_of_emission_density_integral_eqn}, the transport and collision 
kernels are integrated over what was previously the final states. Therefore,
the properties that were defined for these kernels in the previous chapter are 
no longer valid. In particular, the transport kernel is not normalized anymore
and the collision kernel is not equal to the non-absorption probability when
integrated over all final states (which were previously the initial states). In
general, the collision kernel is not even strictly less than or equal to one 
anymore when integrated over all final states. To try and simplify these 
kernels, the $\Sigma_T(\vec{r^{'}},E)$ terms in both the transport and collision
operators will be allowed to cancel each other out. The modified tranpsport
kernel will now be examined.
\begin{equation}
  \frac{T(\vec{r} \to \vec{r^{'}},E,\hat{\Omega})}{\Sigma_T(\vec{r^{'}},E)} = 
  exp\left[-\int_0^{|\vec{r^{'}} - \vec{r}|} 
    \Sigma_T(\vec{r^{'}} - R^{'} \hat{\Omega},E)dR^{'} \right]
  \frac{\delta \left(\Omega - \left[\frac{\vec{r^{'}} - \vec{r}}
      {|\vec{r^{'}} - \vec{r}|}\right]\right)}
       {|\vec{r^{'}} - \vec{r}|^2}
  \label{eq:unnormalized_adjoint_transport_kernel}
\end{equation}
Based on the argument of the delta function, 
$\hat{\Omega} = \frac{\vec{r^{'}} - \vec{r}}{|\vec{r^{'}} - \vec{r}|}$ and 
therefore, $\vec{r^{'}} = \vec{r} + \hat{\Omega}|\vec{r^{'}} - \vec{r}|$. This
equation for $\vec{r^{'}}$ can be substituted back into the equation
\ref{eq:unnormalized_adjoint_transport_kernel}.
\begin{equation*}
  \frac{T(\vec{r} \to \vec{r^{'}},E,\hat{\Omega})}{\Sigma_T(\vec{r^{'}},E)} = 
  exp\left[-\int_0^{|\vec{r^{'}} - \vec{r}|} 
    \Sigma_T \left(\vec{r} + \left[|\vec{r^{'}} - \vec{r}| - R^{'} \right] 
    \hat{\Omega},E \right) dR^{'} 
    \right] \frac{\delta \left(\Omega - \left[\frac{\vec{r^{'}} - \vec{r}}
      {|\vec{r^{'}} - \vec{r}|}\right]\right)}
       {|\vec{r^{'}} - \vec{r}|^2}
\end{equation*}
A new variable of integration can be defined to simplify the exponent. Note
that when $R^{'} = 0$, $R^{''} = |\vec{r^{'}} - \vec{r}|$ and when 
$R^{'} = |\vec{r^{'}} - \vec{r}|$, $R^{''} = 0$.
\begin{eqnarray}
  R^{''} & = & |\vec{r^{'}} - \vec{r}| - R^{'} \nonumber \\
  dR^{''} & = & -dR^{'} \nonumber 
\end{eqnarray}

\begin{eqnarray}
  \frac{T(\vec{r} \to \vec{r^{'}},E,\hat{\Omega})}{\Sigma_T(\vec{r^{'}},E)} & = &
  exp\left[-\int_0^{|\vec{r^{'}} - \vec{r}|} 
    \Sigma_T \left(\vec{r} + R^{''}\hat{\Omega},E \right) dR^{''} 
    \right] \frac{\delta \left(\Omega - \left[\frac{\vec{r^{'}} - \vec{r}}
      {|\vec{r^{'}} - \vec{r}|}\right]\right)}
       {|\vec{r^{'}} - \vec{r}|^2} \nonumber \\
       & = & exp\left[-\int_0^{|\vec{r} - \vec{r^{'}}|} 
    \Sigma_T \left(\vec{r} + R^{''}\hat{\Omega},E \right) dR^{''} 
    \right] \frac{\delta \left(\Omega + \left[\frac{\vec{r} - \vec{r^{'}}}
      {|\vec{r} - \vec{r^{'}}|}\right]\right)}
       {|\vec{r} - \vec{r^{'}}|^2} \nonumber \\
       & = & \tau(\vec{r^{'}},\vec{r},E,-\hat{\Omega}) \\
       & = & \frac{T(\vec{r^{'}} \to \vec{r},E,-\hat{\Omega})}
       {\Sigma_T(\vec{r},E)}
\end{eqnarray}
The complete adjoint of the emission density state transition kernel can now be 
defined.
\begin{equation}
  K^{\dagger}(\vec{r^{'}} \to \vec{r},E^{'} \to E,
  \hat{\Omega^{'}} \to \hat{\Omega}) = 
  \tau(\vec{r^{'}},\vec{r},E,-\hat{\Omega}) 
  \Sigma_S(\vec{r^{'}},E \to E^{'},\hat{\Omega} \to \hat{\Omega^{'}})
\end{equation}
The source term $c(\vec{r},E,\hat{\Omega})$ can also be modified, since it
contains the modified transport kernel.
\begin{eqnarray}
  c(\vec{r},E,\hat{\Omega}) & = & \int \frac{a(\vec{r^{'}},E,\hat{\Omega})}
  {\Sigma_T(\vec{r^{'}},E)}T(\vec{r} \to \vec{r^{'}},E,\hat{\Omega}) dV^{'} 
  \nonumber \\
  & = & \int a(\vec{r^{'}},E,\hat{\Omega}) 
  \tau(\vec{r^{'}},\vec{r},E,-\hat{\Omega}) dV^{'}
\end{eqnarray}

Finally, the adjoint of the emission density FIESK can be rewritten with the 
modifications that were just outlined. By comparing equation
\ref{eq:adjoint_of_emission_density_integral_eqn_2} to equation 
\ref{eq:flux_integral_equation} for the flux, it is clear that
the adjoint of the emission density behaves very similar to the flux. In the 
literature, the adjoint of the emission density is often called ``flux-like'' 
because of this similarity \citep{hoogenboom}. Though it will not be shown, 
the adjoint of the collision density is also ``flux-like.'' Unfortunetly, it 
was shown in the previous chapter that the random walk process to estimate the 
flux was not ideal. While a random walk process could be constructed for the 
adjoint of the emission density, because of its flux-like nature, its random 
walk process will not be ideal either. 
\begin{equation}
  \begin{split}
    \chi^{\dagger}(\vec{r},&E,\hat{\Omega}) =  \int a(\vec{r^{'}},E,\hat{\Omega}) 
    \tau(\vec{r^{'}},\vec{r},E,-\hat{\Omega}) dV^{'} + \\
    & \int\int\int  \tau(\vec{r^{'}},\vec{r},E,-\hat{\Omega}) 
    \Sigma_S(\vec{r^{'}},E \to E^{'},\hat{\Omega} \to \hat{\Omega^{'}})
    dE^{'}d\hat{\Omega^{'}}dV^{'}
  \end{split}
  \label{eq:adjoint_of_emission_density_integral_eqn_2}
\end{equation}

One way to construct a function that will be collision like and still retain
the favorable properties of the dual problem (i.e. the response function 
becomes the source) is to multiply equation 
\ref{eq:adjoint_of_emission_density_integral_eqn_2} by some function 
$\Sigma^{\dagger}(\vec{r},E)$, whose only necessary properties are that is is
strictly positive and that it goes to zero in a vacuum. This manipulation and 
several others are shown in appendix \ref{ch:appendix_a}. However, it will be 
more instructive to take a step back and start over again with the adjoint 
transport equation. Then the process for deriving the random walk process for 
the emission density and collision density outlined in the previous chapter 
can be followed. 
