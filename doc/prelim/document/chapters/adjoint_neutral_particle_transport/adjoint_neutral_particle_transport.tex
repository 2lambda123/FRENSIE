\chapter{Deriving the Monte Carlo Random Walk Process for Adjoint Neutral Particle Transport Problems}
\label{ch:adjoint_neutral_particle_transport}
In chapter \ref{ch:mc_methods}, the dual problem for calculating inner products
over small sphase spaces was introduced. In this chapter, it will be shown how
the dual problem can be solved in the context of neutral particle transport 
problems. This will require both the derivation of the dual problems of interest
and their associated Monte Carlo random walk processes. Since, in the previous
chapter, the FEISKs that described the event densities were shown to have 
ideal random walk PDFs, their dual problems will be evaluated first. The 
random walk process for the more general dual problem, which is characterized
by the adjoint transport equation, will then be determined. 

\section{The Adjoint of the Emission Density FIESK}
Using the procedure outlined in chapter \ref{ch:mc_methods}, the dual emission
density FIESK will be constructed. The function that will solve this dual 
equation will be called the adjoint of the emission density. It is not called
the adjoint emission density because, as will be shown, this function does
not behave like an emission density (or in general an event density). First, 
a linear operator will be constructed from equation 
\ref{eq:emission_density_integral_eqn}, which is the emission density FIESK. 
\begin{equation}
  \begin{split}
    H_{\chi} \cdot &\chi(\vec{r},E,\hat{\Omega}) = 
    \chi(\vec{r},E,\hat{\Omega}) - \\
    & \int\int\int C(\vec{r},E^{'} \to E,\hat{\Omega^{'}} \to \hat{\Omega})
    T(\vec{r^{'}} \to \vec{r},E^{'},\hat{\Omega^{'}}) 
    \chi(\vec{r^{'}},E',\hat{\Omega^{'}}) dV^{'}dE^{'}E\hat{\Omega^{'}}
  \end{split}
\end{equation}
With the operator $H_{\chi}$ defined in this way it is clear from equation
\ref{eq:emission_density_integral_eqn} that 
$H_{\chi} = S(\vec{r},E,\hat{\Omega})$.

To define the adjoint operator, the following equality must hold. Where the 
brackets indicate integration over all phase space.
\begin{equation*}
  \langle \chi^{\dagger}H_{\chi} \cdot \chi \rangle = 
  \langle \chi H_{\chi}^{\dagger} \cdot \chi^{\dagger} \rangle
\end{equation*}
Since the operator $H_{\chi}$ is known, the left bracket will be expanded.
\begin{equation*}
  \begin{split}
    \langle &\chi^{\dagger}H_{\chi} \cdot \chi \rangle =
    \int\int\int \
    \chi^{\dagger}(\vec{r},E,\hat{\Omega}) \chi(\vec{r},E,\hat{\Omega})
    dV dE d\hat{\Omega} - \int\int\int \chi^{\dagger}(\vec{r},E,\hat{\Omega}) \\
    & \cdot \int\int\int 
    C(\vec{r},E^{'} \to E, \hat{\Omega^{'}} \to \hat{\Omega^{'}})
    T(\vec{r^{'}} \to \vec{r},E^{'},\hat{\Omega^{'}})
    \chi(\vec{r^{'}},E^{'},\hat{\Omega^{'}}) dV^{'}dE^{'}d\hat{\Omega^{'}}
    dV dE d\hat{\Omega}
  \end{split}
\end{equation*}
\begin{equation*}
  \begin{split}
    & \qquad \qquad = \int\int\int \
    \chi^{\dagger}(\vec{r},E,\hat{\Omega}) \chi(\vec{r},E,\hat{\Omega})
    dV dE d\hat{\Omega} - \int\int\int \chi(\vec{r^{'}},E^{'},\hat{\Omega^{'}}) \\
    & \cdot \int\int\int 
    C(\vec{r},E^{'} \to E, \hat{\Omega^{'}} \to \hat{\Omega^{'}})
    T(\vec{r^{'}} \to \vec{r},E^{'},\hat{\Omega^{'}})
    \chi^{\dagger}(\vec{r},E,\hat{\Omega}) dV dE d\hat{\Omega}
    dV^{'}dE^{'}d\hat{\Omega^{'}}
  \end{split}
\end{equation*}
From this last manipulation, the adjoint operator can be deduced, which is 
shown in the following equation.
\begin{equation}
  \begin{split}
    H_{\chi}^{\dagger} \cdot &\chi^{\dagger}(\vec{r},E,\hat{\Omega}) = 
    \chi^{\dagger}(\vec{r},E,\hat{\Omega}) - \\
    & \int\int\int T(\vec{r} \to \vec{r^{'}},E,\hat{\Omega}) 
    C(\vec{r^{'}},E \to E^{'},\hat{\Omega} \to \hat{\Omega^{'}})
    \chi(\vec{r^{'}},E',\hat{\Omega^{'}}) dE^{'}E\hat{\Omega^{'}}dV^{'}
  \end{split}
\end{equation}

In the previous chapter, the function $c(\vec{r},E,\hat{\Omega})$ was defined 
in equation \ref{eq:emission_response_function}, which should be used with 
the collision density to calculate a response. By forcing the adjoint 
operator acting on the adjoint of the emission density to equal the 
function $c(\vec{r},E,\hat{\Omega})$, the dual equation, or simply the adjoint
of the emission density FIESK can be created.
\begin{equation}
  \begin{split}
    \chi^{\dagger}(\vec{r},&E,\hat{\Omega}) = c(\vec{r},E,\hat{\Omega}) + \\
    &\int\int\int T(\vec{r} \to \vec{r^{'}},E,\hat{\Omega}) 
    C(\vec{r^{'}},E \to E^{'},\hat{\Omega} \to \hat{\Omega^{'}})
    \chi^{\dagger}(\vec{r^{'}},E',\hat{\Omega^{'}}) dE^{'}E\hat{\Omega^{'}}dV^{'}
  \end{split}
  \label{eq:adjoint_of_emission_density_integral_eqn}
\end{equation}

Unfortunately, in the integral in equation 
\ref{eq:adjoint_of_emission_density_integral_eqn}, the transport and collision 
kernels are integrated over what was previously the final states. Therefore,
the properties that were defined for these kernels in the previous chapter are 
no longer valid. In particular, the transport kernel is not normalized anymore
and the collision kernel is not equal to the non-absorption probability when
integrated over all final states (which were previously the initial states). In
general, the collision kernel is not even strictly less than or equal to one 
anymore when integrated over all final states. To try and simplify these 
kernels, the $\Sigma_T(\vec{r^{'}},E)$ terms in both the transport and collision
operators will be allowed to cancel each other out. The modified tranpsport
kernel will now be examined.
\begin{equation}
  \frac{T(\vec{r} \to \vec{r^{'}},E,\hat{\Omega})}{\Sigma_T(\vec{r^{'}},E)} = 
  exp\left[-\int_0^{|\vec{r^{'}} - \vec{r}|} 
    \Sigma_T(\vec{r^{'}} - R^{'} \hat{\Omega},E)dR^{'} \right]
  \frac{\delta \left(\Omega - \left[\frac{\vec{r^{'}} - \vec{r}}
      {|\vec{r^{'}} - \vec{r}|}\right]\right)}
       {|\vec{r^{'}} - \vec{r}|^2}
  \label{eq:unnormalized_adjoint_transport_kernel}
\end{equation}
Based on the argument of the delta function, 
$\hat{\Omega} = \frac{\vec{r^{'}} - \vec{r}}{|\vec{r^{'}} - \vec{r}|}$ and 
therefore, $\vec{r^{'}} = \vec{r} + \hat{\Omega}|\vec{r^{'}} - \vec{r}|$. This
equation for $\vec{r^{'}}$ can be substituted back into the equation
\ref{eq:unnormalized_adjoint_transport_kernel}.
\begin{equation*}
  \frac{T(\vec{r} \to \vec{r^{'}},E,\hat{\Omega})}{\Sigma_T(\vec{r^{'}},E)} = 
  exp\left[-\int_0^{|\vec{r^{'}} - \vec{r}|} 
    \Sigma_T \left(\vec{r} + \left[|\vec{r^{'}} - \vec{r}| - R^{'} \right] 
    \hat{\Omega},E \right) dR^{'} 
    \right] \frac{\delta \left(\Omega - \left[\frac{\vec{r^{'}} - \vec{r}}
      {|\vec{r^{'}} - \vec{r}|}\right]\right)}
       {|\vec{r^{'}} - \vec{r}|^2}
\end{equation*}
A new variable of integration can be defined to simplify the exponent. Note
that when $R^{'} = 0$, $R^{''} = |\vec{r^{'}} - \vec{r}|$ and when 
$R^{'} = |\vec{r^{'}} - \vec{r}|$, $R^{''} = 0$.
\begin{eqnarray}
  R^{''} & = & |\vec{r^{'}} - \vec{r}| - R^{'} \nonumber \\
  dR^{''} & = & -dR^{'} \nonumber 
\end{eqnarray}

\begin{eqnarray}
  \frac{T(\vec{r} \to \vec{r^{'}},E,\hat{\Omega})}{\Sigma_T(\vec{r^{'}},E)} & = &
  exp\left[-\int_0^{|\vec{r^{'}} - \vec{r}|} 
    \Sigma_T \left(\vec{r} + R^{''}\hat{\Omega},E \right) dR^{''} 
    \right] \frac{\delta \left(\Omega - \left[\frac{\vec{r^{'}} - \vec{r}}
      {|\vec{r^{'}} - \vec{r}|}\right]\right)}
       {|\vec{r^{'}} - \vec{r}|^2} \nonumber \\
       & = & exp\left[-\int_0^{|\vec{r} - \vec{r^{'}}|} 
    \Sigma_T \left(\vec{r} + R^{''}\hat{\Omega},E \right) dR^{''} 
    \right] \frac{\delta \left(\Omega + \left[\frac{\vec{r} - \vec{r^{'}}}
      {|\vec{r} - \vec{r^{'}}|}\right]\right)}
       {|\vec{r} - \vec{r^{'}}|^2} \nonumber \\
       & = & \tau(\vec{r^{'}},\vec{r},E,-\hat{\Omega}) \\
       & = & \frac{T(\vec{r^{'}} \to \vec{r},E,-\hat{\Omega})}
       {\Sigma_T(\vec{r},E)}
\end{eqnarray}
The complete adjoint of the emission density state transition kernel can now be 
defined.
\begin{equation}
  K^{\dagger}(\vec{r^{'}} \to \vec{r},E^{'} \to E,
  \hat{\Omega^{'}} \to \hat{\Omega}) = 
  \tau(\vec{r^{'}},\vec{r},E,-\hat{\Omega}) 
  \Sigma_S(\vec{r^{'}},E \to E^{'},\hat{\Omega} \to \hat{\Omega^{'}})
\end{equation}
The source term $c(\vec{r},E,\hat{\Omega})$ can also be modified, since it
contains the modified transport kernel.
\begin{eqnarray}
  c(\vec{r},E,\hat{\Omega}) & = & \int \frac{a(\vec{r^{'}},E,\hat{\Omega})}
  {\Sigma_T(\vec{r^{'}},E)}T(\vec{r} \to \vec{r^{'}},E,\hat{\Omega}) dV^{'} 
  \nonumber \\
  & = & \int a(\vec{r^{'}},E,\hat{\Omega}) 
  \tau(\vec{r^{'}},\vec{r},E,-\hat{\Omega}) dV^{'}
\end{eqnarray}

Finally, the adjoint of the emission density FIESK can be rewritten with the 
modifications that were just outlined. By comparing equation
\ref{eq:adjoint_of_emission_density_integral_eqn_2} to equation 
\ref{eq:flux_integral_equation} for the flux, it is clear that
the adjoint of the emission density behaves very similar to the flux. In the 
literature, the adjoint of the emission density is often called ``flux-like'' 
because of this similarity \citep{hoogenboom}. Though it will not be shown, 
the adjoint of the collision density is also ``flux-like.'' Unfortunetly, it 
was shown in the previous chapter that the random walk process to estimate the 
flux was not ideal. While a random walk process could be constructed for the 
adjoint of the emission density, because of its flux-like nature, its random 
walk process will not be ideal either. 
\begin{equation}
  \begin{split}
    \chi^{\dagger}(\vec{r},&E,\hat{\Omega}) =  \int a(\vec{r^{'}},E,\hat{\Omega}) 
    \tau(\vec{r^{'}},\vec{r},E,-\hat{\Omega}) dV^{'} + \\
    & \int\int\int  \tau(\vec{r^{'}},\vec{r},E,-\hat{\Omega}) 
    \Sigma_S(\vec{r^{'}},E \to E^{'},\hat{\Omega} \to \hat{\Omega^{'}})
    dE^{'}d\hat{\Omega^{'}}dV^{'}
  \end{split}
  \label{eq:adjoint_of_emission_density_integral_eqn_2}
\end{equation}

One way to construct a function that will be collision like and still retain
the favorable properties of the dual problem (i.e. the response function 
becomes the source) is to multiply equation 
\ref{eq:adjoint_of_emission_density_integral_eqn_2} by some function 
$\Sigma^{\dagger}(\vec{r},E)$, whose only necessary properties are that is is
strictly positive and that it goes to zero in a vacuum. This manipulation and 
several others are shown in appendix \ref{ch:appendix_a}. However, it will be 
more instructive to take a step back and start over again with the adjoint 
transport equation. Then the process for deriving the random walk process for 
the emission density and collision density outlined in the previous chapter 
can be followed. 

\section{The Adjoint Integro-Differential Boltzmann Equation}
The integro-differential transport equation, like a FIESK, can be written in an
operator form. In the following equation, it is assumed that the flux 
distribution is time-independent.
\begin{equation}
  H_B \cdot \varphi(\vec{r},E,\hat{\Omega}) = S(\vec{r},E,\hat{\Omega})
\end{equation}
Using integro-differential transport operator, defined below, and the equality
shown in equation \ref{eq:forward_adjoint_ops} the adjoint integro-differential
transport operator can be derived. This derivation is shown in detail in 
\citep{lewis}. The function $\varphi^{\dagger}(\vec{r},E,\hat{\Omega})$ is the
adjoint flux.
\begin{equation}
  \begin{split}
    H_B \cdot \varphi(\vec{r},E,\hat{\Omega}) &= 
    \left[ \hat{\Omega} \cdot \vec{\triangledown} +
     \Sigma_T(\vec{r},E) \right] \varphi(\vec{r},E,\hat{\Omega}) - \\
     & \int\int \Sigma_S(\vec{r},E^{'} \to E,\hat{\Omega^{'}} \to \hat{\Omega})
    \varphi(\vec{r},E^{'},\hat{\Omega^{'}}) dE^{'} d\hat{\Omega^{'}}
  \end{split}
  \label{eq:integro_diff_trans_op}
\end{equation}
\begin{equation}
  \begin{split}
    H_B^{\dagger} \cdot \varphi^{\dagger}(\vec{r},E,\hat{\Omega}) &= 
    \left[ -\hat{\Omega} \cdot \vec{\bigtriangledown} +
     \Sigma_T(\vec{r},E) \right] \varphi^{\dagger}(\vec{r},E,\hat{\Omega}) - \\
     & \int\int \Sigma_S(\vec{r},E \to E^{'},\hat{\Omega} \to \hat{\Omega^{'}})
    \varphi^{\dagger}(\vec{r},E^{'},\hat{\Omega^{'}}) dE^{'} d\hat{\Omega^{'}}
  \end{split}
  \label{eq:integro_diff_adj_trans_op}
\end{equation}

Finally, the adjoint integro-differential transport equation can be derived by
forcing $H_B^{\dagger} \cdot \varphi^{\dagger}(\vec{r},E,\hat{\Omega})$ to equal
the response function $a(\vec{r},E,\hat{\Omega})$. This will ensure that the
inner product of the adjoint flux and the source will give the same value as
the inner product of the flux and the response function.
\begin{equation}
  \begin{split}
    -\hat{\Omega} &\cdot \vec{\bigtriangledown} 
    \varphi^{\dagger}(\vec{r},E,\hat{\Omega},t)
    + \Sigma_T(\vec{r},E) \varphi^{\dagger}(\vec{r},E,\hat{\Omega},t) = \\
    & \quad a(\vec{r},E,\hat{\Omega},t) +
    \int\int \Sigma_S(\vec{r},E \to E^{'},\hat{\Omega} \to \hat{\Omega^{'}})
    \varphi(\vec{r},E^{'},\hat{\Omega^{'}},t) dE^{'}d\hat{\Omega^{'}} 
  \end{split}
  \label{eq:integro_diff_adj_boltzmann_eqn}
\end{equation}

\section{The Adjoint Transport Equation in Integral Form}
To simplify the adjoint transport equation, the adjoint emission density will 
be defined.
\begin{equation}
  \begin{split}
    \theta^{\dagger}(\vec{r},E,\hat{\Omega}) = a(\vec{r},&E,\hat{\Omega}) + \\
    & \int\int \Sigma_S(\vec{r},E \to E^{'},\hat{\Omega} \to \hat{\Omega^{'}})
    \varphi^{'}(\vec{r},E^{'},\hat{\Omega^{'}}) dE^{'}d\hat{\Omega^{'}}
  \end{split}
  \label{eq:adjoint_emission_density}
\end{equation}
The reduced adjoint transport equation is shown in the following equation.
\begin{equation}
  -\hat{\Omega} \cdot \vec{\bigtriangledown} 
    \varphi^{\dagger}(\vec{r},E,\hat{\Omega},t)
    + \Sigma_T(\vec{r},E) \varphi^{\dagger}(\vec{r},E,\hat{\Omega},t) =
    \theta^{\dagger}(\vec{r},E,\hat{\Omega})
\end{equation}
  
From the reduced adjoint transport equation, the method of characteristics will
be used to transform the reduced transport equation to its integral form. The
characteristic for the adjoint transport equation is the following 
parameterized line.
\begin{equation}
  \vec{r^{'}} = \vec{r} + R\hat{\Omega}
  \label{eq:adjoint_characteristic}
\end{equation}
Using equation \ref{eq:adjoint_characteristic} a directional derivative along
the characteristic can be determined.
\begin{eqnarray}
  \frac{d}{dR} & = & \frac{dx^{'}}{dR}\frac{\partial}{\partial x} +
  \frac{dy^{'}}{dR}\frac{\partial}{\partial y} +
  \frac{dz^{'}}{dR}\frac{\partial}{\partial z} \nonumber \\
  & = & \Omega_x \frac{\partial}{\partial x} +
  \Omega_y \frac{\partial}{\partial y} +
  \Omega_z \frac{\partial}{\partial z} \nonumber \\
  & = & \hat{\Omega} \cdot \vec{\bigtriangledown}
\end{eqnarray}

Using this relationship of the directional derivative along the characteristic,
the transport equation can be reduced to a first order ODE, which can be
easily solved with the integrating factor 
$exp\left[-\int_0^R \Sigma_T(\vec{r}+R^{'}\hat{\Omega},E)dR^{'} \right]$.
\begin{equation*}
  -\frac{d}{dR}\varphi^{\dagger}(\vec{r^{'}},E,\hat{\Omega}) + 
  \Sigma_T(\vec{r^{'}},E)
  \varphi^{\dagger}(\vec{r^{'}},E,\hat{\Omega}) = 
  \theta^{\dagger}(\vec{r^{'}},E,\hat{\Omega})
\end{equation*}
\begin{equation*}
  \begin{split}
    -\frac{d}{dR}\bigg[\varphi^{\dagger}(\vec{r^{'}},E,\hat{\Omega})
      &exp\left[-\int_0^R \Sigma_T(\vec{r}+R^{'}\hat{\Omega},E)dR^{'}\right]
      \bigg] = \\
    & \theta^{\dagger}(\vec{r^{'}},E,\hat{\Omega})
    exp\left[-\int_0^R \Sigma_T(\vec{r}+R^{'}\hat{\Omega},E)dR^{'} \right]
  \end{split}
\end{equation*}
\begin{equation*}
  \begin{split}
    -\varphi^{\dagger}(\vec{r} + R\hat{\Omega},&E,\hat{\Omega})
    exp\left[-\int_0^R \Sigma_T(\vec{r}+R^{'}\hat{\Omega},E)dR^{'}\right] 
    \bigg|_0^{\infty} = \\
    & \int_0^{\infty} 
    \theta^{\dagger}(\vec{r} + R\hat{\Omega},E,\hat{\Omega})
    exp\left[-\int_0^R \Sigma_T(\vec{r}+R^{'}\hat{\Omega},E)dR^{'} \right] dR
  \end{split}
\end{equation*}

\vspace{0.5cm}
\begin{equation}
    \varphi^{\dagger}(\vec{r},E,\hat{\Omega}) = 
    \int_0^{\infty} \theta^{\dagger}(\vec{r} + R\hat{\Omega},E,\hat{\Omega})
    exp\left[-\int_0^R \Sigma_T(\vec{r}+R^{'}\hat{\Omega},E)dR^{'} \right] dR
  \label{eq:line_integral_adj_transport_eqn}
\end{equation}

To get to equation \ref{eq:line_integral_adj_transport_eqn} it was assumed that
the adjoint flux goes to zero as R goes to infinity. This line integral will
now be converted to an integral over all space. First, note that 
$R = |\vec{r^{'}} - \vec{r}|$,
$\hat{\Omega} = \frac{\vec{r^{'}} - \vec{r}}{|\vec{r^{'}} - \vec{r}}|$ and
$dV^{'} = R^2dRd\hat{\Omega}$
\begin{equation}
  \begin{split}
    \varphi^{\dagger}(\vec{r},E,\hat{\Omega}) = 
    \int \theta^{\dagger}(\vec{r^{'}},E,\hat{\Omega})
    exp\Big[-\int_0^{|\vec{r^{'}} - \vec{r}|} 
      &\Sigma_T(\vec{r}+R^{'}\hat{\Omega},E)dR^{'} \Big] \\
    &\cdot \frac{\delta \left(\Omega - \left[\frac{\vec{r^{'}} - \vec{r}}
        {|\vec{r^{'}} - \vec{r}|}\right]\right)}
    {|\vec{r^{'}} - \vec{r}|^2} dV^{'}
  \end{split}
  \label{eq:volume_integral_adj_transport_eqn}
\end{equation}

This integral equation for the adjoint flux could be converted to a FIESK, with
a similar form to the flux FIESK. However, like the flux, the adjoint flux
FIESK will not be well suited for a Monte Carlo random walk process. Instead,
the adjoint emission density FIESK will be derived next.

\section{The Adjoint Emission Density FIESK}
To construct a FIESK for the adjoint emission density equation 
\ref{eq:volume_integral_adj_transport_eqn} must be substituted back into 
equation \ref{eq:adjoint_emission_density}. 
\begin{equation*}
  \begin{split}
    \theta^{\dagger}(\vec{r},&E,\hat{\Omega}) = a(\vec{r},E,\hat{\Omega}) +
    \int\int \Sigma_S(\vec{r},E \to E^{'},\hat{\Omega} \to \hat{\Omega^{'}})
    \int \theta^{\dagger}(\vec{r^{'}},E^{'},\hat{\Omega^{'}}) \\
    & exp\left[-\int_0^{|\vec{r^{'}} - \vec{r}|} 
      \Sigma_T(\vec{r}+R^{'}\hat{\Omega^{'}},E^{'})dR^{'} \right]
    \cdot \frac{\delta \left(\Omega^{'} - \left[\frac{\vec{r^{'}} - \vec{r}}
        {|\vec{r^{'}} - \vec{r}|}\right]\right)}
    {|\vec{r^{'}} - \vec{r}|^2} dV^{'} dE^{'} d\hat{\Omega^{'}}
  \end{split}
\end{equation*}
\begin{equation*}
  \begin{split}
    \theta^{\dagger}(\vec{r},&E,\hat{\Omega}) = a(\vec{r},E,\hat{\Omega}) +
    \int\int \frac{\Sigma^{\dagger}(\vec{r},E^{'})}{\Sigma_T(\vec{r},E^{'})}
    \frac{\Sigma_S(\vec{r},E \to E^{'},\hat{\Omega} \to \hat{\Omega^{'}})}
         {\Sigma^{\dagger}(\vec{r},E^{'})}
    \int \theta^{\dagger}(\vec{r^{'}},E^{'},\hat{\Omega^{'}}) \\
    & \Sigma_T(\vec{r},E^{'}) exp\left[-\int_0^{|\vec{r^{'}} - \vec{r}|} 
      \Sigma_T(\vec{r}+R^{'}\hat{\Omega^{'}},E^{'})dR^{'} \right]
    \cdot \frac{\delta \left(\Omega^{'} - \left[\frac{\vec{r^{'}} - \vec{r}}
        {|\vec{r^{'}} - \vec{r}|}\right]\right)}
    {|\vec{r^{'}} - \vec{r}|^2} dV^{'} dE^{'} d\hat{\Omega^{'}}
  \end{split}
\end{equation*}
\begin{equation}
  \begin{split}
    \theta^{\dagger}(\vec{r},&E,\hat{\Omega}) = a(\vec{r},E,\hat{\Omega}) + 
    \int\int\int P^{\dagger}(\vec{r},E^{'})
    C^{\dagger}(\vec{r},E^{'} \to E,\hat{\Omega^{'}} \to \hat{\Omega}) \\
    & \cdot T^{\dagger}(\vec{r^{'}} \to \vec{r},E^{'},\hat{\Omega^{'}})
    \theta^{\dagger}(\vec{r^{'}},E^{'},\hat{\Omega^{'}})
    dV^{'} dE^{'} d\hat{\Omega^{'}}
  \end{split}
\end{equation}

The factor $P^{\dagger}(\vec{r},E^{'})$ is called the adjoint weight factor, which
is defined in the next equation. The adjoint weight factor contains the 
function $\Sigma^{\dagger}(\vec{r},E^{'})$, which will be called the total
adjoint cross section. The adjoint weight factor takes into account the fact
that the total cross section is not equal to the total adjoint cross section.
In some literature, this factor is called that adjoint non-absorption 
probability \citep{amos}. However, as will be shown in chapter \ref{}, this
factor is not bounded in the interval (0,1) but is instead bounded in the 
interval (0,$\infty$). It is therefore inappropriate to call this factor
a probability.
\begin{equation}
  P^{\dagger}(\vec{r},E) = \frac{\Sigma^{\dagger}(\vec{r},E)}
  {\Sigma_T(\vec{r},E)}
\end{equation}

\begin{equation}
  \Sigma^{\dagger}(\vec{r},E^{'}) = \int\int 
  \Sigma_S(\vec{r},E \to E^{'},\hat{\Omega} \to \hat{\Omega^{'}}) dEd\hat{\Omega}
\end{equation}

The kernel $T^{\dagger}(\vec{r^{'}} \to \vec{r},E,\hat{\Omega})$ is called
the adjoint transport kernel, which is defined in the next equation. Like the
transport kernel, the adjoint transport kernel prevents events from being
sampled in a vacuum.
\begin{equation}
  \begin{split}
  T^{\dagger}(\vec{r^{'}} \to \vec{r},E,\hat{\Omega}) = 
  \Sigma_T(\vec{r},E^{'}) exp\Big[-\int_0^{|\vec{r^{'}} - \vec{r}|} 
      \Sigma_T(&\vec{r}+R^{'}\hat{\Omega^{'}},E^{'})dR^{'} \Big] \\
    & \cdot \frac{\delta \left(\Omega^{'} - \left[\frac{\vec{r^{'}} - \vec{r}}
        {|\vec{r^{'}} - \vec{r}|}\right]\right)}
    {|\vec{r^{'}} - \vec{r}|^2}
  \end{split}
\end{equation}

The kernel $C^{\dagger}(\vec{r},E^{'} \to E,\hat{\Omega^{'}} \to \hat{\Omega})$ is
called the adjoint collision kernel, which is defined in the next equation.
Unlike the collision kernel, it is normalized because there is no equivalent
to absorption for adjoint particles.  
\begin{equation}
  C^{\dagger}(\vec{r},E^{'} \to E,\hat{\Omega^{'}} \to \hat{\Omega}) = 
  \frac{\Sigma_S(\vec{r},E \to E^{'},\hat{\Omega} \to \hat{\Omega^{'}})}
       {\Sigma^{\dagger}(\vec{r},E^{'})}
\end{equation}

The adjoint weight factor, adjoint transport kernel and the adjoint collision
kernel can be combined to create a kernel that characterizes the transition of
a particle from state $y = (\vec{r^{'}},E^{'},\hat{\Omega^{'}})$ of the phase
space $\Gamma$ to another state $x = (\vec{r},E,\hat{\Omega})$. This new kernel
is given in the following equation.
\begin{equation}
  \begin{split}
    M^{\dagger}(\vec{r^{'}} \to \vec{r},E^{'} \to E,\hat{\Omega^{'}} \to \hat{\Omega})
    = C^{\dagger}(\vec{r},E^{'} \to E,&\hat{\Omega^{'}} \to \hat{\Omega})
    P^{\dagger}(\vec{r},E^{'}) \\
    & \cdot T^{\dagger}(\vec{r^{'}} \to \vec{r},E^{'},\hat{\Omega^{'}})
  \end{split}
\end{equation}

The adjoint emission density FIESK is shown below.
\begin{equation}
  \theta^{\dagger}(x) = a(x) + \int_{\Gamma}M^{\dagger}(y \to x) \theta^{\dagger}(y) dy
\end{equation}

\section{The Adjoint Collision Density FIESK}
Before taking a closer look at the kernel $M^{\dagger}(y \to x)$, the adjoint
collision density must be defined. It is related to the adjoint flux and the
adjoint emission density by the following equations, which are analogous to
the relationships between the collision density, flux and emission density.
\begin{eqnarray}
  \xi^{\dagger}(\vec{r},E,\hat{\Omega}) & = & \Sigma_T(\vec{r},E)
  \varphi^{\dagger}(\vec{r},E,\hat{\Omega}) \\
  & = & \int T^{\dagger}(\vec{r^{'}} \to \vec{r},E,\hat{\Omega})
  \theta^{\dagger}(\vec{r^{'}},E,\hat{\Omega}) dV^{'}
  \label{eq:adj_collision_dens_to_adj_emission_dens}
\end{eqnarray}

Like the collision density, the adjoint collision density is the density of
particles exiting a collision. Using equation 
\ref{eq:adj_collision_dens_to_adj_emission_dens} and equation 
\ref{eq:adjoint_emission_density} an integral equation for the collision 
density can be derived.
\begin{equation}
  \xi^{\dagger}(x) = S_c^{\dagger}(x) + \int_{\Gamma} N^{\dagger}(y \to x)
  \xi^{\dagger}(y) dy
\end{equation}
In this equation $S_c^{\dagger}(x)$ is the adjoint first collided source and
$N^{\dagger}(y \to x)$ is a new kernel that also characterizes the transition
of an adjoint particle from state $y = (\vec{r^{'}},E^{'},\hat{\Omega^{'}})$ of
the phase space gamma to another state $x = (\vec{r},E,\hat{\Omega})$. These
functions are defined in the following equations.
\begin{equation}
  S_c^{\dagger}(\vec{r},E,\hat{\Omega}) = \int 
  T^{\dagger}(\vec{r^{'}} \to \vec{r},E,\hat{\Omega}) a(\vec{r^{'}},E,\hat{\Omega})
  dV^{'}
\end{equation}
\begin{equation}
  \begin{split}
    N^{\dagger}(\vec{r^{'}} \to \vec{r},E^{'} \to E,\hat{\Omega^{'}} \to \hat{\Omega})
    = T^{\dagger}(\vec{r^{'}} \to \vec{r},E,\hat{\Omega})
    C^{\dagger}(\vec{r^{'}},&E^{'} \to E,\hat{\Omega^{'}} \to \hat{\Omega}) \\
    & \cdot P^{\dagger}(\vec{r^{'}},E^{'}) 
  \end{split}
\end{equation}

\section{Adjoint Emission and Collision Density State Transition Kernel Properties}
Before the PDFs that govern the random walk process are derived, the adjoint
transport kernel, adjoint collision kernel and adjoint weight factor must be 
investigated a bit further. A comparison of the adjoint transport kernel and 
the transport kernel reveals that the adjoint transport kernel is identical 
the the transport kernel with $\hat{\Omega} = -\hat{\Omega}$. The significance 
of this relationship is that adjoint particles move in the direction opposite 
of the variable $\hat{\Omega}$. It is therefore inappropriate to interpret the 
variable $\hat{\Omega}$ of an adjoint particle as its direction. However, for 
convenience, it will continue to be refered to as an adjoint particle's 
direction. This relationship also indicates that the adjoint transport kernel
is normalized for an infinite medium.
\begin{equation}
  T^{\dagger}(\vec{r^{'}} \to \vec{r},E,\hat{\Omega}) = 
  T(\vec{r^{'}} \to \vec{r},E,-\hat{\Omega}) 
\end{equation}
\begin{equation}
  \int T^{\dagger}(\vec{r^{'}} \to \vec{r},E,\hat{\Omega}) dV = 1
\end{equation}

The adjoint collision kernel can be expanded, which will provide some
additional information.
\begin{eqnarray}
  C^{\dagger}(\vec{r},E^{'} \to E,\hat{\Omega^{'}} \to \hat{\Omega}) & = &
  \frac{\Sigma_S(\vec{r},E \to E^{'},\hat{\Omega} \to \hat{\Omega^{'}})}
       {\Sigma^{\dagger}(\vec{r},E^{'})} \nonumber \\
       & = & \sum_j \frac{\Sigma_{S,j}(\vec{r},E)}{\Sigma^{\dagger}(\vec{r},E^{'})}
       f_j(E \to E^{'},\hat{\Omega} \to \hat{\Omega^{'}}) \nonumber \\
  & = & \sum_{A,j} \frac{\Sigma_A^{\dagger}(\vec{r},E^{'})}
                                          {\Sigma^{\dagger}(\vec{r},E^{'})}
  \frac{\sigma_{j,A}^{\dagger}(E^{'})}{\sigma_A^{\dagger}(E^{'})}
  \frac{\sigma_{j,A}(E)f_{j,A}(E \to E^{'},\hat{\Omega} \to \hat{\Omega^{'}})}
       {\sigma_{j,A}^{\dagger}(E^{'})} \nonumber \\
  & = & \sum_{A,j} \frac{\Sigma_A^{\dagger}(\vec{r},E^{'})}
                       {\Sigma^{\dagger}(\vec{r},E^{'})}
       \frac{\sigma_{j,A}^{\dagger}(E^{'})}{\sigma_A^{\dagger}(E^{'})}
       f_{j,A}^{\dagger}(E^{'} \to E,\hat{\Omega^{'}} \to \hat{\Omega})
  \label{eq:expanded_adj_collision_kernel}
\end{eqnarray}
In the expansion of the adjoint collision kernel, the subscript A denotes a 
particular nuclide and the subscript j denotes a particular type of scattering
reaction. The function $f_{j,A}(E \to E^{'},\hat{\Omega} \to \hat{\Omega^{'}})$
is no longer normalized and cannot be used as a PDF for sampling the outgoing
adjoint particle energy and direction. However, as indicated in equation
\ref{eq:expanded_adj_collision_kernel} a new PDF can be constructed for
adjoint particles.
\begin{equation}
  f_{j,A}^{\dagger}(E^{'} \to E,\hat{\Omega^{'}} \to \hat{\Omega}) = 
  \frac{\sigma_{j,A}(E)f_{j,A}(E \to E^{'},\hat{\Omega} \to \hat{\Omega^{'}})}
       {\sigma_{j,A}^{\dagger}(E^{'})}
\end{equation}
\begin{equation}
  \int\int f_{j,A}^{\dagger}(E^{'} \to E,\hat{\Omega^{'}} \to \hat{\Omega})
  dE d\hat{\Omega} = 1
\end{equation}
Because the there is no equivalent to absorption for adjoint particles, the
adjoint collision kernel is normalized.
\begin{equation}
  \int\int C^{\dagger}(\vec{r},E^{'} \to E,\hat{\Omega^{'}} \to \hat{\Omega})
  dE d\hat{\Omega} = 1
\end{equation}

As mentioned before, the adjoint weight factor $P^{\dagger}(\vec{r},E)$ takes 
into account the fact that the total cross section $\Sigma_T(\vec{r},E)$ is not
equal to the total adjoint cross section $\Sigma^{\dagger}(\vec{r},E)$. 
Unfortunately, this factor is not bounded to the interval (0,1), which means
that an analogue random walk process cannot be derived for the adjoint
emission density or collision density. In addition, this factor is likely to
increase the variance of the estimators used. It will be shown later that in
certain circumstances, a change of variables will actually constrain the 
adjoint weight factor to the interval (0,1). In general, understanding the 
behavior of the adjoint weight factor will be very important to predicting
the variance of the estimators used in a calculation.

Based on the properties of the adjoint transport kernel, adjoint collision
kernel and the adjoint weight factor, the properties of the state transition 
kernels $M^{\dagger}(y \to x)$ and $N^{\dagger}(y \to x)$ are clear.
\begin{enumerate}
  \item $M^{\dagger}(y \to x) > 0$ \\
    $N^{\dagger}(y \to x) > 0$
  \item $\int M^{\dagger}(y \to x)dx = P^{\dagger}(y) < \infty \quad \forall y \in
    \Gamma$ \\
    $\int M^{\dagger}(y \to x)dx = P^{\dagger}(y) < \infty \quad \forall y \in
    \Gamma$
\end{enumerate}

With the state transition kernels $M^{\dagger}(y \to x)$ and $N^{\dagger}(y \to x)$
completely specified, the Monte Carlo random walk process for neutral 
particle adjoint transport can be completely specified. 
The following random walk processes are non-analogue because the properties
of the state transition kernels do not permit an analogue random walk process.
\begin{equation}
  \theta(x)\text{ Random Walk:}
  \begin{cases}
    p^1(x) & = \frac{a(x)}{\int_{\Gamma} a(x)dx} \\
    p(y \to x) & = \frac{M^{\dagger}(y \to x)}{P^{\dagger}(y)} \\
    p(x) & = 0
  \end{cases}
  \label{eq:mc_random_walk_adj_emission_dens}
\end{equation}
\begin{equation}
  \xi(x)\text{ Random Walk:}
  \begin{cases}
    p^1(x) & = \frac{S_c^{\dagger}(x)}{\int_{\gamma} S_c^{\dagger}(x)dx} \\
    p(y \to x) & = \frac{N^{\dagger}(y \to x)}{P^{\dagger}(y)} \\
    p(x) & = 0
  \end{cases}
  \label{eq:mc_random_walk_adj_collision_dens}
\end{equation}

Much like the state transition kernels for the emission density and the 
collision density, the state transition kernels for the adjoint emission 
density and the adjoint collision density only differ in the ordering of the
adjoint transport kernel, adjoint collision kernel and the adjoint weight
factor. Therefore, a single random walk process can be used to estimate both
the adjoint emission density and the adjoint collision density. Figure
\ref{} illustrates the new combined random walk process. In this new process,
particles always start in the true adjoint source $a(\vec{r},E,\hat{\Omega}$
and not the adjoint first collided source. This is again advantagous because
the adjoint first collided source would be challenging to calculate 
analytically. As with the random walk process for the emission and collision
density where absorption is ignored, this random walk process will have to
utilize Russian roulette to force random walks to terminate. In addition,
the variance reduction game called splitting will have to be used when
to prevent a particles weight from becoming to large. Essentially, when a 
particle's weight becomes too large, it is split into two or more particles with
an equal fraction of the original particles weight. 

\section{Estimating Responses}
Due to the way that the adjoint transport equation is constructed, the inner
product of the adjoint flux $\varphi^{\dagger}(\vec{r},E,\hat{\Omega})$ and the
source $S(\vec{r},E,\hat{\Omega})$ will result in the same value as the inner 
product of the flux $\varphi(\vec{r},E,\hat{\Omega})$ and the response
function $a(\vec{r},E,\hat{\Omega})$. Because it is not ideal to estimate the
adjoint flux directly using a Monte Carlo random walk procedure, equivalent
inner products must be constructed that are either in terms of the adjoint
collision density or the adjoint emission density.
\begin{eqnarray}
  I & = & \int S(\vec{r},E,\hat{\Omega})\varphi(\vec{r},E,\hat{\Omega})
  dV dE d\hat{\Omega} \\ 
  & = & \int U(\vec{r},E,\hat{\Omega})\xi(\vec{r},E,\hat{\Omega})
  dV dE d\hat{\Omega} \\
  & = & \int V(\vec{r},E,\hat{\Omega})\theta(\vec{r},E,\hat{\Omega})
  dV dE d\hat{\Omega} \\
\end{eqnarray}
Based on the relationship between the adjoint collision density and the
adjoint flux, $U(\vec{r},E,\hat{\Omega})$ must be defined as follows.
\begin{equation}
  U(\vec{r},E,\hat{\Omega}) = \frac{S(\vec{r},E,\hat{\Omega})}
                                   {\Sigma_T(\vec{r},E,\hat{\Omega})}
\end{equation}
Similarly, the function $V(\vec{r},E,\hat{\Omega})$ must be defined as follows.
\begin{eqnarray}
  V(\vec{r},E,\hat{\Omega}) & = & \int \frac{S(\vec{r^{'}},E,\hat{\Omega})}
  {\Sigma_T(\vec{r^{'}},E,\hat{\Omega})} 
  T^{\dagger}(\vec{r} \to \vec{r^{'}},E,\hat{\Omega}) \\
  & = & \frac{S_c(\vec{r},E,\hat{\Omega})}{\Sigma_T(\vec{r},E)}
\end{eqnarray}
