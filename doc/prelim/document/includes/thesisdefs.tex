% thesisdefs.tex

% This is mostly adapted from withesis.cls.  The original copyright
% notice for withesis.cls follows, preceded by two percent signs (%%):

%% withesis.cls
%% LaTeX Style file for the University of Wisconsin-Madison Thesis Format
%% Adapted from the Purdue University Thesis Format
%% Originally by Dave Kraynie
%% Edits by Darrell McCauley
%% Adapted to UW-Madison format by Eric Benedict  (Noted with <EB>)
%% Updated to LaTeX2e by Eric Benedict 24 July 00
%% 
%%=============================================================================
%% Licensed under the Perl Artistic License.
%% see: http://www.ctan.org/tex-archive/help/Catalogue/licenses.artistic.html
%% for more info...
%%=============================================================================

% withesis.cls is available from CTAN.  The modifications to this file
% are also licensed under the Perl Artistic License.

% --wb, 2008

\makeatletter

\newcounter {tocpage}
\newcounter {lofpage}
\newcounter {lotpage}
\newcounter {listofheading}

\newcommand\@thesistitlemedskip{0.25in}
\newcommand\@thesistitlebigskip{0.6in}
\newcommand{\degree}[1]{\gdef\@degree{#1}}
\newcommand{\project}{\gdef\@doctype{A masters project report}}
\newcommand{\prelim}{\gdef\@doctype{A preliminary report}}
\newcommand{\thesis}{\gdef\@doctype{A thesis}}
\newcommand{\dissertation}{\gdef\@doctype{A dissertation}}
\newcommand{\department}[1]{\gdef\@department{(#1)}}

\newenvironment{titlepage}
 {\@restonecolfalse\if@twocolumn\@restonecoltrue\onecolumn
  \else \newpage \fi \thispagestyle{empty}
% \c@page\z@ -- deleted: count title page in thesis
}{\if@restonecol\twocolumn \else \newpage \fi}

\gdef\@degree{Doctor of Philosophy}    %Default is PhD
\gdef\@doctype{A preliminary report}   %Default is dissertation

\gdef\@department{(Engineering Physics)} % Default is Electical Engineering

\renewcommand{\maketitle}{%
  \begin{titlepage}
%-----------------------------------------------------------------------------
% -- The thesis office doesn't like thanks on title page.  Put it in
% -- the acknowledgments.  This is here so you don't have to change
% -- your titlepage when converting from report style. -> from Purdue, but I
%        left it here since it seems compatible with UW-Madison, Eric
%-----------------------------------------------------------------------------
    \def\thanks##1{\typeout{Warning: `thanks' deleted from thesis titlepage.}}
    \let\footnotesize\small \let\footnoterule\relax \setcounter{page}{1}
    \vspace*{0.1in}
    \begin{center}
      {\textbf{\expandafter\uppercase\expandafter{\@title}}} \\[\@thesistitlebigskip]
       by \\[\@thesistitlemedskip]
      \@author \\[\@thesistitlebigskip]
      \@doctype\ submitted in partial fulfillment of \\
      the requirements for the degree of\\[\@thesistitlebigskip]
      \@degree \\[\@thesistitlemedskip]
      \@department \\[\@thesistitlebigskip]
      at the \\[\@thesistitlebigskip]
      UNIVERSITY OF WISCONSIN--MADISON\\[\@thesistitlebigskip]
      \@date
    \end{center}
  \end{titlepage}
  \setcounter{footnote}{0}
  \setcounter{page}{1} %title page is NOT counted
  \let\thanks\relax
  \let\maketitle\relax \let\degree\relax \let\project\relax \let\prelim\relax
  \let\department\relax
  \gdef\@thanks{}\gdef\@degree{}\gdef\@doctype{}
  \gdef\@department{}
  %\gdef\@author{}\gdef\@title{}
}


%=============================================================================
% ABSTRACT
%=============================================================================
% The abstract should begin with two single-spaced lines describing
% the author and title in a standard format.  After these lines comes
% the standard abstract.
%=============================================================================
\def\abstract{
  \chapter*{Abstract}
  \addcontentsline{toc}{chapter}{Abstract}
  \relax\markboth{Abstract}{Abstract}}
\def\endabstract{\par\newpage}


%=============================================================================
% UMI ABSTRACT
%=============================================================================
% The UMI abstract should begin with the author and title in a standard format.
% After the author comes the advisor and university. After these lines comes
% a bunch of double spaced text to make up the standard abstract.
% After the abstract, the advisor's approval signature follows.
% This page is not numbered and is delivered seperately to the thesis office.
%=============================================================================

\def\advisortitle#1{\gdef\@advisortitle{#1}}
\def\advisorname#1{\gdef\@advisorname{#1}}
\gdef\@advisortitle{Professor}
\gdef\@advisorname{Cheer E.\ Place}

\def\umiabstract{
             \thispagestyle{empty}
                  \addtocounter{page}{-1}
                \begin{center}
                  {\textbf{\expandafter\uppercase\expandafter{\@title}}}\\
                  \vspace{12pt}
                  \@author \\
                  \vspace{12pt}
                  Under the supervision of \@advisortitle\ \@advisorname\\
                  At the University of Wisconsin-Madison
                \end{center}
}

\def\endumiabstract{\vfill \hfill\@advisorname\par\newpage}


%============================================================================
% VERBATIMFILE
%============================================================================
% \verbatimfile{<filename>}    for verbatim inclusion of a file
% - Note that the precise layout of line breaks in this file is important!
% - added the \singlespace - EB
%============================================================================
\def\verbatimfile#1{\begingroup \singlespace
                    \@verbatim \frenchspacing \@vobeyspaces
                    \input#1 \endgroup
}


%=============================================================================
% SEPARATOR Pages
%   Creates a blank page with a text centered horizontally and vertically.
%   The page is neither counted nor numbered.
%   These pages are required in the thesis format before sections such
%   as appendices, vita, bibliography, etc.
%=============================================================================
\def\separatorpage#1{
  \newpage
  \thispagestyle{empty}
  \addtocounter{page}{-1}
  \null
  \vfil\vfil
  \begin{center}
    {\textbf{#1}}
  \end{center}
  \vfil\vfil
  \newpage}


%=============================================================================
% COPYRIGHTPAGE
%=============================================================================
% The copyright must do the following:
% - start a new page with no number
% - place the copyright text centered at the bottom.
%=============================================================================
\def\copyrightpage{
  \newpage
  \thispagestyle{empty}    % No page number
  \addtocounter{page}{-1}
  \chapter*{}            % Required for \vfill to work
  \begin{center}
   \vfill
   \copyright\ Copyright by \@author\ \@date\\
   All Rights Reserved
  \end{center}}


%=============================================================================
% GLOSSARY
%=============================================================================
% The glossary environment must do the following:
% - produce the table of contents entry for the glossary
% - start a new page with GLOSSARY centered two inches from the top
%=============================================================================
\def\glossary{
  \chapter*{GLOSSARY}
  \addcontentsline{toc}{chapter}{Glossary}}
\def\endglossary{\par\newpage}

%=============================================================================
% NOMENCLATURE
%=============================================================================
% The nomenclature environment must do the following:
% - produce the table of contents entry for the nomenclature section
% - start a new page with NOMENCLATURE centered two inches from the top
%=============================================================================
\def\nomenclature{\separatorpage{DISCARD THIS PAGE}
  \chapter*{Nomenclature}
  \addcontentsline{toc}{chapter}{NOMENCLATURE}}
\def\endnomenclature{\par\newpage}

%=============================================================================
% CONVENTIONS
%=============================================================================
% The conventions environment must do the following:
% - produce the table of contents entry for the nomenclature section
% - start a new page with CONVENTIONS centered two inches from the top
%=============================================================================
\def\conventions{\separatorpage{DISCARD THIS PAGE}
  \chapter*{Conventions}
  \addcontentsline{toc}{chapter}{CONVENTIONS}}
\def\endconventions{\par\newpage}


%=============================================================================
% COLOPHON
%=============================================================================
% The colophon environment must do the following:
% - produce the table of contents entry for the nomenclature section
% - start a new page with COLOPHON centered two inches from the top
%=============================================================================
\def\colophon{\separatorpage{DISCARD THIS PAGE}
  \chapter*{Colophon}
  \addcontentsline{toc}{chapter}{Colophon}}
\def\endcolophon{\par\newpage}

%=============================================================================
% LIST OF SYMBOLS
%=============================================================================
% The list of symbols environment must do the following:
% - produce the table of contents entry for the list of symbols section
% - start a new page with LIST OF SYMBOLS centered two inches from the top
%=============================================================================
\def\listofsymbols{\separatorpage{DISCARD THIS PAGE}
  \eject
  \chapter*{LIST OF SYMBOLS}
  \addcontentsline{toc}{chapter}{LIST OF SYMBOLS}}
\def\endlistofsymbols{\par\newpage}

%=============================================================================
% VITA
%=============================================================================
% The vita environment must do the following:
% - produce a separator page with the word vita centered
% - produce the table of contents entry for the vita
% - start a new page with VITA centered two inches from the top
%=============================================================================
\def\vita{
%  \separatorpage{VITA}         % UW doesn't require this EB
  \chapter*{VITA}
  \addcontentsline{toc}{chapter}{VITA}}
\def\endvita{\par\newpage}

%=============================================================================
% ACKNOWLEDGMENTS
%=============================================================================
% The acknowledgments environment must do the following:
% - start a new page with ACKNOWLEDGMENTS centered two inches from the top
%=============================================================================
\def\acks{
  \chapter*{Acknowledgments}
}
\def\endacks{\par\newpage}

%=============================================================================
% DEDICATION
%=============================================================================
% The dedication environment must do the following:
% - start a new page
% - center the text vertically
% - include the text in a center environment
%=============================================================================
\def\dedication{
  \newpage
  \null\vfil
  \begin{center}}
\def\enddedication{\end{center}\par\vfil\newpage}

%=============================================================================
% DATE
%=============================================================================
%\def\today{\ifcase\month\or
  %January\or February\or March\or April\or May\or June\or
  %July\or August\or September\or October\or November\or December\fi
  %\space\number\day, \number\year}
\newcount\@testday
\def\today{\@testday=\day
  \ifnum\@testday>30 \advance\@testday by -30
  \else\ifnum\@testday>20 \advance\@testday by -20
  \fi\fi
  \number\day\ \
  \ifcase\month\or
    January \or February \or March \or April \or May \or June \or
    July \or August \or September \or October \or November \or December
    \fi\ \number\year
}


%  Single counter for theorems and theorem-like environments:
\newtheorem{theorem}{Theorem}[chapter]
\newtheorem{assertion}[theorem]{Assertion}
\newtheorem{claim}[theorem]{Claim}
\newtheorem{conjecture}[theorem]{Conjecture}
\newtheorem{corollary}[theorem]{Corollary}
\newtheorem{definition}[theorem]{Definition}
\newtheorem{example}[theorem]{Example}
\newtheorem{figger}[theorem]{Figure}
\newtheorem{lemma}[theorem]{Lemma}
\newtheorem{prop}[theorem]{Proposition}
\newtheorem{remark}[theorem]{Remark}

%=============================================================================
% TABLE OF CONTENTS; LIST OF FIGURES; LIST OF TABLES
%=============================================================================
% In report style, \tableofcontents, \listoffigures, etc. are always
% set in single-column style.  @restonecol is used to keep track of
% whether we need to switch back to double column style after the toc.
%
% The only known problem now is that the first page with the new
% layout is too long.  The problem seems to be that the change to
% textheight doesn't take place on the first page.  Even if it's the
% first line in the table of contents macro.  Presumably the same
% problem also occurs in the lof and lot.
%
% I'm taking a shot at fixing the problem by dropping in a throw-away
% page between the change to the height parameters and the start of
% the chapter.  Isn't elegance wonderful?
%
%=============================================================================

% \def\@tableof#1#2#3#4#5{
% { % limit scope of following declarations!!
%   \@restonecolfalse\if@twocolumn\@restonecoltrue\onecolumn\fi
%   \addtolength{\textheight}{-40pt}       % -24-16
%   \addtolength{\majorheadskip}{-40pt}    % -24-16
%   \addtolength{\headheight}{52pt}        %  36+16
%   \addtolength{\headsep}{-12pt}          % -12
%   \separatorpage{DISCARD THIS PAGE}
%   \chapter*{#1}
%   #5
%   \relax\markboth{#1}{#1}
%   \hbox to \hsize{#2 \hfil Page}
%   \singlespace
%   \setcounter{#3}{0}
%   \setcounter{listofheading}{1}  % change from 0 to 1 by mccauley, 14may93
%   \def\@oddhead{\vbox to \headheight{\vspace{4pt}
%     \hbox to \hsize{\hfil\textrm{\thepage}} \vfil
%     \ifnum\value{#3}=1
%       \ifnum\value{listofheading}=2
%         \hbox to \hsize{Appendix\hfil} \vspace{4pt} \fi
%       \ifnum\value{listofheading}=1
%         \stepcounter{listofheading} \fi
%       \hbox to \hsize{#2 \hfil Page}
%     \else
%       \setcounter{#3}{1}
%     \fi}}
%   \def\@evenhead{\vbox to \headheight{\vspace{4pt}
%     \hbox to \hsize{\textrm{\thepage}\hfil} \vfil
%     \ifnum\value{#3}=1
%       \ifnum\value{listofheading}=2
%         \hbox to \hsize{Appendix\hfil} \vspace{4pt} \fi
%       \ifnum\value{listofheading}=1
%         \stepcounter{listofheading} \fi
%       \hbox to \hsize{#2 \hfil Page}
%     \else
%       \setcounter{#3}{1}
%     \fi}}
%   \@starttoc{#4}  \if@restonecol\twocolumn\fi
%   \newpage
% }}
 
 %\def\tableofcontents{\@tableof{TABLE OF CONTENTS}{}{tocpage}{toc}{}}
 
 %\def\listoffigures{
 %  \@tableof{LIST OF FIGURES}{Figure}{lofpage}{lof}
 %  {\protect\addcontentsline{toc}{chapter}{LIST OF FIGURES}}}
 
 %\def\listoftables{
 %  \@tableof{LIST OF TABLES}{Table}{lotpage}{lot}
 %  {\protect\addcontentsline{toc}{chapter}{LIST OF TABLES}}}

 %\AtBeginDocument{\renewcommand\contentsname{Table of Contents}

%=============================================================================
% BIBLIOGRAPHY
%=============================================================================
% The thebibliography environment executes the following commands:
%
%  o start a new 'chapter' with BIBLIOGRAPHY as the heading
%  o produce a separator page for the bibliography
%
%  \def\newblock{\hskip .11em plus .33em minus -.07em} --
%      Defines the `closed' format, where the blocks (major units of
%      information) of an entry run together.
%
%  \sloppy  -- Used because it's rather hard to do line breaks in
%      bibliographies,
%
%  \sfcode`\.=1000\relax --
%      Causes a `.' (period) not to produce an end-of-sentence space.
%=============================================================================
% \altbibtitle
%   The default title for the References chapter is ``LIST OF REFERENCES''
%   Since some people prefer ``BIBLIOGRAPHY'', the command
%   \altbibtitle has been added to change the chapter title.
%   This command does nothing more than change REFERENCES to BIBLIOGRAPHY
%============================================================================
%\def\@bibchaptitle{Bibliography}
%\def\altbibtitle{\def\@bibchaptitle{Bibliography}}
%\def\thebibliography#1{
%  \separatorpage{\@bibchaptitle}
%  \global\@bibpresenttrue
%  \chapter*{\@bibchaptitle\markboth{\@bibchaptitle}{\@bibchaptitle}}
%  \addcontentsline{toc}{chapter}{\@bibchaptitle}
%  \vspace{0.375in}    % added to match 4 line requirement
%  \interlinepenalty=10000 % added to prevent breaking of bib entries
%  \singlespace\list
%  {[\arabic{enumi}]}{\settowidth\labelwidth{[#1]}\leftmargin\labelwidth
%    \advance\leftmargin\labelsep \usecounter{enumi}}
%  \def\newblock{\hskip .11em plus .33em minus -.07em}
%  \sloppy
%  \sfcode`\.=1000\relax}
%\let\endthebibliography=\endlist
%\renewcommand{\bibname}{Bibliography}

\makeatother
