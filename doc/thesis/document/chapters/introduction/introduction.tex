\chapter{Introduction}
\label{ch:introduction}
The Monte Carlo method has a rich history going back as far back as Babylonian 
times. However, its use in the field of radiation transport began in the 1940s 
and can be attributed to the work of von Neumann, Ulam, Metropolis, Kahn, Fermi 
and their collaborators \citep{halton_retrospective_1970}. The first successful 
application of the method in the field of radiation transport coincided with 
the construction of the first digital computers \citep{lux_monte_1991}. Because 
computational resources were relatively scarce and expensive, the computer codes
implementing the Monte Carlo method to solve radiation transport problems were
full of approximations to both the physical models and the cross section data.
As the availability of computer resources increased, it became feasible to do 
high fidelity Monte Carlo simulations with regard to both the physical models 
and the cross section data \citep{chucas_preparing_1994}. Today, the Monte 
Carlo method is regarded as the gold standard of computational methods for 
solving radiation transport problems because all variables of interest 
(i.e. energy, direction and position) can be treated on a continuous scale and 
because the problem geometry can be modeled nearly completely. As computers 
continue to grow in size and speed, the Monte Carlo method will continue to be 
used for more and more challenging problems\footnote{Already, the Monte Carlo 
method is appearing in full reactor core simulation codes where it was once 
deemed too costly and inefficient to use \citep{hoogenboom_monte_2011}.}.

\section{The Monte Carlo Method}
\label{sec:monte_carlo_method}
The Monte Carlo method is a stochastic method in which samples are drawn from 
a parent population through sampling procedures governed by a set of 
probability laws. From the samples, statistical data is acquired and analyzed 
to make inferences about the parent population. 

In radiation transport problems, the system of interest is a collection of 
bounded regions which can contain one or more of the following: a material, a 
vacuum, a source, a detector. The parent population is the set of all possible 
radiation histories and the samples are histories drawn from this set. The 
particle history can be regarded as a random walk from a source region to a 
problem domain boundary or some other terminating location (i.e. absorption 
point). Each phase of the random walk is governed by a set of probability laws 
that are related to the material interaction cross sections of the particular 
form of radiation. The portion of a random walk that passes through a finite 
detector region is recorded or scored. 

While radiation transport problems are typically solved by sampling radiation
histories that start in what can be regarded as a model of the physical source
and recorded in what can be regarded as a model of the physical detector, the
opposite can also be true. The process of sampling the starting point of a 
radiation history in the model of the physical source and recording information
in the model of the physical detector is often called a forward process. The 
probability laws used in a forward process can be derived from the Boltzmann 
equation, which will be referred to as the transport equation. The forward 
process is most effective when the detector region is large relative to the 
source region. As the detector region decreases in size, the probability of any 
given history passing through the detector region decreases until, for a point 
detector, the probability goes to zero \citep{spanier_monte_1969}. The process 
of sampling the starting point of a history in the model of a physical detector 
and recording information in the model of the physical source is referred to in 
the literature as an adjoint or reverse process 
\citep{spanier_monte_1969, desorgher_implementation_2010}. The reverse process 
is most effective when the source region is large compared to the detector 
region. When the detector region is a point, only the reverse process can be 
used without resorting to special procedures. The probability laws that govern 
this process can be derived from the adjoint Boltzmann equation, which will be
referred to as the adjoint transport equation. The derivation of these 
probability laws will be a major focus of this report.

\section{Motivations for using the Adjoint Process}
As mentioned in the previous section, one of the primary motivations for using
the adjoint process is that problems with detectors that are small relative to
the source can be solved more efficiently. This motivation for the adjoint
process is encountered often in the literature 
\citep{spanier_monte_1969, bell_nuclear_1979, lewis_computational_1993}. 

While a wide range of shielding problems could benefit from the adjoint process 
based on the above motivation, there is a particular class of problems, which 
are of particular interest in the fusion community, that could benefit. In 
these problems the photon dose in a particular region of an experiment, fusion 
device or fission device, which results from neutron activation of the 
surrounding material, is desired. This information is particularly important 
for planning maintenance on the experiment or device. These problems are often 
solved using a method called the rigorous 2-step (R2S) method 
\citep{pereslavtsev_novel_2013, chen_rigorous_2002, robinson_rigorous_2011}. In 
this method the neutron flux throughout the experiment or device is calculated.
This neutron flux data is then given to an activation code which calculates the 
source of photons due to the decay of material that is activated by the 
neutrons. Finally, the dose due to photons is calculated in the areas of 
interest using a forward process. Usually, the amount of activated material is 
much larger than the region where the dose distribution is desired, which 
indicates that these problems could potentially benefit from the adjoint 
process for photons. The use of the adjoint photon process in the 
rigorous 2-step method has been coined the rigorous 2-step adjoint (R2SA) 
method \citep{robinson_rigorous_2011}. 

Another motivation for using the adjoint process exists. However, discussion
of this motivation requires a physical interpretation of the adjoint flux that
is estimated by using the adjoint process. As explained in the previous section,
during an adjoint process an adjoint random walk starts in the detector and when
it enters the source, information about the problem is gathered. It is therefore
common to interpret the adjoint flux as the importance or sensitivity of a 
particular point of the source (in phase space) to the detector response 
\citep{lewis_computational_1993}. This source importance data can be 
invaluable when the exact source distribution is not known, which
can occur in source distribution optimization calculations. 

In the medical physics community there has been a good deal of work done over 
the last decade in which adjoint data is used for this very reason. In both 
brachytherapy and external beam treatment planning optimization, the use 
adjoint flux data interpreted as a source importance distribution has been 
shown to allow for faster and simpler treatment planning optimization 
algorithms 
\citep{yoo_optimization_2003, chaswal_adjoint_2007, wang_adjoint_2005}. 

The adjoint flux or importance can also be useful when designing
detectors. Essentially, the adjoint flux can allow for the spectral performance 
of a detector to be predicted for an arbitrary source distribution, which 
allows the detector design to be optimized before it is constructed. This can 
be especially important for detectors that utilize rare materials, such as 
$He^3$ detectors \citep{sjoden_deterministic_2002}.

\section{Monte Carlo Codes Available Today}
\label{sec:monte_carlo_codes}
Most Monte Carlo codes available today focus on the forward process described
before. The forward process has been developed to a level where very few 
approximations are used. For instance, it is very common to treat radiation
histories on a continuous energy scale. This is also made possible by the very
accurate cross section data that is available. The adjoint process has not been 
developed to the same level yet. Only a few Monte Carlo codes have implemented
the adjoint process in a way that is relatively free of approximation. The
GEANT4 toolkit has implemented the adjoint process on a continuous energy 
scale for electromagnetic radiation and charged particles. In this implementation there are still some approximations that lead to discrepancies in results 
compared to results from the forward process 
\citep{desorgher_implementation_2010}. FOCUS, a research code written by 
Hoogenboom, was the first code to implement the adjoint process for neutrons
on a continuous energy scale. This code was not able to model the coupled 
adjoint process for neutrons and photons \citep{hoogenboom_adjoint_1977}. Today 
only the commercial United Kingdom code MCBEND has implemented the adjoint
process for neutrons \citep{grimstone_extension_1998}. The implementation in
MCBEND has some approximations that can be eliminated as well. Like FOCUS, 
MCBEND can not model the coupled adjoint process for neutrons and photons. 
Table \ref{table:monte_carlo_codes_today} summarizes the continuous energy 
modeling capabilities of most Monte Carlo codes available today. Please note 
that two of the most powerful and popular codes, MCNP5 and MCBEND are not open 
source codes. GEANT4, though a software development kit and not a true code,
is the only open source software that has some continuous energy adjoint 
capabilities. Several codes, such as MCNP5 and MORSE have implemented the 
adjoint process with a discrete or multigroup energy format.

\begin{table}[ht]
\label{table:monte_carlo_codes_today}
  \caption{\textbf{Continuous Energy Capabilities of Monte Carlo Codes Available
      Today.}
    \textit{The final column shows the proposed capabilities of the 
      Forward-Adjoint Continuous Energy Monte Carlo (FACEMC) code}.}
  \centering
  \begin{tabular}{c c c c c }
    \hline\hline
    Code & $n$ & $\gamma$ &  $n^{\dagger}$ & $\gamma^{\dagger}$ \\ [0.5ex]
    \hline
    EGS4 & - & $\surd$ & - & - \\
    EGSnrc & - & $\surd$ & - & - \\
    ITS6 & - & $\surd$ & - & - \\
    PENELOPE & - & $\surd$ & - & - \\
    MORSE & - & - & - & - \\
    TART2005 & $\surd$ & $\surd$ & - & - \\
    MCNP5/6 & $\surd$ & $\surd$ & - & - \\
    MCNPX & $\surd$ & $\surd$ & - & - \\
    GEANT4 & $\surd$ & $\surd$ & - & $\surd$ \\
    MCBEND & $\surd$ & $\surd$ & $\surd$ & - \\ [1ex]
    \hline
    FACEMC & $\surd$ & $\surd$ & $\surd$ & $\surd$ \\ [1ex]
    \hline
  \end{tabular}
  \label{table:mccodes}
\end{table}

The main reason for the apparent lack of codes that have implemented the 
adjoint process on a continuous energy scale is the lack of available adjoint 
cross section data necessary for the adjoint process. The popular ENDF 
libraries only supply cross section data for the forward process. In addition, 
most of the literature only discusses sampling procedures for the forward 
process based on differential cross sections. Fortunately, Hoogenboom has shown 
that both the total and differential adjoint cross sections can be derived from 
the forward cross sections. The calculation of these cross sections is costly, 
but only needs to be done once and can be done in the popular ENDF format
\citep{hoogenboom_adjoint_1977}. 

\section{The FACEMC Code}
\label{sec:research_outline}
To address the limitations of current Monte Carlo codes and to bring the adjoint
process up to the level of the forward process, the Forward-Adjoint Continuous 
Energy Monte Carlo (FACEMC) code will be developed along with any adjoint
cross sections and sampling techniques that are currently lacking. This code 
will be open source to foster adoption and development by other researchers. As 
mentioned previously, the scope of this code will only encompass fixed source 
problems. The energy range over which the forward and adjoint neutron processes 
will be explored is $10^{-5}$ eV to 20.0 MeV. For the forward and adjoint photon
processes, the energy range that will be explored is 1.0 keV to 20.0 MeV. These 
energy ranges will be sufficient to model a large number of problems important 
to both the nuclear engineering community and the medical physics community. 

\section{Report Outline}
Most of this report will focus on the theory behind the Monte Carlo random 
walk process used to simulate the transport of the types of radiation 
that are of interest. Many texts tend to skip over the detailed derivations
that are shown in the following chapters because the forward process is very
easy to conceptualize. The probability laws that govern the forward process are
also very easy to derive based solely on one's intuition of how radiation
travels through a material. The adjoint process is much more difficult to
conceptualize. To derive the probability laws that govern the adjoint process,
one must rely solely on the adjoint transport equation since one is not likely
to have any intuition for how adjoint radiation travels through a material.
The goal of the following three chapters is therefore to derive the probability 
laws that govern the adjoint process from the adjoint transport equation. 

Once the probability laws that govern the adjoint process have been derived, a
chapter will be devoted to sampling techniques specific to photons and 
adjoint photons. A single chapter will also be devoted to describing the 
FACEMC code. This description will include the proposed code requirements, the 
high-level code design, and the code validation plan. 

