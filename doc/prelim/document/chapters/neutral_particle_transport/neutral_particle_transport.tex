\chapter{Deriving the Monte Carlo Random Walk Process for Neutral Particle Transport Problems}
\label{ch:neutral_particle_transport}
In the previous chapter, the Monte Carlo process was discussed in a very
general mathematical sense. In this chapter, it will be shown how to apply
the general Monte Carlo random walk process to neutral particle transport 
problems. This will require the derivation of the PDFs that govern the random 
walk process from the transport equation. This derivation will be done in
some detail so that each step of the derivation is very clear. The level of
detail will likely seem unnecessary because the PDFs that will be derived
will be very intuitive. However, PDFs for the dual problem, which will be
derived from the adjoint transport equation, will not be intuitive. 
Fortunately, because the adjoint transport equation has a similar form to the
transport equation, the derivation will look almost identical. 

\section{The Transport Equation in Integral Form}
\label{sec:transport_eqn_integral_form}
The equation that describes the average behavior of neutral radiation in a 
medium is the radiation transport equation. Because of its similarity to the 
Boltzmann equation, which describes the kinectic theory of gases, the 
transport equation is often refered to simply as the Boltzmann equation in the 
literature. 

Several assumptions are made in the derivation of the transport 
equation \citep{bell_nuclear_1979}. First, a neutral particle is assumed to be 
a point particle, which means that it can be completely characterized by its 
position and velocity. Second, the medium is assumed to contain a large
enough number of neutral particles that deviations from their expected 
distribution can be ignored. In addition, the number of neutral particles 
present in the medium is not so large that they change the medium during the 
time scale of interest. Finally, neutral particles are assumed to not interact 
with each other. This last assumption will be valid nearly always since even 
the highest density of neutral particles that could potentially be observed 
will still be significantly smaller than the density of the material in which 
the neutral particles travel. The first assumption will not be valid for very 
low energy neutrons and for x-ray photons. However, if polarization and 
interference effects are neglected, which will be the case in this report,
this assumption is still acceptable. 

The transport equation is often given in an integro-differential form, as seen 
in equation \ref{eq:integro_diff_boltzmann_eqn}. In order to use the Monte 
Carlo random walk process that was discussed in the previous chapter, the 
integro-differential form of the transport equation must be converted to a 
FIESK. Once it is in the appropriate form the PDFs that govern the random walk 
process for neutral particles can be determined. Similar derivations to the one 
that will be shown can also be found in several references \citep{lewis_computational_1993, hoogenboom_adjoint_1977, irving_adjoint_1971, bell_nuclear_1979}.
 
\begin{equation}
  \begin{split}
    \frac{1}{v}&\frac{\partial \varphi(\vec{r},E,\hat{\Omega},t)}{\partial t} +
    \hat{\Omega} \cdot \vec{\bigtriangledown} \varphi(\vec{r},E,\hat{\Omega},t)
    + \Sigma_T(\vec{r},E) \varphi(\vec{r},E,\hat{\Omega},t) = \\
    & \quad S(\vec{r},E,\hat{\Omega},t) +
    \int\int \Sigma_T(\vec{r},E^{'} \to E,\hat{\Omega^{'}} \to \hat{\Omega})
    \varphi(\vec{r},E^{'},\hat{\Omega^{'}},t) dE^{'}d\hat{\Omega^{'}} 
  \end{split}
  \label{eq:integro_diff_boltzmann_eqn}
\end{equation}
\begin{eqnarray*}
  \vec{r} & = & \text{spacial coordinate} \nonumber \\
  E & = & \text{particle energy} \nonumber \\
  \hat{\Omega} & = & \text{flight direction} \nonumber \\
  t & = & \text{time} \nonumber \\
  v & = & \text{particle speed} \nonumber \\
\end{eqnarray*}

As mentioned previously, the transport equation only characterizes the average
or expected behaviour of the neutral radiation in a system. Therefore, the
continuous function, $\varphi(\vec{r},E,\hat{\Omega},t)$ in equation 
\ref{eq:integro_diff_boltzmann_eqn} can be interpreted as the expected particle 
flux at $\vec{r}$ and time $t$ for particles with energy $E$ and direction
$\hat{\Omega}$ per unit volume per unit energy per unit solid angle. The 
external particle source density is described by the function 
$S(\vec{r},E,\hat{\Omega},t)$. It can be interpreted as the probability per
unit time that a particle of energy $E$ will appear at $\vec{r}$ per unit 
volume per unit energy per unit solid angle. The function $\Sigma_T(\vec{r},E)$ 
describes the total macroscopic cross section at $\vec{r}$ for particles with 
energy $E$. It can be interpreted as the probability of a particle interaction 
per unit distance of particle travel. In the context of photon transport it is 
often refered to as the attenuation coefficient. The doubly differential 
collision cross section is represented by the function 
$\Sigma_T(\vec{r},E^{'} \to E,\hat{\Omega^{'}} \to \hat{\Omega})$, which can be 
interpreted as the total probability per unit distance per unit energy per unit 
solid angle at $\vec{r}$ for the transfer of a particle with energy $E^{'}$ and 
direction $\hat{\Omega}$ to the energy $E$ and direction $\hat{\Omega}$ as a 
result of a collision. This doubly differential collision cross section can be 
expanded into a macroscopic cross section and a doubly differential transfer 
probability. Also note that the total macroscopic cross section is just the sum 
of all partial macroscopic cross sections.
\begin{eqnarray}
  \Sigma_T(\vec{r},E^{'} \to E,\hat{\Omega^{'}} \to \hat{\Omega}) & = &
  \Sigma_T(\vec{r},E^{'})f(E^{'} \to E,\hat{\Omega^{'}} \to \hat{\Omega}) \\
  & = & \sum_i \Sigma_i(\vec{r},E^{'})
  f_i(E^{'} \to E,\hat{\Omega^{'}} \to \hat{\Omega})
\end{eqnarray}

Because the transport equation only takes into account average behavior, 
reactions that result in the emission of additional particles are handled 
implicitly. The doubly differential transfer probability can only describe the
average outgoing energy and direction distribution for all particles emitted
from the reaction. The doubly differential transfer probability is then 
normalized to the number of particles emitted from the reaction. For absorption
reactions like the (n,$\gamma$) and (n,$\alpha$) neutron reactions, the 
doubly differential transfer probability must be set equal to zero. For 
elastic and inelastic scattering reactions, the transfer probability will be
normalized to unity. For the (n,2n) neutron reaction, the transfer probability
will be normalized to two. For fission reactions, the transfer probability
will be normalized to the average number of neutrons produced by a fission
caused by a neutron of energy $E^{'}$, $\nu(E^{'})$.
\begin{eqnarray}
  &&\int\int f_{e.s.}(E^{'} \to E,\hat{\Omega^{'}} \to \hat{\Omega})dEd\hat{\Omega}
  = 1 \nonumber \\
  &&\int\int f_{i.s.}(E^{'} \to E,\hat{\Omega^{'}} \to \hat{\Omega})dEd\hat{\Omega}
  = 1 \nonumber \\
  &&\int\int f_{(n,2n)}(E^{'} \to E,\hat{\Omega^{'}} \to \hat{\Omega})dEd\hat{\Omega}
  = 2 \nonumber \\
  &&\int\int f_{(n,f)}(E^{'} \to E,\hat{\Omega^{'}} \to \hat{\Omega})dEd\hat{\Omega}
  = \nu(E^{'}) \nonumber
\end{eqnarray}
With all of the partial transfer probabilities normalized in this way, the 
normalization for the total transfer probability can be determined.
\begin{eqnarray}
  \int\int   \Sigma_T(\vec{r},E^{'})
  f(E^{'} \to E,\hat{\Omega^{'}} \to \hat{\Omega}) dEd\hat{\Omega} \nonumber
  & = & \int\int \sum_i \Sigma_i(\vec{r},E^{'})
  f_i(E^{'} \to E,\hat{\Omega^{'}} \to \hat{\Omega}) dEd\hat{\Omega} \nonumber \\
  \int\int f(E^{'} \to E,\hat{\Omega^{'}} \to \hat{\Omega}) dEd\hat{\Omega} 
  & = & \frac{\sum_i c_i\Sigma_i(\vec{r},E^{'})}{\Sigma_T(\vec{r},E^{'})} 
  \nonumber \\
  & = & c(\vec{r},E^{'}) 
\end{eqnarray}
The constant $c_i$ is the number of particles that are emitted from reaction
$i$ and the function $c(\vec{r},E^{'})$ is the mean number of neutrons emerging
per collision at $\vec{r}$ given a particle of energy $E^{'}$.
  
While this averaging and normalizing of the transfer probabilities is necessary
in the context of the transport equation, it is not necessary in a Monte Carlo
random walk process. In simulating the random walks of individual particles, 
one can utilize more detailed transfer probabilities. However, given a suffient
number of random walks, the behavior should be the same as if one were to use
the averaged transfer probabilities present in the transport equation. 

In this report all problems discussed will be assumed to be steady state. 
Therefore, the time dependence of equation \ref{eq:integro_diff_boltzmann_eqn} 
will be ignored from here. To further simplify equation 
\ref{eq:integro_diff_boltzmann_eqn} the right hand side of the equation will be 
refered to simply as the emission density $\chi(\vec{r},E,\hat{\Omega})$.
\begin{equation}
  \begin{split}
    \chi(\vec{r},E,\hat{\Omega}) = S(\vec{r},&E,\hat{\Omega}) + \\
    & \int\int \Sigma_T(\vec{r},E^{'} \to E,\hat{\Omega^{'}} \to \hat{\Omega})
    \varphi(\vec{r},E^{'},\hat{\Omega^{'}}) dE^{'}d\hat{\Omega^{'}}
  \end{split}
  \label{eq:emission_density}
\end{equation}
The reduced transport equation is shown in the following equation.
\begin{equation}
  \hat{\Omega} \cdot \vec{\bigtriangledown} \varphi(\vec{r},E,\hat{\Omega})
  + \Sigma_T(\vec{r},E) \varphi(\vec{r},E,\hat{\Omega}) =  
 \chi(\vec{r},E,\hat{\Omega})
  \label{eq:reduced_transport_eqn}
\end{equation}

From the reduced transport equation, the method of characteristics will be 
used to transform it to its integral form. The characteristic for the transport 
equation is the line defined by the fixed point $\vec{r}$ and the direction 
$\hat{\Omega}$. This line can be parameterized by the variable $R$ resulting in 
the following equation. 
\begin{equation}
  \vec{r^{'}} = \vec{r} - R\hat{\Omega}
  \label{eq:characteristic}
\end{equation}
Using equation \ref{eq:characteristic} a directional derivative along the
characteristic can be determined.
\begin{eqnarray}
  \frac{d}{dR} & = & \frac{dx^{'}}{dR}\frac{\partial}{\partial x} +
  \frac{dy^{'}}{dR}\frac{\partial}{\partial y} +
  \frac{dz^{'}}{dR}\frac{\partial}{\partial z} \nonumber \\
  & = & -\Omega_x \frac{\partial}{\partial x} -
  \Omega_y \frac{\partial}{\partial y} -
  \Omega_z \frac{\partial}{\partial z} \nonumber \\
  & = & -\hat{\Omega} \cdot \vec{\bigtriangledown}
\end{eqnarray}
Using this relationship of the directional derivative along the characteristic
the transport equation can be reduced to a first order ordinary differential 
equation (ODE), which can be easily solved with the integrating factor
$exp\left[-\int_0^R \Sigma_T(\vec{r}-R^{'}\hat{\Omega},E)dR^{'} \right]$.
\begin{equation*}
  -\frac{d}{dR}\varphi(\vec{r^{'}},E,\hat{\Omega}) + \Sigma_T(\vec{r^{'}},E)
  \varphi(\vec{r^{'}},E,\hat{\Omega}) = 
  \chi(\vec{r^{'}},E,\hat{\Omega})
\end{equation*}
\begin{equation*}
  \begin{split}
    -\frac{d}{dR}\bigg[\varphi(\vec{r^{'}},E,\hat{\Omega})
      &exp\left[-\int_0^R \Sigma_T(\vec{r}-R^{'}\hat{\Omega},E)dR^{'}\right]
      \bigg] = \\
    & \chi(\vec{r^{'}},E,\hat{\Omega})
    exp\left[-\int_0^R \Sigma_T(\vec{r}-R^{'}\hat{\Omega},E)dR^{'} \right]
  \end{split}
\end{equation*}
\begin{equation*}
  \begin{split}
    -\varphi(\vec{r} - R\hat{\Omega},&E,\hat{\Omega})
    exp\left[-\int_0^R \Sigma_T(\vec{r}-R^{'}\hat{\Omega},E)dR^{'}\right] 
    \bigg|_0^{\infty} = \\
    & \int_0^{\infty} 
    \chi(\vec{r} - R\hat{\Omega},E,\hat{\Omega})
    exp\left[-\int_0^R \Sigma_T(\vec{r}-R^{'}\hat{\Omega},E)dR^{'} \right] dR
  \end{split}
\end{equation*}

\vspace{0.5cm}
\begin{equation}
    \varphi(\vec{r},E,\hat{\Omega}) = 
    \int_0^{\infty} \chi(\vec{r} - R\hat{\Omega},E,\hat{\Omega})
    exp\left[-\int_0^R \Sigma_T(\vec{r}-R^{'}\hat{\Omega},E)dR^{'} \right] dR
  \label{eq:line_integral_transport_eqn}
\end{equation}
To get to equation \ref{eq:line_integral_transport_eqn} it was assumed that the
flux goes to zero as $R$ goes to infinity. The right hand size of equation 
\ref{eq:line_integral_transport_eqn} can be interpreted as the source of all
particles with energy $E$ along the line $\vec{r}-R\hat{\Omega}$ with direction
$\hat{\Omega}$. This line integral will now be converted to an integral over 
all space. First, note that $R = |\vec{r} - \vec{r^{'}}|$,
$\hat{\Omega} = \frac{\vec{r} - \vec{r^{'}}}{|\vec{r} - \vec{r^{'}}}|$ and
$dV^{'} = R^2dRd\hat{\Omega}$.
\begin{equation}
  \begin{split}
    \varphi(\vec{r},E,\hat{\Omega}) = 
    \int \chi(\vec{r^{'}},E,\hat{\Omega})
    exp\Big[-\int_0^{|\vec{r} - \vec{r^{'}}|} 
      &\Sigma_T(\vec{r}-R^{'}\hat{\Omega},E)dR^{'} \Big] \\
    &\cdot \frac{\delta \left(\Omega - \left[\frac{\vec{r} - \vec{r^{'}}}
        {|\vec{r} - \vec{r^{'}}|}\right]\right)}
    {|\vec{r} - \vec{r^{'}}|^2} dV^{'}
  \end{split}
  \label{eq:volume_integral_transport_eqn}
\end{equation}
With the transport equation now in an integral form, only minor manipulations
are needed in order to get it into a FIESK, after which point the random walk 
process can be determined.

\section{The Flux FIESK}
To get an equation that is only in terms of the flux, equation 
\ref{eq:emission_density} will be substituted back into equation 
\ref{eq:volume_integral_transport_eqn}.
\begin{equation*}
  \begin{split}
    \varphi(\vec{r},E,\hat{\Omega}) = \int &\left[
    S(\vec{r^{'}},E,\hat{\Omega}) + 
    \int\int \Sigma_T(\vec{r^{'}},E^{'} \to E, \hat{\Omega^{'}} \to \hat{\Omega})
    \varphi(\vec{r^{'}},E^{'},\hat{\Omega^{'}})dE^{'}d\hat{\Omega^{'}} \right] \\
    & \cdot exp\left[-\int_0^{|\vec{r} - \vec{r^{'}}|} 
      \Sigma_T(\vec{r}-R^{'}\hat{\Omega},E)dR^{'} \right]
    \cdot \frac{\delta \left(\Omega - \left[\frac{\vec{r} - \vec{r^{'}}}
        {|\vec{r} - \vec{r^{'}}|}\right]\right)} 
    {|\vec{r} - \vec{r^{'}}|^2} dV^{'}
  \end{split}
\end{equation*}

\begin{equation}
  \begin{split}
    \varphi(\vec{r},E,\hat{\Omega}) = & \int S(\vec{r^{'}},E,\hat{\Omega})
    \tau(\vec{r^{'}},\vec{r},E,\hat{\Omega}) dV^{'} + \\
    & \int\int\int \tau(\vec{r^{'}},\vec{r},E,\hat{\Omega})
    \Sigma_T(\vec{r^{'}},E^{'} \to E, \hat{\Omega^{'}} \to \hat{\Omega})
    \varphi(\vec{r^{'}},E^{'},\hat{\Omega^{'}}) dE^{'} d\hat{\Omega^{'}} dV^{'}
  \end{split}
  \label{eq:flux_integral_equation}
\end{equation}
The function $\tau(\vec{r^{'}},\vec{r},E,\hat{\Omega})$ is defined in the following 
equation.
\begin{equation}
  \tau(\vec{r^{'}},\vec{r},E,\hat{\Omega}) = 
  exp\left[-\int_0^{|\vec{r} - \vec{r^{'}}|} 
              \Sigma_T(\vec{r}-R^{'}\hat{\Omega},E)dR^{'} \right]
    \frac{\delta \left(\Omega - \left[\frac{\vec{r} - \vec{r^{'}}}
        {|\vec{r} - \vec{r^{'}}|}\right]\right)} 
    {|\vec{r} - \vec{r^{'}}|^2}
  \label{eq:unnormalized_transport_kernel}
\end{equation}

Equation \ref{eq:flux_integral_equation}, is a FIESK and a random walk process 
can therefore be created to simulate the flux. The source PDF will be created 
first. 
\begin{equation}
  p^1(\vec{r},E,\hat{\Omega}) = \frac{\int S(\vec{r^{'}},E,\hat{\Omega})
    \tau(\vec{r^{'}},\vec{r},E,\hat{\Omega}) dV^{'}}{\int\int S(\vec{r^{'}},E,\hat{\Omega})
    \tau(\vec{r^{'}},\vec{r},E,\hat{\Omega}) dV^{'} dV}
\end{equation}
The state transition PDF can be broken into two parts; one part will govern the
movement of the particle through space and the other part will govern the 
movement of the particle through energy and direction.
\begin{equation*}
  p(\vec{r^{'}} \to \vec{r}, E^{'} \to E, \hat{\Omega^{'}} \to \hat{\Omega}) = 
  p(\vec{r^{'}} \to \vec{r}\quad| E,\hat{\Omega})
  p(E^{'} \to E, \hat{\Omega^{'}} \to \hat{\Omega}\quad|\vec{r^{'}})
\end{equation*}
\begin{eqnarray}
  p(\vec{r^{'}} \to \vec{r}\quad | E,\hat{\Omega}) & = & 
  \frac{\tau(\vec{r^{'}},\vec{r},E,\hat{\Omega})}{w_1(\vec{r^{'}},E,\hat{\Omega})}  \\
  w_1(\vec{r^{'}},E,\hat{\Omega}) & = & \int \tau(\vec{r^{'}},\vec{r},E,\hat{\Omega}) dV \\
  p(E^{'} \to E, \hat{\Omega^{'}} \to \hat{\Omega}\quad|\vec{r^{'}}) & = &
  \frac{\Sigma_T(\vec{r^{'}},E^{'} \to E, \hat{\Omega^{'}} \to \hat{\Omega})}
       {w_2(\vec{r^{'}},E^{'},\hat{\Omega^{'}})} \\
  w_2(\vec{r^{'}},E^{'},\hat{\Omega^{'}}) & = & \int 
  \Sigma_T(\vec{r^{'}},E^{'} \to E, \hat{\Omega^{'}} \to \hat{\Omega}) 
  dE d\hat{\Omega}
\end{eqnarray}

The random walk process to estimate the flux is then completely specified by
the following PDFs. Because the state transition PDF is normalized to unity, 
the probability of a history ending at any given state is zero. To ensure that
random walks actually end, Russian roulette must be used 
\citep{spanier_monte_1969}. 
\begin{equation}
  \text{Flux Random Walk: }
  \begin{cases}
    p^1(\vec{r},E,\hat{\Omega}) & = \frac{\int S(\vec{r^{'}},E,\hat{\Omega})
    \tau(\vec{r^{'}},\vec{r},E,\hat{\Omega}) dV^{'}}{\int\int S(\vec{r^{'}},E,\hat{\Omega})
    \tau(\vec{r^{'}},\vec{r},E,\hat{\Omega}) dV^{'} dV} \\
    p(y \to x) & = p(\vec{r^{'}} \to \vec{r}\quad| E,\hat{\Omega})
    p(E^{'} \to E, \hat{\Omega^{'}} \to \hat{\Omega}\quad|\vec{r^{'}}) \\
    p(x) & = 0
  \end{cases}
  \label{eq:flux_random_walk_pdfs}
\end{equation}

For this random walk process, only the event estimator can be used. In the 
event estimator, the weight of the particle after every collision will be 
multiplied by the factor $w = w_1w_2$. Unfortunately, this factor is not 
bounded to the interval $(0,1)$ and will likely increase the variance of the 
event estimator. Another disadvantage of this random walk process comes about 
from the conditional PDF $p(\vec{r^{'}} \to \vec{r}\quad| E,\hat{\Omega})$. 
Because this PDF does not contain the factor $\Sigma_T(\vec{r},E)$, it is 
possible to sample an event position in a vacuum. This pecuiliar property is 
due to the fact that the flux does not go to zero in a vacuum.

The Monte Carlo random walk process for estimating the flux directly is not 
ideal and in practice is rarely done. Clearly, an event density should have
a preferable random walk process because, intuitively, the event density
goes to zero in a vacuum. 

\section{The Emission Density FIESK}
A FIESK will now be constructed for the emission density. This is accomplished 
by substituting equation \ref{eq:volume_integral_transport_eqn} into equation 
\ref{eq:emission_density}, which defined the emission density. 
\begin{equation*}
  \begin{split}
    \chi(\vec{r},&E,\hat{\Omega}) = S(\vec{r},E,\hat{\Omega}) +
    \int\int \Sigma_T(\vec{r},E^{'} \to E, \hat{\Omega^{'}} \to \hat{\Omega})
    \int\int \chi(\vec{r^{'}},E',\hat{\Omega^{'}}) \\
    & exp\left[-\int_0^{|\vec{r} - \vec{r^{'}}|} 
      \Sigma_T(\vec{r}-R^{'}\hat{\Omega^{'}},E)dR^{'} \right]
    \frac{\delta \left(\Omega^{'} - \left[\frac{\vec{r} - \vec{r^{'}}}
        {|\vec{r} - \vec{r^{'}}|}\right]\right)}
    {|\vec{r} - \vec{r^{'}}|^2} dV^{'}dE^{'}d\hat{\Omega^{'}}
  \end{split}
\end{equation*}

\begin{equation}
  \begin{split}
    \chi(\vec{r},&E,\hat{\Omega}) = S(\vec{r},E,\hat{\Omega}) + \\
    &\int_{\Gamma}C(\vec{r},E^{'} \to E,\hat{\Omega^{'}} \to \hat{\Omega})
    T(\vec{r^{'}} \to \vec{r},E^{'},\hat{\Omega^{'}}) 
    \chi(\vec{r^{'}},E',\hat{\Omega^{'}}) d\Gamma^{'}
  \end{split}
  \label{eq:emission_density_integral_eqn}
\end{equation}

The kernel $T(\vec{r^{'}} \to \vec{r},E,\hat{\Omega})$ is called the transport
kernel, which is defined in the next equation. The transport kernel is simply
the function $\tau(\vec{r^{'}},\vec{r},E,\hat{\Omega})$ defined in equation 
\ref{eq:unnormalized_transport_kernel} multiplied by the total cross section
$\Sigma_T(\vec{r},E)$. This causes the transport kernel to have the desirable
property that events cannot be sampled inside of a vacuum.
\begin{equation}
  \begin{split}
    T(\vec{r^{'}} \to \vec{r},&E,\hat{\Omega}) = \Sigma_T(\vec{r},E) \\
    &\cdot exp\left[-\int_0^{|\vec{r} - \vec{r^{'}}|} 
      \Sigma_T(\vec{r}-R^{'}\hat{\Omega},E)dR^{'} \right] 
    \frac{\delta \left(\Omega - \left[\frac{\vec{r} - \vec{r^{'}}}
        {|\vec{r} - \vec{r^{'}}|}\right]\right)}
    {|\vec{r} - \vec{r^{'}}|^2} 
  \end{split}
\end{equation}
The quantity $T(\vec{r^{'}} \to \vec{r},E,\hat{\Omega})dV$ can be interpreted
as the probability that a particle at $\vec{r^{'}}$ with energy $E$ and 
direction $\hat{\Omega}$ will have its next collision in volume element $dV$
at $\vec{r}$.  

The kernel $C(\vec{r},E^{'} \to E,\hat{\Omega^{'}} \to \hat{\Omega})$ is called
the collision kernel, which is defined in the next equation.
\begin{equation}
  C(\vec{r},E^{'} \to E,\hat{\Omega^{'}} \to \hat{\Omega}) = 
  \frac{\Sigma_T(\vec{r},E^{'} \to E,\hat{\Omega^{'}} \to \hat{\Omega})}
       {\Sigma_T(\vec{r},E^{'})}
\end{equation}

The transition and collision kernels can be combined to create a kernel that
characterizes the transition of a particle from a state 
$y = (\vec{r^{'}},E^{'},\hat{\Omega^{'}})$ of the phase space $\Gamma$ to another 
state $x = (\vec{r},E,\hat{\Omega})$. This new kernel is given in the following 
equation.
\begin{equation}
  K(y \to x) =
  C(\vec{r},E^{'} \to E,\hat{\Omega^{'}} \to \hat{\Omega})
    T(\vec{r^{'}} \to \vec{r},E^{'},\hat{\Omega^{'}})
\end{equation}
The integral equation of the emission density shown in equation 
\ref{eq:emission_density_integral_eqn} can now be represented as
\begin{equation*}
  \chi(x) = S(x) + \int_{\Gamma} K(y \to x)\chi(y)dy,
\end{equation*}
which is a FIESK. 

\section{The Collision Density FIESK}
Before taking a closer look at the kernel $K(y \to x)$, another quantity of 
interest must be discussed: the collision density. It is related to the
flux and the emission density by the following equations.
\begin{eqnarray}
  \psi(\vec{r},E,\hat{\Omega}) & = & \Sigma_T(\vec{r},E)
  \varphi(\vec{r},E,\hat{\Omega}) \\
  & = & \int T(\vec{r^{'}} \to \vec{r},E,\hat{\Omega})
  \chi(\vec{r^{'}},E,\hat{\Omega})dV^{'}
  \label{eq:collision_emission_relation}
\end{eqnarray}
Where as the emission density is the expected density of particles exiting a 
collision or the source, the collision density is the expected density of 
particles entering a collision. Using equation 
\ref{eq:collision_emission_relation} and equation \ref{eq:emission_density}, an 
integral equation for the collision density can be derived:
\begin{equation*}
  \psi(x) = S_c(x) + \int_{\Gamma} L(y \to x)\psi(y)dy.
\end{equation*}
In this equation $S_c(x)$ is the first collided source and $L(y \to x)$ is
a new kernel that also characterizes the transition of a particle from a state 
$y = (\vec{r^{'}},E^{'},\hat{\Omega^{'}})$ of the phase space $\Gamma$ to another 
state $x = (\vec{r},E,\hat{\Omega})$. These functions are defined in the 
following equations.
\begin{equation}
  S_c(x) = \int\int T(\vec{r^{'}} \to \vec{r},E,\hat{\Omega})
  S(\vec{r^{'}},E,\hat{\Omega})dV^{'}
\end{equation}
\begin{equation}
  L(y \to x) =
  T(\vec{r^{'}} \to \vec{r},E,\hat{\Omega})
  C(\vec{r'},E^{'} \to E,\hat{\Omega^{'}} \to \hat{\Omega})
\end{equation}

\section{Emission and Collision Density State Transition Kernel Properties}
Before the PDFs that govern the random walk process for the emission and
collision densities are derived, the transport and collision kernels must be 
investigated a bit further. First, note that for an infinite medium, the 
transport kernel is normalized. 
%\begin{equation}
  
However, if the 
domain of interest in a problem is finite, the transport kernel will no longer
be normalized. Fortunately, this is taken care of by terminating histories 
that exit the domain of interest, which is equivalent to surrounding the domain
of interest with a purely absorbing medium of infinite extent 
\citep{irving_adjoint_1971}.
\begin{equation}
  \int T(\vec{r^{'}} \to \vec{r},E,\hat{\Omega})dV = 1
\end{equation}
The collision kernel can be expanded, which will provided some additional 
information. 
\begin{eqnarray}
  C(\vec{r},E^{'} \to E, \hat{\Omega^{'}} \to \hat{\Omega}) & = &
  \frac{\Sigma_T(\vec{r},E^{'} \to E,\hat{\Omega^{'}} \to \hat{\Omega})}
       {\Sigma_T(\vec{r},E^{'})} \nonumber \\
  \quad & = & \frac{\Sigma_T(\vec{r},E^{'})}{\Sigma_T(\vec{r},E^{'})}
  \frac{\Sigma_T(\vec{r},E^{'} \to E,\hat{\Omega^{'}} \to \hat{\Omega})}
       {\Sigma_T(\vec{r},E^{'})} \nonumber \\
  & = & \frac{\Sigma_T(\vec{r},E^{'})}{\Sigma_T(\vec{r},E^{'})} \sum_j
        \frac{\Sigma_{S,j}(\vec{r},E^{'})}{\Sigma_T(\vec{r},E^{'})}
        f_j(E^{'} \to E,\hat{\Omega^{'}} \to \hat{\Omega}) \nonumber\\
  & = & \frac{\Sigma_T(\vec{r},E^{'})}{\Sigma_T(\vec{r},E^{'})}
        \sum_{A,j} \frac{\Sigma_{S,A}(\vec{r},E^{'})}{\Sigma_T(\vec{r},E^{'})}  
        \frac{\sigma_{j,A}(E^{'})}{\sigma_{S,A}(E^{'})}  \\
        && \qquad \cdot
        f_{j,A}(E^{'} \to E,\hat{\Omega^{'}} \to \hat{\Omega}) \nonumber
\end{eqnarray}
In the expansion of the collision kernel, the subscript A denotes a particular
nuclide and the subscript j denotes a particular type of scattering reaction. 
The function 
$f_{j,A}(E^{'} \to E,\hat{\Omega^{'}} \to \hat{\Omega})$ is a PDF that 
characterizes the outgoing particle energy and direction ($E$,$\hat{\Omega}$) 
given an initial energy and direction ($E$,$\hat{\Omega}$) for scattering 
reaction j. This PDF has the following property:
\begin{equation}
  \int\int f_{j,A}(\vec{r},E^{'} \to E,\hat{\Omega^{'}} \to \hat{\Omega}) 
  dEd\hat{\Omega} = 1
\end{equation}
Therefore, the collision kernel has the following property. The right-hand 
side of equation \ref{eq:collision_op_prop} is often refered to as the 
non-absorption probability ($P_{NA}(E)$).
\begin{eqnarray}
  \int\int C(\vec{r},E^{'} \to E,\hat{\Omega^{'}} \to \hat{\Omega}) 
  dEd\hat{\Omega} & = & \frac{\Sigma_T(\vec{r},E^{'})}{\Sigma_T(\vec{r},E^{'})}
  \label{eq:collision_op_prop} \\
  & = & P_{NA}(E^{'}) \nonumber 
\end{eqnarray}
Because all materials have a non-zero absorption and scattering reaction
at all energies, the non-absorption probability and consequently the 
collision kernel has the following property.
\begin{equation*}
  0 < P_{NA} < 1
\end{equation*}
Now, it is trivial to show that the state transition kernels $K(y \to x)$ and
$L(y \to x)$ have the following properties.
\begin{enumerate}
  \item $K(y \to x) > 0$ \\
    $L(y \to x) > 0$
  \item $\int K(y \to x)dx = P_{NA}(y) < 1 \quad \forall y \in \Gamma$ \\
    $\int L(y \to x)dx = P_{NA}(y) < 1 \quad \forall y \in \Gamma$
\end{enumerate}

With the state transition kernels $K(y \to x)$ and $L(y \to x)$ fully 
characterized, the Monte Carlo random walk process for neutral particle 
transport problems can be completely specified. The following random walk
processes are analogue.
\begin{equation}
  \chi(x)\text{ Random Walk:}
  \begin{cases}
    p^1(x) & = \frac{S(x)}{\int_{\Gamma} S(x)dx} \\
    p(y \to x) &  = K(y \to x) \\
    p(x) & = 1 - P_{NA}(x)
  \end{cases}
  \label{eq:mc_random_walk_emission_dens}
\end{equation}
\begin{equation}
  \psi(x)\text{ Random Walk:}
  \begin{cases}
    p^1(x) & = \frac{S_c(x)}{\int_{\Gamma} S_c(x)dx} \\
    p(y \to x) & = L(y \to x) \\
    p(x) & = 1 - P_{NA}(x)
  \end{cases}
  \label{eq:mc_random_walk_collision_dens}
\end{equation}
To actually simulate the random walks for the emission density, the procedure
outlined in figure \ref{fig:random_walk_process_1} would be used. To
actually simulate the random walks for the collision density, the procedure
outlined in figure \ref{fig:random_walk_process_2} would be used. However,
because the kernels $K(y \to x)$ and $L(y \to x)$ only differ in the ordering
of the collision and transport kernels and because of the relationship between
the emission density and the collision density, both of these quantities can
be estimated during the same random walk process. Figure 
\ref{fig:combined_random_walk_process} illustrates the new combined random
walk process. In this new process, particles always start in the true source
and not the first collided source, which is advantageous because the first 
collided source would be challenging to calculate. Before a collision occurs,
the collision density can be estimated. Once a particle undergoes a scattering
reaction characterized by the collision kernel, the emission density can be
estimated. The transport kernel will then be sampled to determine the new
position of the particle, at which point the collision density can again be
estimated, etc. While this might seem like an interesting property of the 
transport equation, it will actually prove to be quite useful when the value
of an inner product is estimated.

A common modification to the previous analogue random walk process is to 
ignore absorption. This modification results in the following non-analogue
random walk process. As mentioned before, only the event estimator can be used
with this random walk process. At each collision, the particle weight will be
multiplied by $P_{NA}$, which is bounded to the interval $(0,1]$. Therefore,
the weight of a particle will strictly decrease as the number of collisions
increases, which will effect the variance in a positive way. Russian roulette
will also have to be used both as a variance reduction tool and to force random
walks to terminate.
\begin{equation}
  \chi(x)\text{ Random Walk:}
  \begin{cases}
    p^1(x) & = \frac{S(x)}{\int_{\Gamma} S(x)dx} \\
    p(y \to x) &  = \frac{K(y \to x)}{P_{NA}(y)} \\
    p(x) & = 0
  \end{cases}
  \label{eq:mc_random_walk_emission_dens_nonan}
\end{equation}
\begin{equation}
  \psi(x)\text{ Random Walk:}
  \begin{cases}
    p^1(x) & = \frac{\psi_1(x)}{\int_{\Gamma}\psi_1(x)dx} \\
    p(y \to x) & = \frac{L(y \to x)}{P_{NA}(y)} \\
    p(x) & = 0
  \end{cases}
  \label{eq:mc_random_walk_collision_dens_nonan}
\end{equation}

\section{Estimating Responses}
In typical neutral particle transport problems, the inner product of some 
function $a(x)$ and the flux $\varphi(x)$ is desired. If the function $a(x)$ is
a cross section, the inner product that is calculated is often called a 
material response or reaction rate. Because it is challenging to estimate the
flux directly using a Monte Carlo random walk procedure, equivalent inner 
products must be constructed that are either in terms of the collision density 
or the emission density. 
\begin{eqnarray}
  I & = & \int a(\vec{r},E,\hat{\Omega}) \varphi(\vec{r},E,\hat{\Omega}) 
  dVdEd\hat{\Omega} \\
  & = & \int b(\vec{r},E,\hat{\Omega}) \psi(\vec{r},E,\hat{\Omega})  
  dVdEd\hat{\Omega} \\
  & = & \int c(\vec{r},E,\hat{\Omega}) \chi(\vec{r},E,\hat{\Omega}) 
  dVdEd\hat{\Omega}
\end{eqnarray}
Based on the relationship between the collision density and the flux, $b(x)$
must be defined as follows, where $\Sigma_T(x)$ is the total cross section.
\begin{equation}
  b(\vec{r},E,\hat{\Omega}) = \frac{a(\vec{r},E,\hat{\Omega})}
  {\Sigma_T(\vec{r},E)}
  \label{eq:collision_response_function}
\end{equation}
Similarly, the function $c(x)$ must be defined as follows.
\begin{equation}
  c(\vec{r},E,\hat{\Omega}) = \int \frac{a(\vec{r^{'}},E,\hat{\Omega})}
  {\Sigma_T(\vec{r^{'}},E)} T(\vec{r} \to \vec{r^{'}},E,\Omega)dV'
  \label{eq:emission_response_function}
\end{equation}

Clearly, the function $b(x)$ will be easier to evaluate, which presents an 
interesting tradeoff. Estimators will be easier to use if the random walk 
process for the collision density is used. However, the random walk process
for the collision density will be harder to conduct than the random walk 
process for the emission density because of the difficulty in evaluating the 
first collided source. Fortunately, the combined random walk process allows 
one to take advanted of the function $b(x)$ without ever having to evaluate
the first collided source. For this combined random walk process, it turns out 
that the track length estimator actually allows one to estimate the value of 
the flux, so neither $b(x)$ or $c(x)$ ever have to be used.
