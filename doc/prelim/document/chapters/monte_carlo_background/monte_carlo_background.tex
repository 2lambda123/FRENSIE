\chapter{Monte Carlo Methods for Fredholm Integral Equations}
\label{ch:mc_methods}
While Monte Carlo processes can be used to estimate the solutions to a variety 
of quadrature problems, the focus in this report will fall solely on the 
specific Monte Carlo process that is used to estimate the solutions of Fredholm 
integral equations of the second kind (FIESKs). This class of equations is of 
interest because, through a series of manipulations that will be shown in the 
following chapters, both the transport equation and the adjoint transport 
equation become a FIESK. Before discussing the particular Monte Carlo process 
that is used to solve a FIESK, some background into this class of integral 
equations will be given. 

\section{The Fredholm Integral Equations of the Second Kind}
\label{sec:integral_equations}
A FIESK has the following form:
\begin{equation}
  F(x) = S(x) + \lambda \int_a^b K(x,y) F(y)dy.
  \label{eq:fredholm_int_eqn}
\end{equation}
In this equation, the functions $K(x,y)$ and $S(x)$ are known. The 
function $S(x)$ is a forcing function and the function $K(x,y)$ is called the 
kernel of the integral equation, which can be interpreted as a function 
characterizing the transition from some initial state y to the state x. Due to 
this interpretation, the kernel is sometimes written as $K(y \to x)$. This 
notation will be adopted throughout the rest of this report.

Another equation exists that is very similar to the FIESK. It is called the
Volterra Integral equation of the second kind: 
\begin{equation}
  F(x) = S(x) + \lambda \int_a^x K(x,y) F(y) dy.
  \label{eq:volterra_int_eqn}
\end{equation} 
The only difference between the two equations is that the Volterra Integral 
equation of the second kind has a limits of integration that is variable. The 
Volterra integral equation is particularly important for radiation transport 
because it comes about whenever there is a preferred direction for the
independent variable\footnote{Due to conservation of energy and momentum, 
the energy of a particle after scattering off of a massive scattering center 
cannot be greater than the energy prior to the scattering event, assuming that 
the massive scattering center is stationary}. However, the Volterra integral 
equation is just a special case of the FIESK and in fact, all Volterra integral
equations can be written as a FIESK through the use of a modified kernel 
\citep{rahman_integral_2007}:
\begin{equation}
  K^{'}(y \to x) = 
  \begin{cases}
    K(y \to x) & \text{if }y < x \\
    0 & \text{if }y > x.
  \end{cases}
\end{equation}
It is for this reason that only FIESKs will be of interest in the rest of this 
report.

There are many ways to analytically solve a FIESK 
\citep{rahman_integral_2007, morse_methods_1953}. The method of 
successive approximations will be shown because of its relevance to the
Monte Carlo process that will be described in the next sections. To start,
set the zeroth order approximation equal to the forcing function $S(x)$. The
first through n-th order approximations will then be defined as follows:
\begin{align}
  f_0(x) & = S(x) \nonumber \\
  f_1(x) & = S(x) + \lambda \int_a^b K(y \to x)f_0(y)dy \nonumber \\
  f_2(x) & = S(x) + \lambda \int_a^b K(y \to x)f_1(y)dy \nonumber \\
  & \vdots \nonumber \\
  f_n(x) & = S(x) + \lambda \int_a^b K(y \to x)f_{n-1}(y)dy.
  \label{eq:nth_order_fiesk_approx} 
\end{align}
As the order of the approximation goes to $\infty$, the exact solution of
the FIESK is recovered:
\begin{equation}
  F(x) = \lim_{n \to \infty} f_n(x).
\end{equation}

The solution can also be represented as a Neumann series. First, define a new
function $g_n(x)$:
\begin{equation}
  g_n(x) = 
  \begin{cases}
    \quad \int_a^b K(y \to x) S(y)dy & n = 1 \\
    \quad \int_a^b K(y \to x) g_{n-1}(y)dy & n > 1.
  \end{cases}
\end{equation}
Then equation \ref{eq:nth_order_fiesk_approx} can be written in terms of the 
function $g_n(x)$:
\begin{equation}
  f_n(x) = S(x) + \sum_{j=1}^n \lambda^j g_j(x).
\end{equation}
Finally, the exact solution of the FIESK becomes
\begin{equation}
  F(x) = S(x) + \sum_{j=1}^{\infty} \lambda^j g_j(x).
\end{equation}
When the function $g_n(x)$ is expanded, the solution in terms of a Neumann 
series has the following form:
\begin{equation}
  \begin{split}
    F(x) = S(x) &+ \lambda \int_a^b K(y \to x)S(y)dy \\
    & + \lambda^2 \int_a^b \int_a^b K(y \to x)K(y_1 \to y)S(y_1)dy_1dy \\
    & + \lambda^3 \int_a^b \int_a^b \int_a^b K(y \to x)K(y_1 \to y)K(y_2 \to y_1)
    S(y_2)dy_2dy_1dy \\
    & + \cdots 
  \end{split}
  \label{eq:neumann_series_soln_expansion}
\end{equation}

If an integral operator $\hat{K} = \int_{\Gamma} K(y \to x)dy$\footnote{$\hat{K} \cdot f = \int_{\Gamma} K(y \to x)f(y)dy$} 
is defined, then 
the Neumann series will converge if and only if the spectrum of $\hat{K}$ is 
contained on the open unit disk \citep{rahman_integral_2007,morse_methods_1953,spanier_monte_1969}. In other words, successive application of the operator
$\hat{K}$ do not cause the function being operated on to grow in magnitude.

\section{The Monte Carlo Random Walk Process}
\label{sec:mc_random_walk_process}
The Monte Carlo process that is used to estimate the solution of a FIESK
simulates the movement of entities from state to state in some phase space. 
The process of moving some entity from state to state is commonly referred to 
as a random walk processes. The derivation of this Monte Carlo process is 
outside the scope of this report. It will only be proven that this process does 
indeed recover the solution to a FIESK given an appropriate random variable. 
The work by Spanier and Gelbard should be consulted for a detailed discussion 
of the process \cite{spanier_monte_1969}. 

The random walk process for the solution of a Fredholm integral equation of
the second kind is completely specified by the following probability
distribution functions (PDFs) \citep{spanier_monte_1969}: 
\begin{equation}
  \text{Random Walk: }
  \begin{cases}
    p^1(x) & = P(x_1 = x) \\
    p(y \to x) & = P(x_{n+1} = x \text{ | } x_n = y, k > n)  \\
    p(x) & = P(k = n \text{ | } x_n = x).
  \end{cases}
  \label{eq:mc_random_walk_pdfs}
\end{equation}
The variable $x$ represents one point, or state, in continuous phase space 
$\Gamma$. The PDF $p^1(x)$ characterizes the probability that the first event 
of a random walk will occur in state $x$ ($x_1 = x$). The PDF $p(y \to x)$ characterizes the probability of a transition from an initial state $y$ to a new 
state $x$ ($x_{n+1} = x$ given that $x_n = y$). Finally, the probability $p(x)$ 
characterizes the probability of termination in state $x$. The probability
of survival in state $x$ will be represented by $q(x)$. The random walk is 
assumed to undergo $k$ events where $x_1$ is the first event and $x_k$ is the 
final event. The PDFs that define the random walk process have the following 
properties:
\begin{enumerate}
  \item $p^1(x) \geq 0$ \\
  $\int_{\Gamma} p^1(x)dx = 1$ \\ \\
    and
  \item $p(y \to x) \geq 0$ \\
  $\int_{\Gamma} p(y \to x)dx = q(y) = 1 - p(y)$.
\end{enumerate}

The PDF $p^1(x)$ can be generalized to a PDF that characterizes the probability 
of an entity having its $n^{th}$ event in state $x$:
\begin{equation}
  P^n(x) = 
  \begin{cases} 
    P(x_n = x \text{ | } k > n-1) & \text{if } n >= 2 \\
    P(x_1 = x) = p^1(x) & \text{if } n = 1.
  \end{cases}
  \label{eq:pn_pdf}
\end{equation}
To calculate the PDF in equation \ref{eq:pn_pdf}, the following recursion
relation may be used:
\begin{equation}
  P^n(x) = \int_{\Gamma} p(y \to x) P^{n-1}(y)dy
  \label{eq:pn_recursion_rel}
\end{equation}

A discrete random variable $X(x)$ can now be defined on the space of random 
walks, which represents the number density of events that happen in state $x$. 
The expected value of this discrete random variable is
\begin{align}
  E[X(x)] & = 1 \cdot P^1(x) + 1 \cdot P^2(x) + \ldots \nonumber \\
  & = \sum_{n=1}^{\infty} P^n(x).
  \label{eq:expec_coll_dens}
\end{align}

Finally, a continuous random variable $P(x)$ can be defined, which is equal to
$E[X(x)]$. This new random variable is called the continuous event density 
at $x$, or simply the event density\footnote{The collision density is the name 
given in most texts but for the sake of generality, event density has been 
chosen.}. Using this random variable for the event density, and equation 
\ref{eq:pn_recursion_rel}, a proof will be shown that the Monte Carlo random 
walk process outlined in equation \ref{eq:mc_random_walk_pdfs} will indeed 
recover the solution of a Fredholm integral equation, which has the form shown 
in equation \ref{eq:fredholm_int_eqn}. This proof assumes that $S(x)$ and 
$K(y \to x)$ from equation \ref{eq:fredholm_int_eqn} have the same properties 
as $p^1(x)$ and $p(y \to x)$ respectively:
\begin{align}
  P(x) & = \sum_{n=1}^{\infty} P^n(x) \nonumber \\
  & = p^1(x) + \sum_{n=2}^{\infty} \int_{\Gamma} p(y \to x) P^{n-1}(y)dy \nonumber\\
  & = p^1(x) + \int_{\Gamma} p(y \to x) \sum_{n=2}^{\infty} P^{n-1}(y)dy \nonumber\\
  & = p^1(x) + \int_{\Gamma} p(y \to x) \sum_{n=1}^{\infty} P^{n}(y)dy \nonumber\\
  & = p^1(x) + \int_{\Gamma} p(y \to x) P(y)dy \nonumber.
\end{align}

The random variables $X(x)$ and $P(x)$ are purely mathematical contrivances
used to motivate the use of the random walk process and will never be used 
to estimate values of interest in a problem. While one could count the
number of events that occur at a particular point in phase space, given the 
continuous nature of the phase space, the probability of any one event 
occurring at exactly the point of interest is zero. In the next section, more 
practical random variables will be discussed that allow one to estimate 
the solution of a FIESK in some finite portion of the phase space.

To conduct the random walk process that has been outlined, one must first 
sample a starting state $x_1$ from the PDF $p^1(x)$. With probability $p(x_1)$ 
this random walk will end in state $x_1$. If the random walk continues, a new 
state $x_2$ is sampled from
\begin{align}
  p(x \text{ | } x_1) & = \frac{p(x_1 \to x)}{\int p(x_1 \to x)dx} \nonumber \\
  & = \frac{p(x_1 \to x)}{q(x_1)}.
\end{align}
This process will then continue until the random walk eventually ends in some 
state $x_k$. The variable $\alpha = (x_1,\ldots,x_{k-1},x_k)$ will be used to 
represent the random walk. 

Depending on how the random walk process is derived from the model equation (in
the form of a FIESK), the random walk process can fall into one of two 
categories: analogue or non-analogue. In an analogue random walk process, the 
random walks simulate the physical behavior of the entity that would be 
expected based on the model equation. A common misconception is that an 
analogue process is one where the weight of a particle is always unity. For 
multiplying systems where particle multiplication is treated implicitly, an 
analogue process will use particle weights. However, this is rarely done and 
the latter description will usually suffice. Non-analogue random walk processes 
typically employ one or more variance reducing techniques to aid in the 
estimation of rare events. A few common forms of variance reduction are 
importance sampling, Russian roulette and splitting, all of which are 
described in detail by Spanier and Gelbard \cite{spanier_monte_1969}. 

To determine if a random walk process is analogue, one must look at the source 
PDF $p^1(x)$ and the state transition PDF $p(y \to x)$. If the source PDF is 
equal to the source function $S(x)$ divided by a normalization constant and the 
state transition PDF is equal to the state transition kernel $K(y \to x)$ the 
random walk process is analogue (usually).

In terms of radiation transport, analogue random walk processes are 
rarely used due to the difficulty in estimating low probability events. For
example, if an estimate of some quantity on the back side of a thick shield is 
desired and an analogue random walk process is employed, the probability of any 
random walk passing through the shield model will be very small. Acquiring good
estimates of the quantity of interest will therefore require a very large
amount of random walks to compensate for the very small probability of any one
random walk passing through the shield. 

\section{Monte Carlo Inner Product Estimators}
\label{sec:mc_int_eqn_estimators}
As mentioned in the previous section, the continuous nature of the phase space
where the Monte Carlo random walk process is conducted makes the estimation of
quantities at a point very challenging. In most problems where Monte Carlo is 
used, one is instead interested in the estimation of a quantity in some finite 
portion of the phase space. This quantity of interest can be represented by
an inner product of two functions. One function will be a known function and
the other will be the solution to a FIESK, which is unknown. 

As an example, consider a photon shielding problem where the dose in a 
particular region behind the shield is desired. In this problem, the known 
function is the flux-to-dose conversion function and the unknown function is 
the photon flux distribution in the region of interest behind the shield. The 
inner product of the flux-to-dose conversion function and the photon flux 
distribution in the region of interest gives the dose in the region of 
interest. In equation \ref{eq:inner_product}, $g(x)$ is the known function and 
$F(x)$ is the solution to the FIESK of interest:
\begin{equation}
  I = \int_{\Gamma} g(x)F(x)dx.
  \label{eq:inner_product}
\end{equation}

In this section, two random variables, often referred to as estimators, will 
be introduced that can be used to gather information from the random walks to 
estimate the value $I$. These estimators are defined over the entire phase 
space instead of one particular point, which makes them quite useful. 

The first estimator that will be discussed gathers information about every
event of the random walk, but only contributes to the estimation of $I$ when
the random walk terminates. Because of this property, it is called the
termination estimator. To conduct the random walks using the procedure
described in the previous section, the PDFs $p^1(x)$ and $p(y \to x)$ must be 
constructed by normalizing $S(x)$ and $K(y \to x)$ respectively. This random
variable will then reincorporate the magnitude of these functions into
the estimate of $I$. In equation \ref{eq:termination_estimator}, the variable 
$\alpha$ is a random walk beginning at $x_1$ and ending at $x_k$: 
\begin{equation}
  \varepsilon(\alpha) = \frac{S(x_1)}{p^1(x_1)}w(x_1 \to x_2)\cdots 
  w(x_{k-1} \to w_k)\frac{g(x_k)}{p(x_k)}.
  \label{eq:termination_estimator}
\end{equation}
The weight function $w(y \to x)$ is defined in the following equation:
\begin{equation}
  w(y \to x) = 
  \begin{cases}
    \frac{K(y \to x)}{p(y \to x)} & \text{if } p(y \to x) \neq 0 \\
    0 & \text{ otherwise}.
  \end{cases}
\end{equation}
Note that for an analogue random walk process, the weight function after each
event will be unity\footnote{As mentioned in the previous section, the weight
function for an analogue random walk process where multiplication is treated 
implicitly will not generally be unity.}, since $p(y \to x)$ is equal to 
$K(y \to x)$. In addition, $\frac{S(x_1)}{p^1(x_1)}$ will be equal to the 
constant $C = \int_{\Gamma} S(x)dx$. Therefore, for an analogue random walk 
process, the termination estimator reduces to
\begin{equation*}
  \varepsilon(\alpha) = C \frac{g(x_k)}{p(x_k)}.
\end{equation*}

It will now be shown that the expected value of this estimator is equal to the 
value of the inner product, which will prove that this estimator is unbiased 
\citep{spanier_monte_1969}. Equation \ref{eq:neumann_series_soln_expansion},
which shows $F(x)$ as a Neumann series will be useful in verifying this proof:
\begin{align}
  E\left[\varepsilon(\alpha)\right] & = \sum_{\alpha} 
  P(\alpha)\varepsilon(\alpha) \nonumber \\
  & = \sum_{k=1}^{\infty} \int \cdots \int dx_1 \cdots dx_k p^1(x_1)
  p(x_1 \to x_2) \cdots p(x_{k-1} \to x_k)p(x_k) \cdot \nonumber \\
  & \qquad \frac{S(x_1)}{p^1(x_1)}w(x_1 \to x_2)\cdots 
  w(x_{k-1} \to w_k)\frac{g(x_k)}{p(x_k)} \nonumber \\
  & = \sum_{k=1}^{\infty} \int \cdots \int dx_1 \cdots dx_k S(x_1)K(x_1 \to x_2)
  \cdots K(x_{k-1} \to x_k)g(x_k) \nonumber \\
  & = \int_{\Gamma} g(x)F(x)dx \nonumber.
\end{align}

If the function g(x) is only non-zero in some finite portion of the phase space
$\Gamma_d$, which will be the case for most problems, only the random walks 
that terminate in the phase space $\Gamma_d$ will contribute to the 
estimation of $I$. 

Another more efficient estimator exists, which estimates $I$ every time a 
random walk has an event in the phase space $\Gamma_d$. This estimator, defined 
in equation \ref{eq:collision_estimator_2}, is called the event estimator since 
every event can potentially contribute to the estimation of the value of 
interest\footnote{The collision estimator is the name given in most texts but 
for the sake of generality, event estimator has been chosen.}: 
\begin{align}
  \eta(\alpha) & = \sum_{m=1}^k \frac{S(x_1)}{p^1(x_1)}w(x_1 \to x_2) \cdots 
  w(x_{m-1} \to x_m) g(x_m) \nonumber \\
  & = \sum_{m=1}^k W_m(\alpha) g(x_m).
  \label{eq:collision_estimator_2}
\end{align} 
The value $W_m(\alpha)$ can be interpreted as the weight of the entity after 
the $m^{th}$ event of the random walk. 

Again, for an analogue random walk process, the weight function after each 
event will be unity and the value $\frac{S(x_1)}{p^1(x_1)}$ will be equal to 
the constant $C = \int_{\Gamma} S(x)dx$. Therefore, for an analogue random walk 
process, the event estimator reduces to
\begin{equation*}
  \eta(\alpha) = C \sum_{m=1}^k g(x_k).
\end{equation*}

The proof that this estimator is an unbiased random variable is quite lengthy 
and requires a more in-depth discussion of probability theory. Details of this
proof can be found in the work by Spanier and Gelbard \cite{spanier_monte_1969}.

As the size of the phase space $\Gamma_d$ decreases, the event estimator 
will perform slightly better than the termination estimator because each random
walk will have more opportunities to contribute to the estimate of the inner 
product $I$. Note that as the size of the phase space $\Gamma_d$ decreases to a 
point, the function $g(x)$ becomes a delta function and both of the estimators 
become ineffective at estimating the value $I$. In the next section, this 
problem will be addressed specifically without resorting to special estimators\footnote{Kalos has shown how to estimate the flux at a point using a modified
collision estimator \citep{kalos_estimation_1963}}.

When the Monte Carlo random walk process is used to calculate the inner product
$I$, only a finite number of random walks can be simulated and therefore, only 
an estimate of the value $I$ can be obtained. The estimate will
correspond to the following equation, where $N$ is the number of random walks
conducted and $f(\alpha_i)$ is the value of the estimator for random
walk $\alpha_i$:
\begin{equation}
  \bar{I} = \frac{1}{N} \sum_{i=1}^N f(\alpha_i).
  \label{eq:inner_product_estimate}
\end{equation}

To quantify the uncertainty of the estimate of $I$, the sample standard 
deviation must also be calculated. The unbiased sample standard deviation is 
\begin{equation}
  s = \sqrt{\frac{1}{N-1}\sum_{i=1}^N f(\alpha_i)^2 - \frac{N}{N-1}\bar{I}}.
  \label{eq:inner_product_stddev}
\end{equation}

Before moving on, an assumption that was made, which has not been mentioned
until now, must be discussed. In order for any of the above estimators to 
converge to the value $I$, they must be bounded \citep{spanier_monte_1969}. This
means that the random walks must terminate after $k < \infty$ events with 
probability unity. Because the random walk process is in effect estimating the
terms of a Neumann series, the convergence criteria for a Neumann series will
also apply to the random walks. As mentioned in section 
\ref{sec:integral_equations}, the Neumann series will converge if the spectrum
of the operator $\hat{K} = \int_{\Gamma} K(y \to x)dy$ is contained on the open
unit disk, and consequently random walks will terminate with probability unity.
In the context of neutron transport problems, this means that only subcritical 
problems can be handled. This is an acceptable limitation since this report 
will not address criticality problems.

\section{The Dual Fredholm Integral Equation of the Second Kind}
\label{sec:dual_problems}
It has been noted in the previous two sections that estimating a value of 
interest at a point using a Monte Carlo random walk process is challenging.
In this section, a new problem will be introduced that will allow for the 
estimation of a value of interest at a point using the same general
Monte Carlo random walk process from section \ref{sec:mc_random_walk_process}. 

First consider an operator $H$ with the following property when operating
on the function F(x), where x again is a state in the continuous phase space
$\Gamma$: 
\begin{equation}
  H \cdot F(x) = F(x) - \int_{\Gamma} K(y \to x)F(y)dy.
  \label{eq:forward_operator}
\end{equation}
Based on the equation \ref{eq:fredholm_int_eqn}, which describes a FIESK, 
$H \cdot F(x)$ is equal to $S(x)$, the forcing function. If $F(x)$ is assumed 
to be a real-valued function, the adjoint operator $H^{\dagger}$ is defined by 
the following identity \citep{morse_methods_1953}:
\begin{equation}
  \int_{\Gamma}F^{\dagger}(x)H \cdot F(x)dx  = 
  \int_{\Gamma}F(x)H^{\dagger} \cdot F^{\dagger}(x)dx.
  \label{eq:forward_adjoint_ops}
\end{equation}
The function $F^{\dagger}(x)$ is the solution of a different FIESK, which will
be called the dual FIESK. From this identity the adjoint operator 
$H^{\dagger}$ can be determined:
\begin{align}
  \int_{\Gamma} F^{\dagger}(x)H \cdot F(x)dx & = \int_{\Gamma} 
  \left[F^{\dagger}(x)F(x) -
  F^{\dagger}(x)\int_{\Gamma}K(y \to x)F(y)dy \right]dx \nonumber \\
  & = \int_{\Gamma} \left[F^{\dagger}(x)F(x) -
  \int_{\Gamma}K(y \to x)F^{\dagger}(x)F(y)dy \right]dx \nonumber \\
  & = \int_{\Gamma} F(y)F^{\dagger}(y)dy - 
  \int_{\Gamma}F(y)\int_{\Gamma}K(y \to x)F^{\dagger}(x)dxdy \nonumber \\
  & = \int_{\Gamma}F(x)H^{\dagger} \cdot F^{\dagger}(x)dx \nonumber
\end{align}
\begin{equation}
  H^{\dagger} \cdot F^{\dagger}(x) = F^{\dagger}(x) - 
  \int_{\Gamma}K(x \to y)F^{\dagger}(y)dy.
  \label{eq:adjoint_operator}
\end{equation}

If the function $H^{\dagger} \cdot F^{\dagger}(x)$ is set equal to the function
$g(x)$ from the inner product that was defined in equation 
\ref{eq:inner_product}, the dual FIESK can be defined as
\begin{equation}
  F^{\dagger}(x) = g(x) + \int_{\Gamma}K(x \to y)F^{\dagger}(y)dy.
  \label{eq:dual_fredholm_int_eqn}
\end{equation}
The forcing function for the dual FIESK is the function $g(x)$, which means
that random walks for the dual FIESK will start in the region $\Gamma_d$ where
the function $g(x)$ is defined. The kernel of the dual FIESK is the same as 
the kernel for the original FIESK. However, in the dual FIESK the integration 
is over final states, which will cause the behavior of the kernel, and thus
the random walks, to be different.

With the dual FIESK defined in this way, the inner product from equation
\ref{eq:inner_product} can now be defined as
\begin{align}
  I & = \int_{\Gamma} F(x)g(x)dx \nonumber \\
  & = \int_{\Gamma} F(x) H^{\dagger} \cdot F^{\dagger}(x)dx \nonumber \\
  & = \int_{\Gamma}  F^{\dagger}(x) H \cdot F(x) dx \nonumber \\
  I & = \int_{\Gamma} F^{\dagger}(x)S(x) dx.
  \label{eq:inner_product_dual}
\end{align}
As equation \ref{eq:inner_product} and \ref{eq:inner_product_dual} indicate, 
either the solution to the original FIESK or to the dual FIESK can be used to 
determine the value I, which is why the two equation are duals. The question
of which one to use can be answered by analyzing the functions $g(x)$ and
$S(x)$. 

In the limiting case where $g(x)$ is a delta function, which will occur when
the value of I is desired at a point, only a Monte Carlo random walk process for
the dual FIESK, combined with any of the estimators described in the previous 
section, will be able to estimate I. Conversely, in the limiting case where 
$S(x)$ is a delta function, only a Monte Carlo random walk process for the 
original FIESK, combined with any of the estimators described in the previous 
section, will be able to estimate I. This is because delta functions can be 
handled easily as source terms by the Monte Carlo random walk process but are 
very challenging to handle with estimators. 

In general, if the portion of the phase space $\Gamma_s$, in which the source 
is non-zero is larger than the portion of the phase space $\Gamma_d$, in which 
the function $g(x)$ is non-zero, the dual problem will be be able to take 
advantage of estimators with smaller variance, assuming that the dual FIESK has 
similar properties to the original FIESK\footnote{The properties of the dual
FIESK are often much different, as will be shown in chapter 
\ref{ch:adjoint_particle_transport}. In general though, this point is still 
valid since the probability of random walks entering a larger region will be
greater than the probability of random walks entering a smaller region.}. This 
is one of the primary motivations for studying the dual problem in this report.

\section{Chapter Summary}
Several points from this chapter must be emphasized before moving on to the
next chapter. They are summarized in the following list:
\begin{itemize}
  \item The FIESK is an important equation with applications in radiation
    transport problems.
  \item While the FIESK can be solved analytically using the method of 
    successive approximations, it is more common to solve it numerically using
    the Monte Carlo random walk process. This is especially true when the
    system the FIESK describes is complex.
  \item The Monte Carlo random walk process is governed by a set of PDFs that
    are derived from the FIESK.
  \item The solution to any FIESK can be estimated using the Monte Carlo
    random walk process assuming that the PDFs derived from the FIESK have
    certain necessary properties. 
  \item To estimate the solution of a FIESK in some finite portion of phase
    space, the termination estimator and the event estimator can be used.
  \item If an estimation of the solution of a FIESK at a point of the phase 
    space is desired, the Dual FIESK and its associated Monte Carlo random 
    process must be used (unless special estimators are used which will not be 
    discussed in this report).
  \item More generally, if the size of the source region is larger than
    the size of the detector region (the region of interest), the Dual FIESK
    and its associated Monte Carlo random walk process should be used. 
\end{itemize}
