\chapter{Alternative Derivation of the Adjoint Emission and Collision Densities}
\label{ch:appendix_a}

In chapter \ref{ch:adjoint_neutral_particle_transport} a FIESK for the adjoint
of the emission density was created. However, it was pointed out that the 
adjoint of the emission density was ``flux-like'' and therefore not ideal to
estimate via a Monte Carlo random walk process. It was also briefly mentioned
that a better function could be constructed if the adjoint of the emission
density was multiplied by some function $\Sigma^{\dagger}$, whose only necessary
properties are that it is strictly positive and goes to zero in a vacuum. In 
the following sections, two functions will be used to modify the adjoint of
the emission density. In addition, the adjoint of the collision density, which
is also ``flux-like,'' will be modified in a similar way.

\section{The Adjoint Collision Density}
To derive the adjoint collision density, equation 
\ref{eq:adjoint_of_emission_density_integral_eqn_2} will be multiplied by
the total cross section $\Sigma_T(\vec{r},E,)$. In the following equation,
$\xi^{\dagger}(\vec{r},E,\hat{\Omega})$ is the adjoint collision density.
\begin{equation}
  \xi^{\dagger}(\vec{r},E,\hat{\Omega}) = \Sigma_T(\vec{r},E)
  \chi^{\dagger}(\vec{r},E,\hat{\Omega})
\end{equation}

\begin{equation*}
  \begin{split}
  \xi^{\dagger}(\vec{r},&E,\hat{\Omega}) = \int a(\vec{r^{'}},E,\hat{\Omega}) 
    \Sigma_T(\vec{r},E) \tau(\vec{r^{'}},\vec{r},E,-\hat{\Omega}) dV^{'} + \\
    & \int\int\int  \frac{\Sigma_T(\vec{r},E)}{\Sigma_T(\vec{r^{'}},E^{'})}
      \tau(\vec{r^{'}},\vec{r},E,-\hat{\Omega}) 
    \Sigma_S(\vec{r^{'}},E \to E^{'},\hat{\Omega} \to \hat{\Omega^{'}})
    dE^{'}d\hat{\Omega^{'}}dV^{'}
  \end{split}
\end{equation*}
\begin{equation*}
  \begin{split}
  &\text{} \qquad = \int a(\vec{r^{'}},E,\hat{\Omega}) 
    T(\vec{r^{'}} \to \vec{r},E,-\hat{\Omega}) dV^{'} + \\
    & \int\int\int
      T(\vec{r^{'}} \to \vec{r},E,-\hat{\Omega})
      \frac{\Sigma_S(\vec{r^{'}},E \to E^{'},\hat{\Omega} \to \hat{\Omega^{'}})}
           {\Sigma_T(\vec{r^{'}},E^{'})}
    dE^{'}d\hat{\Omega^{'}}dV^{'}
  \end{split}
\end{equation*}

As mentioned in chapter \ref{ch:neutral_particle_transport}, the transport 
kernel is normalized, assuming that random walks that exit the domain of 
interest are terminated. Interestingly, the particles characterized by the 
adjoint collision density FIESK travel in the direction opposite of the 
variable $\hat{\Omega}$, which is indicated by the negative sign in front of
this variable in the transport kernel. Unfortunately, the part of the state
transition kernel that describes the collisions is still poorly behaved. 
However, it can be normalized by introducing the factor 
$P^{\dagger}(\vec{r^{'}},E^{'},\hat{\Omega^{'}})$.

